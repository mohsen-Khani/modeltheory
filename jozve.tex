\documentclass[12pt,a4paper]{report}

\usepackage{amsmath}
\usepackage{amsthm}
\usepackage{amsfonts}
\usepackage{amssymb}
\usepackage{marginnote}
\usepackage{framed}
\usepackage[all]{xy}
\usepackage{amsthm,thmtools,xcolor}
\usepackage{hyperref}
\usepackage{xcolor}
 \usepackage{authblk}
\usepackage{comment} 
 
\newtheoremstyle{colorhead}  % Name
{3pt}                        % Space above
{3pt}                        % Space below
{}                           % Body font
{}                           % Indent amount
{\color{blue}}               % Theorem head font
{:}                          % Punctuation after theorem head
{.5em}                       % Space after theorem head
{}                           % Theorem head spec (can be left empty, meaning "normal")
 
\theoremstyle{colorhead}

\newtheorem{thm}{قضیه} 


%\theoremstyle{definition}


%\newtheorem{thm}{قضیه}[section]
	

\newtheorem{tam}[thm]{تمرین}
\newtheorem{mesal}[thm]{مثال}
\newtheorem{prop}[thm]{گزاره}
\newtheorem{defn}[thm]{تعریف}
\newtheorem{tav}[thm]{توجه}
\newtheorem{nokte}[thm]{نکته}
\newtheorem{yad}[thm]{یادآوری}
\newtheorem{claim}[thm]{ادعای}
\newtheorem{lem}[thm]{لم}

\DeclareMathOperator{\Th}{Th}
\DeclareMathOperator{\diag}{Diag}
\DeclareMathOperator{\qftp}{qftp}
\DeclareMathOperator{\dom}{dom}
\DeclareMathOperator{\range}{range}
\DeclareMathOperator{\var}{var}
\DeclareMathOperator{\chara}{char}
\DeclareMathOperator{\tp}{tp}
\DeclareMathOperator{\Aut}{Aut}
\usepackage{xepersian}
\linespread{1.5}
\settextfont{XB Niloofar}
\setdigitfont{XB Niloofar}
\linespread{1.5}
\title{مباحثی در منطق}
\author{مسعود پورمهدیان، محسن خانی\\
دانشگاه صنعتی امیرکبیر
}

\begin{document}
\sloppy 
\maketitle
\begin{abstract}
هدف نهایی در این دوره‌ی درسی، اثبات «قضیه‌ی جازمیت»، ثابت‌شده توسطِ «مُرْلی» است. بنا به این قضیه، هر تئوریِ الف‌یک‌جازم، در هر کاردینالِ ناشمارای 
$\kappa$،
 جازم است. تئوریِ
 $T$
  را در کاردینالِ 
  $\kappa$
   جازم می‌خوانند، هرگاه همه‌ی مدلهای با اندازه‌ی
  $\kappa$
    از آن، با یکدیگر یکریخت باشند.
درس را با مطالعه‌ی برخی ویژگی‌های جبری در منطق، مانند حذف سور و مدل‌کامل بودن می‌آغازیم. در ادامه به ساختمانهای نظریه‌ی مدلی، مانند فراضربها و ساختارهای سودومتناهی خواهیم پرداخت، و در نهایت قضیه‌ی مُرلی را پس از پرداختن به همه‌ی پیشنیازهای نظریه‌ی مدلی آن، مانند تئوری‌های پایدار، مرتبه‌ی مرلی، و تئوریهای بسیارکمینه، اثبات خواهیم کرد. در بخش دیگری از درس به «حذف موهومیات»‌ خواهیم پرداخت و ثابت خواهیم کرد که در تئوری میدانهای بسته‌ی جبری، موهومیات قابل‌ حذفند. پیشنیازهای این درس، گذرانده بودن درس منطق ریاضی، و آشنایی مقدماتی با نظریه‌ی مدل هستند. علاوه‌ بر جزوه‌ای که مدرس و دستیار برای درس تهیه خواهند کرد، منابع زیر 
را نیز به دانشجویان پیشنهاد می‌کنیم.
\begin{latin}
\begin{itemize}
\item 
    A Course in Model Theory, Katrin Tent, Martin Ziegler, Cambridge University Press.
\item
Model Theory: An Introduction, David Marker, Springer Science and Business Media.
\item  
Model Theory, Chen Chung Chang, H. Jerome Keisler, Dover Books on Mathematics
\end{itemize}
    \end{latin}
\end{abstract}
\tableofcontents
\pagebreak 
\chapter{پیشنیازهای نظریه‌ی مدلی}
\section{جلسه‌ی اول}
\subsection{مقدمات}
هدف از
نظریه‌ی مدل 
\LTRfootnote{model theory}
به عنوان شاخه‌ای
از منطق ریاضی، مطالعه‌ی ساختارهای ریاضیاتی
است با بهره‌گیری از زبان و منطق صوری
\LTRfootnote{formal logic}
 و با بررسی فرمولهایی که در آن ساختارها درستند. ساختارهای ریاضیاتی گاهی جبریند، همچون گروهها،‌فضاهای برداری، مدولها، میدانها و حلقه‌ها، گاهی
 ترکیبیاتی، مانند گرافها،‌ و گاه آنالیزی مانند فضاهای متریک. مطالعه‌ی هر نوع از این ساختارها، انتخاب
\textbf{ زبانی }
مناسب می‌طلبد که در آن بتوان ویژگی‌های این ساختارها را اصلبندی کرد. 
به علاوه نیاز به انتخاب 
\textbf{منطقی}
مناسب و وابسته به زبان است
 که 
قابلیت حمل مفاهیم مورد نیاز را داشته باشد.
\par 
در برخورد نظریه‌ی مدلی با ریاضیات دو رهیافت کلی وجود دارد، که اولی را  رهیافت کلاسیک و دومی را رهیافت نوین 
می‌توان
نامید. مطالعه‌ی بسیاری ساختارهای برآمده از جبر و آنالیز عموماً موضوع رهیافت نخست بوده است. در زیر ایندو را بیشتر واکاویده‌ایم:
\par \noindent
\textbf{رهیافت اول. }
در این رهیافت، ساختارهای معروف و حائزِ اهمیت را در ریاضیات در نظر گرفته زبان مناسبی برای مطالعه‌ی آنها انتخاب و آنها را در این زبان اصل‌بندی می‌کنیم. 
ساختارهایی را که دارای اصلبندی‌های \textbf{کامل}
مناسبی باشند که خوشرفتاری جبری یا آنالیزی آنها را توجیه کند، اصطلاحاً 
\textbf{رام}
\LTRfootnote{tame}
می‌خوانند. 
\footnote{شاید 
گروتندیک نخستین کسی باشد که 
از اصطلاح رام
برای اطلاق به ساختارهای ریاضی استفاده کرده است. 
در مقدمه‌ی کتاب «توپولوژی رام» نوشته‌ی وَن‌دن‌دریز، یا در پایان‌نامه‌ی دکتری نویسنده‌ی دوم، درباره‌ی 
وجوه مختلف
رام بودن یا نبودن ساختارها توضیح داده شده است. 
}
برای نمونه، میدان اعداد مختلط، به عنوان مدلی از تئوری‌ کامل 
\textbf{میدانهای بسته‌ی جبری}
مورد مطالعه قرار می‌گیرد و میدان اعداد حقیقی، به عنوان مدلی از تئوری کامل
\textbf{میدانهای بسته‌ی حقیقی}.
ایندو ساختار را نخستین بار تارسکی اصلبندی و خوشرفتاریهای جبریشان را از دیدگاه نظریه‌ی مدلی توجیه کرده است. روش تارسکی، اثبات 
\textbf{حذف سور}
برای این تئوریها بوده است. 
\par 
برای آشنا کردن بیشتر خواننده با طعم‌وبوی چنین رویکردی، درباره‌ی 
نتیجه‌ی
تارسکی در
میدانهای بسته‌ی حقیقی توضیح کوتاهی می‌دهیم. بنا به قضیه‌ی تارسکی، تئوری میدانهای بسته‌ی حقیقی سورها را حذف می‌کند؛ 
یعنی هر فرمولی را دراین تئوری معادلی بدون سور دارد. یک مصداق آشنای این گفته، در زیر آمده است:
\[
\exists x\quad ax^2+bx+c=0\leftrightarrow b^2-4ac>0.
\]
فرمولهای بدون سور در میدانهای بسته‌ی حقیقی، دقیقاً 
\textbf{مجموعه‌های شبه‌جبری }
را تعریف می‌کنند.
از طرفی سور وجودی در منطق، معادل تصویرگیری در جبر است. از این رو، بیان جبری قضیه‌ی تارسکی، قضیه‌ی زیر است:
\begin{prop}
هر تصویر یک مجموعه‌ی شبه‌جبری، خود مجموعه‌ای است شبه‌جبری.
\end{prop}
بسیاری مفاهیم در هندسه‌ی جبری مختلط و حقیقی با رهیافت اول تحت عنوان ساختارهای رام مطالعه شده‌اند. 
\par 
\noindent
\textbf{رهیافت دوم.}
در این رهیافت، از یک تئوری نظریه‌ی مدلی 
دارای 
ویژگیهای مطلوب،
 آغاز و سعی بر 
\textbf{رسته‌بندی }
\LTRfootnote{categoricity}
مدلهای این تئوری می‌کنیم.
نیز می‌کوشیم تا جایگاه خودِ این تئوری را در 
\textbf{رده‌بندی 
نظریه‌ی مدلیِ 
}
\LTRfootnote{classification}
تئوریها مورد مطالعه قرار می‌دهیم. 
\par 
قضیه‌ی مُرلی، که در این دوره‌ی درسی بدان پرداخته خواهد شد،
قضیه‌ای از  نوع رسته‌بندی است.
\begin{prop}[قضیه‌ی مُرلی]
اگر تئوریِ
$T$
در زبانِ شمارایِ
$L$
یک تئوریِ
$\aleph_1$ ــ
جازم باشد، آنگاه در هر کاردینالِ 
ناشمارایِ
$\kappa$
نیز جازم است. 
\end{prop}
قضیه‌ی بالا سرآغاز نظریه‌ی 
\textbf{پایداری}
\LTRfootnote{stability}
در نظریه‌ی مدل است که توسط شلاه بسط داده شده است. 
ادامه‌ی بسط نظریه‌ی پایداری نیز به 
\textbf{نظریه‌ی رده‌بندی }
\LTRfootnote{classification theory}
انجامیده است که آن نیز حاصل کار شلاه 
\LTRfootnote{Shelah, Shahron,  Hebrew University of Jerusalem}
است.
خواننده را برای آشنایی بیشتر با این رده‌بندی به تارنمای زیر ارجاع می‌دهیم:
\newline
\url{http://forkinganddividing.com/}
\begin{figure}
\caption{رده‌بندی نظریه‌ی مدلی تئوریها}
\includegraphics[scale=.50]{classification.png}
\end{figure}
\subsection{زبان}
\label{vorud}
\begin{defn}
[زبان صوری]
منظور از یک 
\textit{زبان صوریِ }
\LTRfootnote{formal language}
مرتبه‌ی اول
مجموعه‌ای است چون
$L$
متشکل از سه مجموعه‌ی جدا از همِ
$\mathbf{F},\mathbf{R},\mathbf{C}$. 
اطلاعات زیر نیز همواره در چنین زبانی لحاظ می‌شوند.
\begin{enumerate}
\item 
مجموعه‌ی
$\mathbf{F}$
مجموعه‌ی نمادهای تابعی زبان نامیده می‌شود و برای هر نماد تابعیِ
$f\in \mathbf{F}$
یک عدد طبیعیِ 
$n_f$
به نام تعداد متغیرهای نماد تابعیِ 
$f$
در نظر گرفته می‌شود. 
\item 
مجموعه‌ی
$\mathbf{R}$
مجموعه‌ی نمادهای محمولی زبان نامیده می‌شود و برای هر نماد محمولی
$R\in \mathbf{R}$
یک عدد طبیعیِ 
$n_R$
به نام تعداد متغیرهای نماد محمولیِ
$R$
در نظر گرفته می‌شود. 
\item 
مجموعه‌ی
$\mathbf{C}$
مجموعه‌ی ثوابت نامیده می‌شود. 
\end{enumerate}
\end{defn}
\begin{mesal}
\begin{enumerate}
\item 
زبانِ تهی،
$L=\emptyset$.
\item 
$\mathbf{F}=\{(f,2),(g,3)\}$, $\mathbf{R}=\mathbf{C}=\emptyset$.
\item 
$L^1_{\text{گروه}}=\{*,e\}$.
\item 
$L^2_{\text{گروه}}=\{*,e,^{-1}\}$.
\item 
$L_{\text{میدان}}=L_{\text{حلقه}}\{+,\times,0,1\}$.
\item 
زبانهای محمولی مانند
$L_{\text{گراف}}$
متشکل از
$\mathbf{F}=\mathbf{C}=\emptyset, \mathbf{R}=\{(R,2)\}$ 
و 
$L_{\text{ترتیب}}=\{\leq\}$؛
توجه کنید که ایندو در واقع یک زبانند.
\item
زبانهای ترکیبی مانند
$L_{\text{حلقه‌های‌مرتب}}=\{+,\times,0,1,\leq\}$.
\end{enumerate}
\end{mesal}
در مواجهه با یک ساختارِ ریاضیاتی، نظریه‌مدل‌دان در 
انتخاب زبانْ مختار است؛ 
ولی
این انتخاب را زمانی می‌توان معقول دانست  که  زبان یادشده گنجایش اطلاعات ریاضی ساختار مورد مطالعه را داشته باشد، و نیز تئوری‌ای که در این زبان اصلبندی می‌شود، 
با جبر،‌ هندسه، آنالیز یا 
ترکیبات ساختار مورد نظر همسو باشد.
\begin{defn}[ساختار]
گیریم
$L=\mathbf{F}\cup \mathbf{R}\cup \mathbf{C}$
زبانی صوری باشد. منظور از یک
$L$
ــ ساختار، یا یک
$L$
ــ تعبیر،
\LTRfootnote{$L$-structure, $L$-interpretation}
چندتایی‌ای است چون
$\mathfrak{M}$
که از گردهم‌آمدن موارد زیر حاصل شود.
\begin{enumerate}
\item 
یک مجموعه‌ی ناتهی مانندِ
$M$
که بدان 
\textbf{عالمِ سخن}
\LTRfootnote{universe}
یا
\textbf{جهانِ ساخت}
یا
\textbf{دامنه‌ی ساخت}
گفته می‌شود.
\item
به ازای هر
$f\in F$
تابعی چون
$f^\mathfrak{M}:M^{n_f}\to M$،
که بدان تعبیر تابعِ 
$f$
در ساختارِ
$\mathfrak{M}$
گفته می‌شود.
\item 
به ازاء هر
$R\in \mathbf{R}$
رابطه‌ای چون
$R^\mathfrak{M}\subseteq M^{n_R}$
که آن را تعبیرِ رابطه‌ی یادشده در ساختار
$\mathfrak{M}$
می‌خوانیم. (به بیان دیگر،
برای هر رابطه‌ی
$R$،
 تابعی چون
$R^\mathfrak{M}:M^{n_R}\to \{0,1\}$، 
که آن را تابع‌ِ مشخصه‌ی رابطه‌ی یادشده می‌خوانیم).
\item
به ازاءِ هر 
$c\in \mathbf{C}$
عنصری چون
$c^{\mathfrak{M}}\in M$
که بدانْ تعبیر ثابتِ 
$c$
در ساختارِ
$\mathfrak{M}$
می‌گوییم.
\end{enumerate}
نمایش اطلاعات بالا عموماً به صورت زیر است:
\[
\mathfrak{M}=\langle M, \{f^\mathfrak{M}\}_{f\in \mathbf{F}}, \{R^\mathfrak{M}\}_{R\in \mathbf{R}}, \{c^\mathfrak{M}\}_{c\in \mathbf{C}}\rangle.
\]
\end{defn}
ممکن است خواننده دچار این ابهام شود که در فضاهای متریک، تابعِ متریک که از
$M$
به
$\mathbb{R}$
تعریف می‌شود، در تعریف بالا نمی‌تواند مصداق داشته باشد. پاسخ این است که برای مطالعه‌ی چنین فضایی، از زبانها و  ساختارهای چندبخشی 
\LTRfootnote{many-sorted}
استفاده می‌شود، که از پرداختن بدانها فعلاً خودداری می‌کنیم.
\footnote{ 
به
دانشجوی علاقه‌مند پیشنهاد می‌کنیم آنها را با کمک دستیار فراگرفته به کلاس عرضه کند.}
نیز در منطق پیوسته،
که آن نیز در چارچوب این دوره‌ نمی‌گنجد،
 تعبیر هر رابطه‌ی
$R$
تابعی است چون
$R^\mathfrak{M}:M^{n_R}\to [0,1]$.
\begin{mesal}
\hfill 
\begin{enumerate}
\item 
ساختارهای
$\mathfrak{Z}=\langle Z,+,0\rangle$
یا
$\mathfrak{R}=\langle R,\times, 1\rangle$
در زبان
$L=\{*,e\}$.
\item 
ساختارِ
$\mathfrak{R}=\langle \mathbb{R},\leq\rangle$
در زبانِ
$L=\{(R,2)\}$
\item 
ساختارِ
$\mathfrak{Z}=\langle \mathbb{Z},R^\mathfrak{Z}\rangle$
در زبانِ
$L=\{(R,3)\}$، 
آنجا که
\mbox{$R^\mathfrak{Z}=\{(x,y,z)|x\leq y\leq z\}$}.
\end{enumerate}
\end{mesal}
\begin{defn}[نشاندن]
\label{neshandan}
گیریم
$L$
زبانی صوری باشد و
$\mathfrak{M},\mathfrak{N}$
دو ساختار در این زبان.  تابعِ یک‌به‌یکِ
$e:M\to N$
را یک
$L$ ــ
\textit{نشاندن}
\LTRfootnote{$L$-embedding}
 می‌خوانیم هرگاه 
\begin{enumerate}
\item 
برای هر
$c\in \mathbf{C}$
داشته باشیم
$e(c^\mathfrak{M})=c^\mathfrak{N}$.
\item 
برای هر نماد تابعیِ
$n$ 
موضعیِ
$f\in \mathbf{F}$
و
هر
$a_1,\ldots, a_n\in M$
داشته باشیم
\[
e(f^\mathfrak{M}(a_1,\ldots, a_n))=f^\mathfrak{N}(e(a_1),\ldots,e(a_n)).
\]
\item 
برای هر نمادِ‌ محمولیِ
$n$
موضعیِ
$R\in \mathbf{R}$
و هر
$a_1,\ldots,a_n\in M$
داشته باشیم
\[
\langle a_1,\ldots,a_n\rangle\in R^\mathfrak{M} \Leftrightarrow 
\langle e(a_1),\ldots,e(a_n)\rangle\in R^\mathfrak{N}.
\]
\end{enumerate}
\end{defn}
\begin{defn}[زیرساخت]
ساختارِ
$\mathfrak{M}$
را یک
 زیرساخت
 \LTRfootnote{$L$-substructure}
 از ساختارِ
 $\mathfrak{N}$
 می‌خوانیم، و آن را با
 $\mathfrak{M}\subseteq \mathfrak{N}$
 نمایش می‌دهیم، هرگاه اولاً
 $M\subseteq N$
 و ثانیاً تابع شمول،
 $i:M\to N$،
 یک
 $L$ ــ
 نشاندن باشد. 
\end{defn}
معلوم است که هرگاه
$\mathfrak{M}\subseteq \mathfrak{N}$، 
آنگاه برای هر
$f\in \mathbf{F}$
(با
$n$ موضع)
داریم
$f^\mathfrak{N}|_{M^n}=f^\mathfrak{M}$، 
و برای
هر
$R\in \mathbf{R}$
داریم
$R^\mathfrak{N}\cap M^n=R^\mathfrak{M}$.
\begin{mesal}
\hfill 
\begin{enumerate}
\item 
ساختارِ
$\langle \mathbb{Q},\leq\rangle$
زیرساختاری از
$\langle \mathbb{R},\leq \rangle$
است. 
\item
نگاشت
\[
e:\langle \mathbb{R},+,\cdot\rangle\to \langle \mathbb{R}^+,\times,1\rangle
\]
با ضابطه‌ی
$e(x)=e^x$
یک نشاندن است. 
\end{enumerate}
\end{mesal}
\subsection{نحو}
در نحو
\LTRfootnote{syntax}،
هر زبانِ مرتبه‌‌ی اولِ
$L$
را به همراه
نمادهای منطقی 
مرتبه‌ی اول زیر 
در نظر می‌گیریم. 
\begin{enumerate}
\item 
نمادهای موجود در 
مجموعه‌ی 
$L$.
\item 
یک نماد متمایزِ دوموضعی به نام تساوی، که آن را با
$\approx$
نشان می‌دهیم.
\item 
 مجموعه‌ی  متغیرها، که آن را با 
$\var$
نشان می‌دهیم.
\item 
ادوات منطقی، شامل 
\begin{itemize}
\item 
ادوات بولی یا گزاره‌ای:
$\wedge,\vee, \leftarrow,\leftrightarrow, \neg$.
\item 
سورها،
$\forall,\exists$.
\end{itemize}
\item 
پرانتزهای باز و بسته، 
$(,)$
که از آنها صرفاً برای رفع ابهام بهره می‌جوییم.
\end{enumerate}
\begin{defn}[ترمها]
مجموعه‌ای را که از موارد زیر حاصل شود، مجموعه‌ی
$L$ ــ
\textit{ترمها}
\LTRfootnote{term}
 می‌خوانیم.
 \begin{enumerate}
 \item 
 ثوابت و متغیرها جزوِ
 $L$ ــ
 ترمها هستند.
 \item 
 هرگاه 
 $f$
 یک نماد تابعی
 $n$
 موضعی  و 
 $t_1,\ldots,t_n$
 ترمهایی در زبان
 $L$
 باشند،‌ آنگاه
 $f(t_1,\ldots,t_n)$
 نیز
 $L$
 ــ
 ترم است.
 \item 
 $L$ ــ
 ترمها تنها از 
 موارد ۱ و ۲ حاصل می‌شوند.
  \end{enumerate}
\end{defn}
\begin{defn}[فرمولها]
فرمولها در زبانِ 
$L$
به طریق زیر تعریف می‌شوند.
\begin{enumerate}
\item 
$L$ ــ
فرمولهای بسیط (یا اتمیک)
یه یکی از دو صورت زیر هستند:
\begin{enumerate}
\item 
عباراتی چون
$R(t_1,\ldots, t_n)$
که در آن
$R$
رابطه‌ای است
$n$
موضعی و 
$t_1,\ldots,t_n$
ترم‌هایی در زبانِ
$L$
هستند.
\item 
عبارتی چون
$t_1\approx t_2$
که در آن
$t_1,t_2$
ترم هستند.
\end{enumerate}
\item 
اگر
$\phi_1$
و
$\phi_2$
دو
$L$ ــ‌ 
فرمول باشند، آنگاه
$\phi_1\wedge \phi_2$،
$\phi_1\vee \phi_2$،
$\phi_1\to \phi_2$،
$\phi_1,\leftrightarrow\phi_2$،
و
$(\neg\phi_1)$
نیز 
$L$ ــ
فرمولند.
\item 
اگر
$\phi$
یک
$L$ ــ‌
فرمول باشد و 
$x$
یک متغیر، آنگاه
$\forall x \phi$
و
$\exists x\phi$
نیز
$L$
ــ فرمولند.
\item 
$L$ ــ
فرمولها تنها از موارد بالا حاصل می‌شوند.
\end{enumerate}
\end{defn}
به متغیرهایی که در دامنه‌ی هیچ سوری واقع نباشند، متغیرهای آزاد، و به آنهایی که 
تحت سورند، متغیر پایبند می‌گوییم. منظور از نمادِ
$\phi(x_1,\ldots,x_n)$
این است که متغیرهای آزادِ 
فرمولِ
$\phi$
در میانِ
$x_1,\ldots,x_n$
(و نه لزوماً‌ همه‌ی آنها)
هستند. یک متغیر می‌تواند در یک فرمول،‌ حضوری آزاد و حضوری پایبند داشته باشد؛ برای مثال، متغیرِ
$x$
در فرمولِ زیر.
\[
(\exists x\quad R(x,y))\wedge R(x,z).
\]
\subsection{ تئوری صدق تارسکی}
فرض کنید
$\mathfrak{M}$
یک 
$L$ ــ‌
ساختار باشد،
$\phi(x_1,\ldots,x_n)$
یک
$L$
ــ‌ فرمول و
$a_1,\ldots,a_n$
عناصری باشند
در
$M$.
 این را که فرمولِ
 $\phi$
 در مصادیق
 $a_1,\ldots,a_n$
 در ساختارِ
 $\mathfrak{M}$
 صادق است، به صورتِ
 \[
 \mathfrak{M}\models \phi(x_1,\ldots,x_n)[x_1/a_1,\ldots, x_n/a_n]،
 \]
 یا با تسامح،‌ به صورت
 \[
 \mathfrak{M}\models \phi(a_1,\ldots,a_n)
 \]
 نشان داده به صورت استقرایی زیر تعریف می‌کنیم:
\begin{itemize}
\item 
$\mathfrak{M}\models t_1=t_2[\bar{a}]\Leftrightarrow t_1^\mathfrak{M}(\bar{a})=t_2^\mathfrak{M}[\bar{a}]$.
\item 
$\mathfrak{M}\models R(t_1,\ldots,t_n)[\bar{a}]\Leftrightarrow 
R^\mathfrak{M}(t_1^\mathfrak{M}[\bar{a}],\ldots,t_n^\mathfrak{M}[\bar{a}])$.
\item 
$\mathfrak{M}\models \neg \phi[\bar{a}]\Leftrightarrow \mathfrak{M}\not\models
\phi[\bar{a}]$.
\item 
$\mathfrak{M}\models \phi_1[\bar{a}] \text{ و }
\mathfrak{M}\models \phi_2[\bar{a}]
\Leftrightarrow
\mathfrak{M}\models (\phi_1\wedge\phi_2)[\bar{a}]$
\item 
$\mathfrak{M}\models \exists x \quad \phi[\bar{a}]$
اگر عنصری چون
$b\in M$
موجود باشد به طوری که
$\mathfrak{M}\models \phi[\bar{b}]$.
\end{itemize}
\par 
بدیهی است که تعریف بالا، به طور خاص،‌ وقتی که
$\phi$
یک جمله (یعنی فرمولِ بدون متغیر آزاد‌) باشد نیز کارگر است.
در صورتی که
$\mathfrak{M}\models \phi$
گوییم
$M$
مُدِلی برای 
$\phi$
است. نقیض این سخن را با
$\mathfrak{M}\not\models \phi$
نشان می‌دهیم. 
\par 
به یک مجموعه‌ از
$L$ ــ
جملات، 
\textit{تئوری}
\LTRfootnote{theory}
می‌گوییم. تئوریِ
$T$
را 
\textit{ارضاشدنی }
\LTRfootnote{satisfiable}
می‌خوانیم هرگاه مدلی برای آن موجود باشد؛ یعنی ساختاری چون
$\mathfrak{M}$
موجود باشد،‌ به طوری که برای 
هر
$\phi\in T$
داشته باشیم
$\mathfrak{M}\models \phi$.
در این صورت می‌نویسیم
$\mathfrak{M}\models T$.
\subsection{	معادل و نشاندن مقدماتی}
در بخشِ
\ref{vorud}
درباره‌ی نشاندنها و زیرساختها سخن گفتیم. در برخی تئوریها، زیرساختها همه‌ی ویژگی‌های مرتبه‌ی اول یک ساختار را به ارث می‌برند. به هر زیرساختِ اینچنین، زیرساختی مقدماتی می‌گوییم (این مفهوم را در ادامه تعریف کرده‌ایم).
برای مثال، اگر
$M_1,M_2$
دو میدانِ بسته‌ی جبری باشند و 
$M_1\subseteq M_2$،
آنگاه هر چندجمله‌ای‌ای با ضرایبِ در
$M_1$
اگر در
$M_2$
ریشه داشته باشد، مسلماً در
$M_1$
هم ریشه دارد. 
\par 
در این بخش  (در طی چند تمرین) نخست به بررسی این نکته پرداخته‌ایم که زیرساختها چه ويژگی‌هایی از
 ساختار شامل خود 
به ارث می‌برند،‌ و سپس محکی برای وارسی این ارائه می‌کنیم  که 
چه هنگام یک زیرساخت، مقدماتی است.
\begin{tam}
گیریم
$M\subseteq N$؛
نشان دهید که
$\mathfrak{M}\subseteq \mathfrak{N}$
اگروتنهااگر برای هر فرمولِ
بدونِ سورِ
$\phi(x_1,\ldots,x_n)$
و هر
$a_1,\ldots, a_n\in M$
داشته باشیم
\[
\mathfrak{M}\models \phi(a_1,\ldots,a_n)\Leftrightarrow\mathfrak{N}\models \phi(a_1,\ldots,a_n).
\]
\end{tam}
مجموعه‌ی همه‌ی فرمولهای \textbf{بدون سور} با پارامتر در
$M$
را 
(منظور جمله‌هایی است به شکلِ
$\phi(a_1,\ldots,a_n)$
که در آن
$a_1,\ldots,a_n\in M$)
با
$\diag(\mathfrak{M})$
نشان می‌دهیم. مشخص است که
$\diag(\mathfrak{M})$
(دیاگرام اتمیکِ
$\mathfrak{M}$)
را می‌توان به عنوان یک تئوری، ولی در زبانِ
$L_M$ ــ
یعنی زبانی که از افزودن ثابت برای عنصر در
$M$
به
$L$ 
حاصل شده است ـــ
 مورد مطالعه قرار داد.
پس حکم تمرین بالا را می‌توان بدین صورت بازنوشت:
\begin{tam}
نشان دهید که
$\mathfrak{N}\models^{L_M} \diag(\mathfrak{M})$
اگروتنهااگر 
$L$ ــ
نشاندنی از 
$\mathfrak{M}$
در
$\mathfrak{N}$
موجود باشد.
\end{tam}
\par 
به طور مشابه،‌ با
$\diag_{el}(\mathfrak{M})$
(دیاگرام مقدماتیِ
$\mathfrak{M}$)
مجموعه‌ی همه‌ی جمله‌هایی را در زبانِ
$L_M$
نشان می‌دهیم که در
$\mathfrak{M}$
درستند. 
نیز با
$\Th(\mathfrak{M})$
مجموعه‌ی همه‌ی جمله‌هایی را 
\textbf{در زبانِ
$L$}
نشان می‌دهیم که در ساختارِ
$\mathfrak{M}$
درستند. 
\par 
نشاندنِ
$j:\mathfrak{M}\to \mathfrak{N}$
را مقدماتی می‌خوانیم هرگاه همه‌ی فرمولها (و نه فقط فرمولهای بی‌سور) تحت آن حفظ شوند؛ به بیان دیگر هرگاه
برای هر فرمولِ
$\phi(x_1,\ldots,x_n)$
و
$a_1,\ldots,a_n\in M$
داشته باشیم
\[
\mathfrak{M}\models \phi(a_1,\ldots,a_n)\Leftrightarrow\mathfrak{N}\models
\phi(a_1,\ldots,a_n).
\]
هرگاه نگاشت شمول، یک نشاندن مقدماتی باشد، می‌نویسیم
$\mathfrak{M}\prec \mathfrak{N}$، 
و می‌گوییم که
$\mathfrak{M}$
زیرساختی مقدماتی از
$\mathfrak{N}$
است (یا 
$\mathfrak{N}$
توسیعی مقدماتی از
$\mathfrak{M}$
است).
\begin{tam}
نشان دهید که 
$\mathfrak{N}\models^{L_M} \diag_{el}(\mathfrak{M})$
اگروتنهااگر نشاندنی مقدماتی از
$\mathfrak{M}$
در 
$\mathfrak{N}$
در زبانِ
$L$
موجود باشد. 
\end{tam}
نشاندنِ
$e:\mathfrak{M}\to\mathfrak{N}$
را یک 
\textit{ایزومرفیسم }
می‌خوانیم هرگاه یک‌به‌یک و پوشا باشد.
\begin{tam}
نشان دهید که هرگاه
$f:\mathfrak{M}\to \mathfrak{N}$
یک ایزومرفیسم باشد، آنگاه 
$\mathfrak{M}$
و
$\mathfrak{N}$
هم‌ارز مقدماتیند؛ یعنی هر 
$L$ ــ
جمله، در
$\mathfrak{M}$
درست است اگروتنهااگر در
$\mathfrak{N}$
درست باشد (این را با نمادِ
$\mathfrak{M}\equiv \mathfrak{N}$
نشان می‌دهیم).
\end{tam}
توجه کنید که هم‌ارز مقدماتی بودن
$\mathfrak{M}$
و
$\mathfrak{N}$
معادل این است که
$\mathfrak{M}\models \Th(\mathfrak{N})$.
\begin{tam}
آيا عکسِ تمرین بالا درست است؟ نشان دهید هرگاه
$M,N$
هر دو متناهی باشند و
$\mathfrak{M}\equiv \mathfrak{N}$
آنگاه
$M$
و
$N$
ایزومرفند.
\end{tam}
\begin{tam}
تحقیق کنید که
$\langle \mathbb{Z},\leq\rangle \not\prec \langle \mathbb{R},\leq\rangle$.
نیز نشان دهید که
$\langle \mathbb{Q},\leq \rangle\prec \langle \mathbb{R},\leq\rangle$.
(اثبات این دومی با ابزارهایی که هم‌اکنون در درست داریم آسان نمی‌نماید!)
\end{tam}
\begin{tam}[محک تارسکی]
گیریم
$\mathfrak{M}\subset \mathfrak{N}$؛
نشان دهید که
$\mathfrak{M}\prec \mathfrak{N}$
اگروتنهااگر 
برای هر فرمولِ
$\phi(x,\bar{y})$
(دقت کنید که
$x$
تک‌متغیر است و نگفته‌ایم که 
$\phi$
فرمولی بدون سور است)
و هر
$\bar{b}\in M$،
اگر
$\mathfrak{N}\models \exists x \quad \phi(x,\bar{b})$
آنگاه
\[
\mathfrak{N}\models \exists x\in M \quad \phi(x,\bar{b}).
\]
\end{tam}
حتماً خود به زیرکی دریافته‌اید که نوشتن عبارت بالا در منطق مرتبه‌ی اول، حداقل در زبانِ
$L$،
مجاز نیست. چنین جمله‌ای را تنها زمانی می‌توان نوشت که 
در زبان محمولی برای ساختارِ کوچکتر، در اینجا
$\mathfrak{M}$،
داشته باشیم. با این حال هدفمان از آنگونه نوشتن تأکید بر این نکته بوده  که منظورْ 
$\mathfrak{M}\models \exists x \quad \phi(x,\bar{b})$
نبوده است، که در آن صورت محک تارسکی،‌ بدیهی می‌بود!
\begin{tam}
گیریم
$L=\{\leq\}$؛
نشان‌ دهید که
$\Th(\langle \mathbb{Z},\leq\rangle)$
دارای مدلی است که ترتیب اعدادِ گویا در آن می‌نشیند (یعنی 
ساختاری چون
$\mathfrak{M}\equiv \langle \mathbb{Z},\leq\rangle$
به همراه نشاندنی از 
$\langle \mathbb{Q},\leq\rangle$
به
$M$
موجود هستند).
\end{tam}
\begin{tam}

گیریم 
$\mathfrak{M}_0\subseteq \mathfrak{M}_1\subseteq \mathfrak{M}_2$،
$\mathfrak{M}_0\prec \mathfrak{M}_2$
و
$\mathfrak{M}_1\prec \mathfrak{M}_2$.
نشان دهید که
$\mathfrak{M}_0\prec \mathfrak{M}_1$.
\end{tam}
\begin{tam}
گیریم
$f:\mathfrak{M}\to \mathfrak{M}$
یک 
$L$ ــ
نشاندن باشد. نشان دهید 
زوج 
$(\mathfrak{N},g)$
چنان موجود است که
$\mathfrak{M}\subseteq \mathfrak{N}$
و
$g$
اتومرفیسمی از
$\mathfrak{N}$
است که
$f$
را بگستراند.
\end{tam}

\begin{tam}
در زبان گروهها،
برای
$m\not=n$
نشان دهید که
$\mathbb{Z}^m_{p^\infty }\not\equiv \mathbb{Z}^n_{p^\infty }$.
منظور از
$\mathbb{Z}_{p^\infty }$
گروه متشکل از همه‌ی ریشه‌های
$p^k$
اُمِ واحد است.
\end{tam}
\begin{tam}
در زبان گروهها، نشان دهید که برای
$m\not=n$
داریم
$\mathbb{Z}^m\not\equiv\mathbb{Z}^n$.
\end{tam}
\begin{tam}[ضرب مستقیم]
دو
$L$ ــ
ساختارِ
$\mathfrak{M}_1,\mathfrak{M}_2$
را در نظر بگیرید. 
$L$
ــ ساختارِ
$\mathfrak{M}_1\otimes \mathfrak{M}_2$
را طوری تعریف کنید که جهانِ آن
$M_1\times M_2$
باشد و توابعِ طبیعیِ تصویر، یعنی
$\pi_i:\mathfrak{M}_1\times \mathfrak{M}_2\to \mathfrak{M}_i$،
نسبت بدان
ویژگی جهانی زیر را داشته باشند:
\par 
برای هر
$L$ ــ
ساختارِ
$\mathfrak{N}$
و هموفریسمهای
$\phi_i:\mathfrak{N}\to \mathfrak{N}_i$
(برای
$i=1,2$)
همومرفیسمِ یکتای
$\psi:\mathfrak{N}\to \mathfrak{M}_1\otimes \mathfrak{M}_2$
موجود باشد، به طوری که
$\phi_i=\pi_i\circ \psi$.
\end{tam}
تعریف همومرفیسم شبیه تعریف نشاندن است (تعریفِ
\ref{neshandan})،‌
 با این تفاوت که شرط یک‌به‌یک بودن در آن نیاز نیست و 
موردِ ۳ به صورت زیر است: 
\[
\langle a_1,\ldots,a_n\rangle\in R^\mathfrak{M} \Rightarrow 
\langle e(a_1),\ldots,e(a_n)\rangle\in R^\mathfrak{N}.
\]
\pagebreak
\section{ جلسه‌ی دوم، مثالهایی از تئوریها}
\subsection{گروهها }
در زبانِ
$L_1=\{*\}$
اصول زیر را تئوری‌ِ‌ گروهها می‌خوانیم و آن را با
$T_{\text{گروه}}$
نشان می‌دهیم.
\begin{itemize}
\item 
$\sigma_1: \quad \forall x\forall y\forall z \quad \left((x*y)*z=x*(y*z)\right)$
\item $\sigma_2: \quad \exists e \left( \forall x\quad  x*e=e*x=x\quad \wedge\quad  \forall x
 \exists y \quad x*y=e\right)$
\end{itemize}
به آسانی می‌توان دید که
$\langle G,*\rangle \models T_{\text{گروه}}$
اگروتنهااگر
$\langle G,*\rangle$
یک گروه باشد. نیز داریم
\[
T_{\text{گروه}}\models \text{«عنصر خنثی یکتاست.»}
\]
\[
نیز
T_{\text{گروه}}\models \text{«وارون هر عنصر یکتاست»}
\]
\begin{tam}
بررسی کنید که هرگاه
$\langle G,*\rangle \models T_{\text{گروه}}$،
هر زیرساخت از
$\langle G,*\rangle$
یک شبه‌گروه
\LTRfootnote{semigroup}
است.
\footnote{منظور از شبه‌گروه ساختاری است به همراه یک عملِ شرکت‌پذیر.}
\end{tam}
در تمرینِ بالا مشاهده کردیم که در زبانِ
$L_1$،
گروه‌بودن (یعنی مدلی از
$T_{\text{گروه}}$
بودن)
از
$G$
به زیرساختهای آن به ارث نمی‌رسد. درباره‌ی این که در یک تئوریِ داده‌شده،‌ 
زیرساختها چقدر به ساختارهای شامل خود شبیهند، در ادامه بیشتر خواهیم گفت.
\par 
گروهها را می‌توان در زبانهای مجهزتری نیز اصلبندی کرد. زبانهای
$L_2=\{*,e\}$
و
\mbox{$L_3=\{*,^{-1},e\}$}
را در نظر بگیرید. در زبانِ
$L_2$
می‌توان تئوریِ
$T^2_{\text{گروه}}$
را از اجتماع دو اصل زیر با اصلِ
$\sigma_1$
حاصل کرد.
\begin{itemize}
\item 
$\theta_1:\quad \forall x \quad x*e=e*x=x$.
\item 
$\theta_2:\quad \forall x\exists y\quad x*y=y*x=e$.
\end{itemize}
\begin{tam}
بررسی کنید هر زیرساخت از یک مدل از
$T^2_{\text{گروه}}$
یک تکواره
\LTRfootnote{monoid}
است.
\footnote{منظور شبه‌گروهی است دارای عنصر خنثی.}
\end{tam}
در زبانِ
$L_3$
نیز می‌توان تئوریِ
$T^3_{\text{}}$
را اجتماع اصولِ
$\sigma_1,\theta_1$
و
اصل زیر در نظر گرفت.
\begin{itemize}
\item $\theta_3:\quad \forall x\quad x*x^{-1}=x^{-1}*x=e$.
\end{itemize}
\begin{tam}
بررسی کنید که هر زیرساخت از یک مدل از 
$T^3_{\text{گروه}}$
خودْ مدلی از
$T^3_{\text{گروه}}$
(یعنی یک گروه) است.
\end{tam}
\subsection{حلقه‌ها و میدانها}
برای حلقه‌ها نیز می‌توان زبانهای مختلفی برگزید. 
ساختارِ
$\langle R,+,\cdot,-,0,1\rangle$
را
در زبانِ
\mbox{$L^1_{\text{}}=\{+,\cdot,0,1\}$}
مدلی از تئوریِ حلقه‌ها،‌ یا به طور کوتاه،
حلقه‌ 
می‌خوانیم هرگاه از اصول زیر پیروی کند (که مجموعه‌ی آنها را 
$T_{\text{حلقه}}$
می‌خوانیم).
\begin{itemize}
\item 
$\langle R,+,0\rangle \models T^2_{\text{گروه}}\cup \{\forall x\forall y \quad x+y=y+x\}$
\item 
$\forall x\forall y\forall z\quad \left(x\cdot(y+z)=x\cdot y+x\cdot z\right)$.
\end{itemize}
\begin{tam}
زبانِ
$L^2_{\text{حلقه}}=\{+,-,\cdot,0,1\}$
را برای حلقه‌ها در نظر گرفته نقش اصول مربوط به
$\{-\}$
را در شناسایی مدلها و زیرساختارهاشان بررسی کنید. برای مثال بررسی کنید که در این زبان، هر زیرساخت از یک مدلِ دلخواه، حوزه‌ای صحیح است.
\end{tam}
بدین‌ترتیب تئوری حلقه‌های جابجایی،
$T_{\text{ح‌جابجایی}}$
 از اجتماع
$T_{\text{حلقه}}$
با
تک‌اصلِ
\mbox{$\{\forall x\forall y\quad x\cdot y=y\cdot x\}$}
حاصل می‌شود؛ تئوریِ
$T_{\text{ح‌صحیح}}$،
برای حوزه‌های صحیح، از اجتماعِ
$T_{\text{ح‌جابجایی}}$
با تک‌اصلِ
\mbox{
$\{\forall x\forall y\quad x\cdot y=0\to x=0\vee y=0\}$}،
و تئوریِ میدانها،
$T_{\text{میدان}}$
از اجتماعِ
$T_{\text{ح‌جابجایی}}$
با اصلِ
زیر:
\[
\forall x \quad (x\not=0\to \exists y\quad x\cdot y=y\cdot x=1).
\]
\subsubsection{میدانهای با مشخصه‌ی معین}
دو میدانِ
$\mathbb{Z}_p$
و
$\mathbb{C}$
(به همراه عملهای جمع  و ضرب و صفر و  یکشان)  مدلهایی 
ناهم‌ارز‌مقدماتیند
از
$T_{\text{میدان}}$؛
در اولی جمله‌ی زیر برقرار است ولی در دومی خیر.
\[
\forall x\quad \underbrace{x+x+\ldots x}_{\text{ $p$ بار}}=0.
\]
«مشخصه‌ی میدانها» میدانها را بر اساس این تفاوت متمایز می‌کند.
بنا به تعریف، مشخصه‌ی یک میدانِ
$F$
عددی اول چون 
$p$
است، هرگاه
\[
\forall x\quad \underbrace{x+x+\ldots x}_{\text{ $p$ بار}}=0.
\]
نیز گوییم مشخصه‌ی میدانِ
$F$
صفر است هرگاه چنین عدد اول
$p$
موجود نباشد. بدین‌ترتیب می‌توان با افزودن جمله‌هایی به تئوری‌ میدانها، مشخصه‌ی آنها را نیز تعیین کرد.
\begin{tam}
میدانی نامتناهی با مشخصه‌ی عدد اولِ
$p$
بیابید.
\end{tam}
«متناهی بودن یا نبودن» چشم‌ِاسفندیار نظریه‌ی مدل است؛ در درسهای بعد دانش مدل‌تئوریک کافی را
برای تجربه‌کردنِ این سخن فراهم خواهیم آورد. 
برای حال، به تمرین زیر بسنده کرده‌ایم.
\begin{tam}
آیا می‌توان یک تئوریِ
$T$
نوشت، به طوری که
$\langle F,+,\cdot,0,1\rangle\models T$
اگروتنهااگر
$F$
میدانی با مشخصه‌ی متناهی باشد؟
\end{tam}
پیش از آن که بیشتر درباره‌ی تئوریهای مربوط به میدانها بگوییم، در بخش بعد اندکی درباره‌ی جبر میدانها یادآوری می‌کنیم. بخش بعد تنها یادآور برخی مفاهیم جبری است و خواندن آن اختیاری. 
\subsubsection{انحراف از بحث، جبر میدانها}
فرض کنید
$F$
یک میدان با مشخصه‌ی 
$p<\infty$
باشد. آنگاه اشتراک همه‌ی زیرمیدانهای آن، 
که آن را 
با
$P$
نشان می‌دهیم،
میدانی ایزومرف با
$\mathbb{Z}_p$
است. اگر
مشخصه‌ی
$F$
صفر باشد، آنگاه
$P\cong \mathbb{Q}$.
میدانِ
$P$
را زیرمیدانِ اولیه‌ی
$F$
می‌خوانیم. 
اگر
$F$
میدانی متناهی باشد، آنگاه
عددی اول چون
$p$
و عددی چون
$n\in \mathbb{N}$
موجودند، به طوری که
$|F|=p^n$ 
و 
$\chara(F)=p$.
گروه ضربی متشکل از همه‌ی عناصرِ
ناصفر یک میدانِ متناهیِ
$F$
گروهی دوری است. نیز چنین میدانِ
$F$
توسیعی ساده از زیرمیدانِ
اولیه‌ی خود،
$\mathbb{Z}_p$
است؛ یعنی
$F=\mathbb{Z}_p(u)$
برای یک
$u\in F$.
\par 
اگر
$F$
میدانی با مشخصه‌ی
$p$
باشد، آنگاه 
برای هر
$r\geq 0$
نگاشتِ
$\phi:F\to F$
که هر 
$x$
را به 
$x^{p^r}$
می‌برد، یک
$\mathbb{Z}_p$
مونومرفیسم است. در صورتی که
$F$
متناهی باشد، این نگاشت یک
$\mathbb{Z}_p$
اتومرفیسم است. 
\begin{prop}
میدانِ
متناهیِ
$F$
دارای 
$p^n$
عضو است اگروتنهااگر میدانِ شکافنده‌ی 
(تعریف در زیر)
چندجمله‌ایِ
$x^{p^n}-x$
روی
$\mathbb{Z}_p$
باشد. برای هر عدد اول
$p$
و هر
$n\in \mathbb{N}$
میدانی (یکتا به پیمانه‌ی ایزومرفیسم) از اندازه‌ی
$p^n$
موجود است.
اگر
$K$
یک توسیع میدانی از
$F$
با بُعد متناهی باشد، آنگاه
$K$
متناهی و گالوا (تعریف در زیر) روی
$F$
است. نیز گروه گالوای
 $\Aut_F^K$
 (تعریف در زیر)
دوری است. 
\end{prop}
فرض کنید
$K\subseteq F$
یک توسیع میدانی باشد. عنصرِ
$u\in F$
را روی
$K$
جبری می‌خوانیم هرگاه ریشه‌ای از یک چندجمله‌ای مانند
$f\in K[x]$
باشد؛ در غیر این صورت،
$u$
را روی
$K$
متعالی می‌خوانیم. اگر
همه‌ی عناصرِ
$F$
روی
$K$
جبری باشند، 
$F$
را توسیعی جبری از
$K$،
و در غیر این صورت آن را توسیعی متعالی از
$K$
می‌خوانیم. 
\par 
اگر
$u\in F$
روی
$K$
جبری باشد، آنگاه
$K(u)$،
میدان تولیدشده توسط
$K,u$
در
$F$،
ایزومرف با
$K(x)$،
میدان متشکل از توابعِ گویا است. اگر
$u$
روی
$K$
جبری باشد آنگاه
\begin{enumerate}
\item 
$K(u)=k[u]$.
\item 
$K(u)\cong K[x]/(f)$
که در آن
$f$
چندجمله‌ایِ 
کمینالِ
$u$
و فرضاً دارای  درجه‌ی
$n$
است.
\item 
$[K(u):K]=n$.
\item 
$K(u)$
به عنوان یک فضای برداری روی
$K$
توسط عناصرِ
$\{1,u,u^2,\ldots,u^{n-1}\}$
تولید می‌شود. 
\end{enumerate}
میدانِ
$F$
را یک توسیع گالوایی از
$K$
می‌خوانیم هرگاه
\[
K=\{x\in F|\forall f\in \Aut_K^F \quad f(x)=x\}.
\]
منظور از
$\Aut^F_K$
مجموعه‌ی همه‌ی اتومرفیسمهایی از
$F$
است که
$K$
را نقطه‌وار حفظ می‌کنند.
\begin{prop}[قضیه‌ی بنیادین نظریه‌ی گالوا]
اگر
$F$
یک توسیع گالواییِ با بعدِ متناهی از
$K$
باشد، آنگاه میان دو مجموعه‌ی زیر تناظری یک‌به‌یک است:
\begin{itemize}
\item 
همه‌ی میدانهای $H$ که
$K\subseteq H\subseteq F$.
\item 
همه‌ی زیرگروه‌های گروهِ
$\Aut^F_K$.
\end{itemize}
تناظرِ یادشده، نگاشتی است که هر
میدانِ
$E$
را به
$\Aut^F_E$
می‌فرستد و ویژگی‌های زیر را داراست:
\begin{itemize}
\item 
برای هر دومیدانِ
$K\subseteq H_1\subseteq H_2\subseteq F$
داریم
$[H_2:H_1]=[\Aut^F_{H_1}:\Aut^F_{H_2}]$.
بنابراین مرتبه‌ی
گروه
$\Aut^F_K$
برابر است با
$[F:K]$.
\item 
$F$
روی هر میدانِ
$K\subseteq H\subseteq F$
گالواست؛ ولی
$H$
روی
$K$
وقتی و تنها وقتی گالواست که 
$\Aut^F_H$
زیرگروهی نرمال از
$\Aut^F_K$
باشد. در این صورت داریم
$\Aut^H_K\cong \Aut^F_K/\Aut^F_H$.
\end{itemize}
\end{prop}
گوییم چندجمله‌ایِ
$f\in F[x]$
روی میدانِ
$F$
شکافته می‌شود
\LTRfootnote{splits}
هرگاه آن را بتوان به صورتِ
$u_0(x-u_0)(x-u_2)\ldots (x-u_n)$
نوشت که در آن
$u_i\in F$.
به میدانِ
$F(u_1,\ldots, u_n)$
میدانِ شکافنده‌ی 
چندجمله‌ایِ
$f$
روی میدانِ
$F$
می‌گوییم. برای هر
$f\in F[x]$
میدان شکافنده‌ای چون
$F_1$
موجود است به طوری که
$[F_1:F]\leq n!$.
\par 
میدانِ
$F$
را بسته‌ی جبری می‌خوانیم هرگاه هر چند جمله‌ایِ
$f\in F[x]$
در
$F$
ریشه داشته باشد (یعنی
$F$
میدان شکافنده‌ی همه‌ی چندجمله‌ایهای
$f\in F[x]$
باشد).  هر میدانِ
$K$
دارای یک بستارِ جبری است؛ یعنی میدانی بسته‌ی جبری
چون
$K\subseteq F$
موجود است که $F$
روی
$K$
جبری است  (معادلاً
$F$
میدان شکافنده‌ی همه‌ی چندجمله‌های تحویل‌ناپذیر در
$K[x]$
است).
\par 
اگر
$F$
روی
$K$
جبری باشد، آنگاه
$|F|\leq \aleph_0|K|$.
هر دو بستارِ جبری از یک میدانِ 
$K$
با هم
 $K$
 ــ
ایزومرفند. 
\par 
چندجمله‌ایِ تحویل‌ناپذیرِ
$f\in K[x]$
را جدایی‌پذیر
\LTRfootnote{separable}
خوانیم هرگاه در یک میدانِ شکافنده‌ی
$f$
روی
$K$
همه‌ی ریشه‌های 
این چندجمله‌ای، ساده باشند. عنصرِ جبریِ
$u\in F-K$
را جدایی‌پذیر خوانیم هرگاه چندجمله‌ای کمینال آن جدایی‌پذیر باشد. توسیعِ
$F$
از
$K$
را جدایی‌پذیر خوانیم هرگاه همه‌ی عناصر آن روی
$K$
جدایی‌پذیر باشند. موارد زیر با هم معادلند:
\begin{itemize}
\item 
$F$
توسیعی جبری و گالوا از
$K$
است.
\item 
$F$
میدان شکافنده‌ی مجموعه‌ای از چندجمله‌ای‌های جداشدنی در
$K[x]$
است.
\item 
$F$
میدان شکافنده‌ی یک مجموعه از چندجمله‌ها در
$K[x]$
است و 
$F$
روی
$K$
جداشدنی است.
\end{itemize}
توسیعِ جبریِ
$F$
از
$K$
را نُرمال می‌خوانیم هرگاه هر چندجمله‌ایِ
$f\in K[x]$
که ریشه‌ای در
$F$
دارد، همه‌ی ریشه‌هایش در
$F$
باشند (معادلاً هرگاه
$F$
روی
$K$
جبری و میدان شکافنده‌ی مجموعه‌ای از چندجمله‌ها در
$K[x]$
باشد).
\begin{prop}
توسیعِ جبری
$F$
از
$K$
گالواست اگروتنهااگر
نرمال  و جدایی‌پذیر باشد. اگر مشخصه‌ی
$F$
صفر باشد، آنگاه
$F$
روی
$K$
گالواست اگروتنهااگر نرمال باشد.
\end{prop}
گیریم
$E$
توسیعی جبری از
$K$
باشد. میدانِ
$F$
را بستارِنرمال
\LTRfootnote{normal closure}
میدان
$E$
روی
$K$
می‌خوانیم هرگاه شرایط زیر برآورده شود:
\begin{itemize}
\item 
$F$
روی
$K$
نرمال باشد.
\item
هیچ زیرمیدانِ سره از
$F$
شاملِ
$E$، 
روی
$K$
نرمال نباشد.
\item
اگر
$E$
روی
$K$
جدایی‌پذیر باشد،
$F$
روی
$K$
گالوا باشد.
\item
$[F:K]$
متناهی باشد اگروتنهااگر
$[E:K]$
متناهی باشد.
\end{itemize}
\begin{prop}[قضیه‌ی اساسیِ  جبر]
\label{asasijabr}
میدانِ اعداد مختلط، بسته‌ی جبری است (و هر بستار جبریِ اعداد حقیقی با آن ایزومرف است).
\end{prop}
گفتیم که هر میدانی دارای بستارِ جبری است. بستارِ جبریِ‌
میدانِ
$\mathbb{F}_p$
میدانِ
$\bigcup_{n\in\mathbb{N}} \mathbb{F}_{p^n}$
است (این میدان نیز دارای مشخصه‌ی
$p$
است).
\subsubsection{میدانهای بسته‌ی جبری}
تئوریِ میدانهای بسته‌ی جبری، که آن را با
ACF
نشان می‌دهیم از اجتماعِ
$T_{\text{میدان}}$
با طرح‌اصول زیر حاصل می‌شود:
\[
\theta_n: \quad \forall a_1\ldots a_n \quad \exists x \quad x^n+a_1x^{n-1}+
\ldots +a_n=0.
\]
همانگونه که از توضیحِ پس از گزاره‌ی
\ref{asasijabr}
معلوم است، می‌توان میدانهای بسته‌ی جبری‌ای با مشخصه‌ی ناصفر داشت؛ بنابراین دو مدل از
ACF
لزوماً هم‌ارز مقدماتی نیستند. به بیانِ بهترِ
ACF
کامل 
\LTRfootnote{complete}
نیست.
\par 
تئوریِ
$T$
را کامل می‌خوانیم هرگاه
همه‌ی مدلهای آن  هم‌ارز مقدماتی باشند. به بیان دیگر هرگاه 
$T$
سازگار باشد و 
برای هر جمله‌ی 
$\theta$
یا
$T\models \theta$
یا
$T\models \neg \theta$.
\par 
با
$ACF_p$
و
$ACF_0$
به ترتیب تئوری میدانهای بسته‌ی جبری با مشخصه‌ی 
$p$ 
و با مشخصه‌ی صفر را نشان می‌دهیم.
\begin{prop}[تارسکی]
$ACF_0$
و
$ACF_p$
تئوریهایی کامل هستند.
\end{prop}
\begin{tam}
فرمولهای بدون سور (بدون‌پارامتر، و با پارامتر) را در ساختارِ
$\langle F,+,\cdot,0,1\rangle$
شناسایی کنید.
\end{tam}
\subsection{مدولها}
گیریم
$R$
حلقه‌ای یکدار و نه لزوماً جابجایی باشد. گوییم
$M$
یک
$R$ ــ
مدول است، هرگاه
\begin{enumerate}
\item 
$M$
گروهی باشد آبلی،
\item 
نگاشتی
$R\times M\to M$ 
موجود باشد، 
با ضابطه‌ی
$(r,m)\mapsto rm$
به طوری که
\begin{enumerate}
\item 
$r(m+n)=rm+rn$.
\item 
$(r+s)m=rm+sm$.
\item 
$r(sm)=(rs)m$.
\end{enumerate}
\end{enumerate}
\begin{mesal}
\hfill
\begin{enumerate}
\item 
اگر
$K$
یک میدان باشد، هر 
$K$ ــ
مدول یک فضای برداری است.
\item 
$\mathbb{Z}$ ــ
مدولها دقیقاً همان گروههای آبلی هستند:
$na=a+\ldots +a$
و
$(-n)a=-(na)$.)
\item 
$\mathbb{Q}$ ــ
مدولها، دقیقاً همان گروههای آبلیِ بدون تاب هستند.
\item
$\mathbb{F}_p$ ــ
مدولها،‌ گروههای آبلیِ 
دارای مرتبه‌ی 
$p$
هستند ($p$ عددی اول است).
\end{enumerate}
\end{mesal}
زبانِ
$L_{mod}(R)=\{0,+,-,r\}_{r\in R}$
را در نظر بگیرید. هر
$R$ ــ
مدول یک
$L_{mod}(R)$ ــ
ساختار است که در آن، تابعِ 
$r$
به صورتِ
$r(x)=rx$
تعبیر شده است (توجه کنید که در این زبان نمی‌توان روی عناصرِ‌موجود در
$R$
سور بست). در زبانِ یادشده،
$R$ ــ
مدولها تشکیل کلاسی مقدماتی می‌دهند، با
اصولی شاملِ
خانواده‌ی زیر از اصول (دریافتن سایر اصول این تئوری بر عهده‌ی خواننده است)
\[
\{\forall x \quad (rs)x=r(s(x))\}_{r,s\in R}
\]
\begin{tam}
فرمولهای بدون سور را در زبان مدولها بررسی کنید.
\end{tam}
\subsection{مجموعه‌های مرتب}
مجموعه‌های مرتب را در زبانِ
$L_{\text{ترتیب}}=\{\leq\}$
مطالعه می‌کنیم. تئوری مجموعه‌های جزئاًمرتب 
با اصول زیر اصلبندی می‌شود:
\begin{itemize}
\item 
$\forall x \quad x\leq x$.
\item 
$\forall xy \quad (x\leq y\wedge y\leq x\to x=y)$.
\item 
$\forall xyz \quad (x\leq y\wedge y\leq z\to z\leq z)$.
\end{itemize}
اصل زیر، بیان «خطی» بودن ترتیب است:
\begin{itemize}
\item 
$\forall x\forall y \quad x\leq y\vee y\leq x$.
\end{itemize}
اصل زیر بیانگر چگال بودن ترتیب است:
\begin{itemize}
\item 
$\forall xy\quad (x<y\to \exists z \quad x<z<y)$.
\end{itemize}
\begin{tam}
تئوریِ مجموعه‌های مرتب خطی گسسته بدون ابتدا و انتها را اصل‌بندی کنید.
\end{tam}
\subsection{گروههای مرتب خطی}
تئوری گروههای مرتب خطی از افزدون دو اصل زیر به اصول تئوری گروهها حاصل می‌شود:
\begin{itemize}
\item $\forall xyz \quad (x+y<x+z\to y<z)$.
\item $\forall xyz \quad (y+x<z+x\to y<z)$.
\end{itemize}
در این تئوری می‌توان ثابت کرد که گروههای مرتب خطی، بدون تاب هستند.
به طور مشابه می‌توان تئوری میدانهای مرتب را نوشت که از آن نتیجه می‌شود که این میدانها مشخصه‌ی صفر دارند. 
\subsection{	انحراف از بحث، جبرِ میدانهای بسته‌ی حقیقی}
میدانِ
$K$
را حقیقی می‌خوانند هرگاه در آن
$-1$
را نتوان به صورت مجموعی از مربعات نوشت. برای مثال، هر میدان مرتب، حقیقی است.
نیز، هر میدان حقیقی، ترتیب‌پذیر است. 
 اگر
$R$
میدانی حقیقی باشد که هر توسیع جبری سره‌اش غیرحقیقی شود
(یعنی هیچ توسیع‌ جبری‌ سره‌ای نداشته باشد که حقیقی باشد)
به
$R$
یک میدان بسته‌ی حقیقی می‌گوییم. میدان اعداد حقیقی 
(بنا به قضیه‌ی اساسی جبر)
 یک چنین میدانی است.
همان طور که می‌دانیم، تنها فاصله‌ی میدان اعداد حقیقی، تا بسته‌ی جبری شدن، 
$\sqrt{-1}$
است. همین، برای همه‌ی میدانهای بسته‌ی حقیقی برقرار است.
\begin{prop}
میدانِ حقیقیِ
$R$،
بسته‌ی حقیقی است اگروتنهااگر
$R(i)$
یک میدان بسته‌ی جبری باشد (این گفته در واقع نتیجه‌ای از قضیه‌ی اساسی جبر است).
\end{prop}
\par
هر میدان حقیقی دارای یک بستار حقیقی است؛‌ یعنی اگر
$K$
حقیقی باشد، میدانی چون
$K\subseteq R$
چنان موجود است که 
$R$
توسیعی جبری از
$K$
و خودْ بسته‌ی حقیقی است. 
\par 
هر میدان بسته‌ی حقیقی دارای ترتیبی یکتاست.  
\begin{prop}
یک میدانِ حقیقی، بسته‌ی حقیقی است اگروتنهااگر ویژگیِ مقدار میانی داشته باشد؛ یعنی هر چندجمله‌ایِ
$p$
که در 
$b$
مقداری مثبت دارد و در
$a<b$
مقداری منفی، در نقطه‌ای در بازه‌ی
$(a,b)$
صفر شود.
\end{prop}
گزاره‌ی زیر نیز محک دیگری برای بسته‌ی حقیقی بودن یک میدان ارائه می‌کند.
\begin{prop}
میدان
$R$
بسته‌ی حقیقی است اگروتنهااگر برای
هر
$a\in R$
یکی از
$a,-a$
دارای مجذور، و هر چند جمله‌ای از درجه‌ی فرد دارای ریشه باشد.
\end{prop}
\subsection{میدانهای بسته‌ی حقیقی}
گفتیم میدانهای بسته‌ی حقیقی، بنا تعریف میدانهایی هستند که در آنها
$-1$
مجموعی از مربعات نیست 
ولی در هر توسیع جبریشان،
$-1$
مجموعی از مربعات است. این گفته را نمی‌توان به صورت مرتبه‌ی اول نوشت. با این حال، در بخش قبل تعاریف معادلی برای این میدانها عرضه کردیم که قابل
بیان در زبان مرتبه‌ی اولند.
\par 
تئوری میدانهای بسته‌ی حقیقی، از اجتماعِ تئوری میدانهای مرتب با اصول زیر حاصل می‌شود (مورد سوم، طرح‌اصل است).
\begin{itemize}
\item 
$\forall x\quad (x>0\to \exists y \quad y^2=x)$.
\item 
$\forall xy \quad x^2+y^2=0\to x=y=0$.
\item 
$\sigma_n:\quad \forall \alpha_1\ldots \alpha_n \quad \forall a<b
\quad (a^n+\alpha_1a^{n-1}+\ldots+\alpha_n>0 \quad \wedge \quad
b^n+\alpha_1b^{n-1}+\ldots+\alpha_n<0\to 
\exists a<x<b\quad x^n+\alpha_1x^{n-1}+\ldots +\alpha_n=0)$
\end{itemize}
تئوری یادشده را با
$RCF$
نمایش می‌دهیم.
\begin{prop}[تارسکی]
$RCF$
کامل و دارای حذف سور و از این رو تصمیمپذیر است.
\end{prop}
\subsection{انحراف از بحث، حساب در نظریه‌ی مدل}
تا اينجا چند مثال از تئوریها ديده‌ايم. برخي از اين مثالها تئوریهای كامل هستند، از اين حيث كه درستی يا نادرستی هر جمله‌ی داده شده از اصول آنها قابل استنتاج است. به درست آوردن يك تئوری كامل چندان دشوار نيست. براي هر
$L$ ـ‌ـ
ساختارِ
$\mathfrak{M}$
تئوریِ
$\Th(\mathfrak{m}=\{phi| \mathfrak{M}\models \phi\}$ 
يك تئوری كامل است كه آن را تئوری كامل ساختار
$\mathfrak{M}$
می‌خوانیم.
\par 
 با اين حال وجود یک اصلبندی مناسب براي تئوریِ
  كامل يك ساختار همواره مطلوب محقق نظریه‌ی مدل است. 
  برای مثال ديديم كه تئوریِ
$ACF$
كه معادل است با تئوری كامل ساختار اعداد مختلط
دارای يك اصلبندیِ كامل  محاسبه‌پذير است.
 سوال اين است كه آيا ممکن است یک تئوری كامل مناسب برای كل دنیای رياضيات نوشت. اين اصلبندی بايد آنقدر جامع باشد كه دو روي متفاوت رياضيات یعنی هندسه و حساب را دربربگيرد؛ درست همانگونه كه چند اصل ساده‌ی هندسه‌ی اقليدسی براي اين هندسه مكفی است. 
\par
هيلبرت را شايد بتوان مهمترين رياضيدان قائل به امکان  ارائه‌ی دستگاهی از اصول براي ریاضیات دانست. 
او در سخنرانی تاريخيش در کونیکسبرگ
در سال 
۱۹۳۰
تأکید 
كرده بود كه هيچ سؤال بی‌پاسخی 
در ریاضیات
باقی نخواهند ماند
و به زودي رياضيات دارای دستگاهی كامل از اصول خواهد شد كه درستی یا نادرستی هر جمله‌ای
از آن اصول نتيجه شود. درست بودن اين گفته معادل امكان ساخت رايانه‌ای است كه اصولی اوليه  را به دست گيرد و خود همه‌ی رياضيات را توليد كند.  در همان سال و گويا چندی بعد،‌ در همان فراهمايی، گودل سایر رياضيدانان را از قضيه‌ی ناتماميت  خود آگاه كرده بود. بنا به قضيه‌ی گودل برای هر اصلبندی مناسبی كه برای حساب در نظر بگيريم جمله‌ای هست كه نه از این اصلبندی  اثبات و نه با کمک آن رد می‌شود. اثبات گودل، 
از نوع
اثباتهای خود‌بازگشت بود؛
يعني اثباتی با ایده‌های  مشابه به تناقضات دروغگو يا راسل. 
\par
از قضيه تماميت گودل نتيجه می‌شود كه عاری بودن يا نبودن رياضيات از تناقض قابل اثبات نيست. در واقع اصول زرملو ــ فرانكل كه امروز به عنوان دستگاه کارآمدی براي مبانی رياضيات در نظر گرفته می‌شود تنها در صورتی سازگار است كه متناقض باشد! امروزه موضوع كار بسیاری منطقدانان بررسی سازگاری قضيه‌هاي معروف ریاضیاتی با اصول زرملو ــ فرانكل و بررسي اثباتپذيری يا عدم اثباتپذيری آنها در اين دستگاه از اصول است. 
\par 
برای حساب،‌ و در ادامه‌ی اصول زرملو ــ‌ فرانکل،
اصول پئانو در نظر گرفته مي‌شود كه در زير درباره‌ي آن توضیحی داده‌ايم. خواننده‌ی علاقه‌مند  را به مطالعه‌ی مقاله‌ی «تجاهل بورباكی» ترجمه نويسنده‌ی دوم ترغيب می‌كنيم. 
\subsection{تئوری حساب}
برای ساختارِ
$\langle \mathbb{N},s,+,\times,0,1,\leq\rangle$
که در آن
$s(x)=x+1$
مجموعه‌ی اصول زیر (موسوم به اصول پئانو) را در نظر می‌گیریم.
\begin{itemize}
\item 
$\forall x\quad x+0=x$
\item 
$\forall xy (x+s(y)=s(x+y)$.
\item 
$\forall x \quad (x\times 1=x)$.
\item 
$\forall xy \quad (x\times s(y)=s\times y+x)$.
\item 
$\forall x \quad (x<s(x) \wedge \neg \exists y \quad x<y<s(x)$.
\item 
برای هر فرمولِ
$\phi(x,\bar{y})$
اصلِ
$I_\phi(x,\bar{y})$
که به صورت زیر نوشته می‌شود:
\[
\forall \bar{y}\quad \phi(0,\bar{y})\wedge \forall x (\phi(x,\bar{y})\to \phi(s(x),\bar{y})\to \forall x\phi(x,\bar{y}).
\]
\end{itemize}
اصول پئانو را می‌توان بخش مقدماتی حساب به شمار آورد. همانگونه که در بخش قبل گفتیم، این اصول قابلیت در خود گنجاندن همه‌ی حساب را ندارند. علاوه بر
اثباتی که گودل بر گفته ارائه کرده است، جملاتی پیدا شده است که در
اعداد طبیعی صادقند ولی از این اصول نتیجه نمی‌شوند (قضیه‌ی پاریس هرینگتون را ببینید). نیز قضیه‌ی آخر فرما که  معادله‌ی
$x^n+y^n=z^n$
برای
$n>2$
ریشه‌‌ی غیربدیهی ندارد، در اعداد طبیعی درست است، ولی نتیجه شدن یا نشدن آن از اصول پئانو، هنوز سوالی باز است. 
در واقع اثبات قضیه‌ی فرما، از منطق مرتبه‌ی اول فاصله‌ی زیادی دارد.
\subsection{گرافها}
تئوری گرافها در 
زبانِ 
$L_{\text{گراف}}=\{R\}$
شامل اصول زیر است.
\begin{itemize}
\item 
$\forall x\quad \neg R(x,x)$.
\item 
$\forall xy\quad (R(x,y)\to R(y,x))$.
\end{itemize}
گرافی که علاوه بر اصول فوق، از اصل زیر نیز پیروی کند، «تصادفی» خوانده می‌شود. 
\[
\forall x_1\ldots x_n\quad \forall y_1\ldots y_n 
(\bigwedge_{i,j\in \{1,\ldots,n\}} x_i\not=y_i\to \exists x \quad (\bigwedge_{i=1,\ldots,n} R(x,x_i)\wedge \bigwedge_{j=1,\ldots,n} \neg R(x,y_i))\]
\subsection{انحراف از بحث، گرافهای تصادفی و قاعده‌ی صفرویک}
بعداً تکمیل خواهد شد.
\subsection{فضاهای متریک}
برای هر
$r\in \mathbb{Q}\cap [0,1]$
یک محمولِ
$R_r(x,y)$
در زبان در نظر می‌گیریم، که قرار است نقشِ
$d(x,y)\leq r$
را ایفا کند.
اصول زیر را برای فضاهای متریک با متر کراندار در نظر می‌گیریم.
\begin{itemize}
\item 
$\forall xy \quad R_r(x,y)\leftrightarrow R_r(y,x)$.
\item 
$\forall xy (R_0(x,y)\leftrightarrow x=y)$.
\item 
$\forall xy (R_1(x,y)$.
\item 
$\forall xyz \quad (R_r(x,y)\wedge R_s(y,z)\to R_{r\dotplus s}(x,z))$.
\end{itemize}
که در آن،
$r\dotplus s=\min\{1,r+s\}$.
\begin{tam}
نشان دهید که هر مدل از تئوری بالا را می‌توان به عنوان یک فضای متریک با مقادیر متریِ واقع در بازه‌ی
$[0,1]$
در نظر گرفت (تعریف کنید
$d(x,y)=\inf\{s|R_s(x,y)\}$).
\end{tam}
\pagebreak
\section{جلسه‌ی سوم}
کلاسِ
$\mathcal{K}$
از
$L$ ــ
ساختارها را 
\textbf{مقدماتی}
\LTRfootnote{elementary class}
 می‌خوانیم هرگاه این کلاس، مجموعه‌ی همه‌ی
$L$ ــ
ساختارهایی باشد که  مدل یک تئوری مانند
$T$
هستند؛ به بیان دیگر هرگاه یک تئوریِ 
$T$
موجود باشد، به طوری که
$T=MOD(T)$
که 
در آن
$MOD(T)=\{\mathfrak{M}|\mathfrak{M}\models T\}$.
\begin{tam}
ثابت کنید که کلاس میدانهای متناهی، مقدماتی نیست.
\end{tam}
در زیر چند شرط لازم را برای مقدماتی بودن یک کلاس
$\mathcal{K}$
فهرست کرده‌ایم.
\begin{enumerate}
\item 
اگر
$\mathcal{K}$
دارای 
مدلهای متناهیِ‌ باندازه‌ی‌کافی‌بزرگ باشد، آنگاه
$\mathcal{K}$
حاوی مدلهایی نامتناهی باشد.
\item 
اگر
$\mathcal{K}$
حاوی 
مدلی نامتناهی باشد، آنگاه
$\mathcal{K}$
حاوی مدلهای نامتناهی از هر اندازه‌ی دلخواه باشد.
\item
$\mathcal{K}$
نسبت به هم‌ارزی مقدماتی بسته باشد.
\item 
$\mathcal{K}$
تحت فراضربها بسته باشد؛ یعنی هرگاه
$\langle \mathfrak{M}_i\rangle_{i\in I}$
دنباله‌ای از عناصرِ
$\mathcal{K}$
باشد و 
$\mathbf{F}$
فرافیلتری
روی
$I$،
آنگاه
$\prod_\mathbf{F}\mathfrak{M}_i\in \mathcal{K}$.
\end{enumerate}
\begin{mesal}
مفهوم کامل بودن یک میدان مرتب (یعنی اینکه هر زیرمجموعه‌ از آن دارای سوپرمم و اینفیمم باشد)‌ مفهومی مقدماتی نیست. در آنالیز مقدماتی ثابت می‌شود
که هر میدان مرتب کامل با
$\langle \mathbb{R},+,.,\leq\rangle$
ایزومرف است. پس موردِ ۲ در بالا 
در این تئوری مصداق ندارد.
\end{mesal}
تئوریِ
$T$
 را 
\textbf{ ارضاپذیر }
\LTRfootnote{satisfiable}
 می‌خوانیم هرگاه مدلی برای آن موجود باشد.
\begin{thm}[فشردگی]
تئوریِ
$T$
ارضاپذیر است اگروتنهااگر
هز زیرمجموعه‌ی متناهی از آن ارضاپذیر باشد. 
\end{thm}
اثباتِ قضیه‌ی فشردگی، جزو برنامه‌ی این درس نیست و فرض ما بر این است که دانشجو پیشتر اثباتی از آن را در درس نظریه‌ی مدلِ ۱ دیده است. با این حال یادآوری می‌کنیم
که قضیه‌ی فشردگی را می‌توان با روشهای مختلفی ثابت کرد. نخستین روش استفاده از قضیه‌ی درستی و تمامیت گودل است؛ یعنی قضیه‌ای که بنا بر آن (برای هر مجموعه‌ی
$\Sigma$
از جمله‌ها)
\[
\Sigma \models T \Leftrightarrow \Sigma\vdash T.
\]
به ویژه، بنا به قضیه‌ی یاددشده
\[
\Sigma\models \perp \Leftrightarrow \Sigma\vdash \perp;
\]
به بیان دیگر،
$\Sigma$
سازگار 
است اگروتنهااگر دارای مدل باشد.  از طرفی 
$\Sigma$
سازگار است 
اگروتنهااگر هر بخش متناهی از آن سازگار باشد، اگروتنهااگر هر بخش متناهی از آن دارای مدل باشد. پس 
$\Sigma$
دارای مدل است اگروتنهااگر هر بخش متناهی از آن دارای مدل باشد.
\par 
روش دیگر برای اثبات فشردگی، بهره‌گیری از ساختمانهای هنکین
\LTRfootnote{Henkin}
 است.  در این روش، تکنیکهای اثبات قضیه‌ی درستی و تمامیت، مستقیماً برای بناکردن یک مدل برای
$\Sigma$
استفاده می‌شوند. 
\par 
روش سوم، که در تمرینهای درس بدان پرداخته خواهد شد، استفاده از فراضربهاست. در این روش مدل مورد نظر از فراضربی از
مدلهای موجود برای هر بخشِ متناهی 
از
$\Sigma$
حاصل می‌شود.
\par 
برای یک زبان داده‌شده‌ی 
$L$
تعریف 
کنید
$\|L\|:=\max\{\aleph_0,|L|\}$؛
به بیان دیگر،
\[
\|L\|=\begin{cases}
|L| & |L|\geq \aleph_0,
\\
\aleph_0 & |L|
\end{cases}
\]
\begin{thm}[لُوِنهایم اسکولمِ کاهشی]
اگر تئوریِ
$T$
ارضاپذیر باشد،  مدلی با اندازه‌ی کمتریامساویِ
$\|L\|$
دارد. 
\end{thm}
\begin{thm}[لُونهایم اسکولمِ افزایشی]
اگر تئوریِ
$T$
ارضاپذیر 
و دارای مدلی نامتناهی باشد،
آنگاه برای هر کاردینالِ نامتناهیِ 
$\kappa\geq \|L\|$
مدلی از اندازه‌ی
$\kappa$
دارد. 
\end{thm}
\begin{tam}
فرض کنید تئوریِ
$T$
 دارای 
 مدلی نامتناهی باشد. با استفاده از لمهای لونهایم‌اسکولم و فشردگی، نشان دهید که 
 آنگاه 
 $T$
 مدلی با اندازه‌ی
دقیقاً
برابر با
$\kappa$
 دارد. 
\end{tam}
گفتیم که اگر کلاسی مقدماتی باشد، تحتِ فراضربها بسته است. در زیر شرطی لازم و کافی برای مقدماتی بودن یک کلاس آورده‌ایم.
\begin{thm}
\hfill
\begin{enumerate}
\item 
کلاسِ
$\mathcal{K}$
از
$L$  ــ‌
ساختارها مقدماتی است اگروتنهااگر تحت فراضربها و تحت هم‌ارزی مقدماتی بسته باشد.
\item 
کلاسِ
$\mathcal{K}$
از
$L$  ــ‌
ساختارها مقدماتی است اگروتنهااگر تحت فراضربها و تحت ایزومرفیسم بسته باشد.
\end{enumerate}
\end{thm}
مورد دوم، قضیه‌ای از شلاه و کیسلر 
\LTRfootnote{Shelah, Kiesler}
است که اثبات آن را به عنوان پروژه بر عهده‌ی دانشجویان وامی‌نهیم. 
\begin{proof}[اثبات قسمت اول]
گیریم
$\mathcal{K}$
تحت 
فراضربها و هم‌ارزی مقدماتی بسته باشد؛ هدفمان یافتن تئوریِ 
$T$
است به طوری که
\[
\mathcal{K}=MOD(T).
\]
ادعا می‌کنیم که تئوریِ
$T$
در پایین،‌ همانگونه است که می‌خواهیم:
\[
T=\bigcap_{i\in I} \Th(M_i)=\{\phi|\forall \mathfrak{M}\in \mathcal{K} \quad \mathfrak{M}\models \phi\}.
\]
\begin{tav}
بنابراین، ثابت خواهیم کرد که هرگاه
$\langle\mathfrak{M}_i\rangle_{i\in I}$
 خانواده‌ای از 
$L$  ــ
ساختارها باشد، 
آنگاه تئوری اشتراک آنها هم‌ارز مقدماتی با 
تئوری فراضربی از آنهاست.
\end{tav}
نخست توجه کنید که 
$T$
بوضوح ارضاپذیر است
و 
$\mathcal{K}\subseteq MOD(T)$.
\par 
فرض کنید
$\mathfrak{N}$
مدلی 
از
$T$
باشد.  نشان خواهیم داد 
که 
خانواده‌ی
$\langle \mathfrak{M}_i\rangle_{i\in I}$
و فرافیلترِ
$F$
روی
$I$
 چنان 
 موجودند که 
 $\mathfrak{N}\equiv \prod_F \mathfrak{M}_i$.
 از آنجا که 
 $\mathcal{K}$
 تحت 
 هم‌ارزی مقدماتی و فراضربها بسته است، از این نتیجه خواهد شد که
 $\mathfrak{N}\in \mathcal{K}$.
 \par
برای
هر
گزاره‌ی
 $\theta\in\Th(\mathfrak{N})$
 مدلی 
 چون
 $\mathfrak{M}_\theta$
 چنان
 موجود است که 
 $\mathfrak{M}_\theta \models \theta$
 (در غیر این صورت
 $\neg\theta\in\bigcap_{i\in I} \Th(\mathfrak{M}_i)= T$).
 حال می‌گیریم
 \[
 I=\{\Delta|\Delta\subseteq_{\text{متناهی}} \Th(\mathfrak{N})\}
 \]
 و برای هر 
 $\Delta\in I$
 قرار می‌دهیم
 \[
 \Sigma_\Delta=\{\Delta'\in I|\Delta\subseteq \Delta'\}.
 \]
 مجموعه‌ی
 $X=\{\Sigma_\Delta|\Delta\in I\}$
 ویژگی اشتراک متناهی ناتهی دارد و از این رو در فرافیلتری چون
 $F$
 روی
 $I$
 واقع می‌شود. 
\begin{tam}
برای به پایان رساندن اثبات، نشان دهید که
 $\prod_ F \mathfrak{M}_i\equiv \mathfrak{N}$.
 \end{tam}
 \end{proof}
در نظریه‌ی مدل، مدلها را با استفاده از فرمولهای صادق در آنها مطالعه می‌کنیم. در برخی تئوریها همه‌ی فرمولها معادل با نوع خاصی از فرمولهایند.
\begin{defn}
تئوریِ
$T$
را
\textbf{دارای حذف سور }
\LTRfootnote{quantifier elimination}
می‌خوانیم 
هرگاه هر فرمولی به پیمانه‌ی آن معادلی بدون سور داشته باشد؛ یعنی برای هر فرمولِ
$\phi(x_1,\ldots,x_n)$
فرمولی 
بدون سور چون
$\psi(x_1,\ldots,x_n)$
موجود باشد، به طوری که
\[
T\models \forall x_1,\ldots ,x_n \quad \phi(x_1,\ldots,x_n)\leftrightarrow \psi(x_1,\ldots,x_n).
\]
\end{defn} 
برای اینکه تعریف بالا جمله‌ها را نیز در برگیرد، نیاز است که  
زبانِ
$L$
حاوی 
حداقل یک ثابت باشد.
\par 
حذف سور را از نظرگاههای زیر،
 یک «ویژگی جبری» از تئوریها به حساب می‌آورند.  نخست این که هر فرمول بدون سور، در یک ساختار جبری ترکیبی بولی 
 از چندگوناها 
 \LTRfootnote{variety}
 را به دست می‌دهد. برای مثال، در تئوریِ
  $ACF$
  فرمولهای 
  بدون سور، چندگوناهای جبری را، یعنی ترکیبات بولی مجموعه‌های به شکل
  $\{\bar{x}|f_1(\bar{x})=f_2(\bar{x})=\ldots=f_n(\bar{x})=0\}$
   را به دست می‌دهند که در این نمایش
  $f_i$
  ها چندجمله‌ایند. 
  نیز در
  $RCF$
  فرمولهای بدون سور، 
  مجموعه‌های شبه‌جبری 
  \LTRfootnote{semialgebraic}
  را به دست می‌دهند؛ یعنی مجموعه‌هایی که از ترکیبات بولی مجموعه‌هایی به شکل زیر حاصل می‌شود:
  $\{\bar{x}|f_1(\bar{x})>0, \ldots, f_n(\bar{x})>0\}$.
  پس 
  حذف سور داشتن در این تئوریها یعنی برابر بودنِ مجموعه‌های تعریف‌پذیر با 
  ترکیبات بولی چندگوناها. 
  \par 
 دوم این که اگر
  $\mathfrak{M},\mathfrak{N}$
  دو
  مدل از یک تئوریِ دارای حذف سورِ
  $T$
  باشند به طوری که
  $\mathfrak{M}\subseteq \mathfrak{N}$،
  آنگاه
  $\mathfrak{M}\prec \mathfrak{N}$
  (علت: می‌دانیم که اگر
  $\mathfrak{M}\subseteq \mathfrak{N}$
  آنگاه 
  $\mathfrak{M}$
  و
  $\mathfrak{N}$
  درباره‌ی 
  فرمولهای بدون سور با پارامتر در
  $M$
  همنظرند. حال
  بنا به حذف سور، همه‌ی فرمولها را می‌توان بدون سور در نظر گرفت).
  \begin{mesal}
  تئوریهای
  $ACF$
  و
  $RCF$
  سورها را حذف می‌کنند.
  \end{mesal}
در ساختارِ
$\langle \mathbb{R},+,\cdot,0,1,\leq\rangle$
 فرمولِ
 $\exists x\quad ax^2+bx+c=0$
 دارای معادل بدون سورِ
 $b^2-4ac\geq 0$
 است. این را می‌توان با روشهای مقدماتی جبری ثابت کرد، اما یافتن معادل بدون سور برای همه‌ی فرمولها بدین سادگی نیست. عموماً برای بررسی حذف سور از محکهایی استفاده می‌شود که در تمرینهای درس، به یکی از آنها یعنی وجود سامانه‌های رفت و برگشتی خواهیم پرداخت. 
 \par 
 تئوریهای دارای حذف سور را گاهی 
\textbf{زیرساختارْکامل}
 می‌خوانند:
 \begin{tam}
موارد زیر با هم معادلند:
 \begin{enumerate}
 \item 
  تئوریِ
  $T$
  سورها را حذف می‌کند.
 \item 
 $\diag(\mathfrak{A})\cup T$
 برای 
 برای هر مدلِ
 $\mathfrak{M}\models T$
 و هر زیرساختِ
  $\mathfrak{A}\subseteq \mathfrak{M}$
  یک تئوریِ
  کامل است. 	
 \end{enumerate}
 \end{tam}
تئوریِ
$T$
را 
\textbf{مدل‌کامل}
 می‌خوانند هرگاه هر فرمول به پیمانه‌ی آن دارای معادلی وجودی باشد؛ یعنی برای هر فرمولِ
$\phi(x_1,\ldots,x_n)$
فرمولی 
بدون سور چون
$\psi(x_1,\ldots,x_n,y_1,\ldots,y_m)$
موجود 
باشد، به طوری که
\[
T\models \forall x_1,\ldots,x_n \quad \phi(x_1,\ldots,x_n)\leftrightarrow \exists y_1,\ldots y_m \quad \psi(x_1,\ldots,x_n,y_1,\ldots,y_m).
\]
(واضح است که)
   حذف
   سور، مدل‌کامل بودن را نتیجه می‌دهد؛ ولی عکس این برقرار نیست.
   \begin{tam}
   نشان دهید که 
   $RCF$
   در زبان
   $L=\{+,-,\cdot,0,1\}$
    مدل‌کامل است ولی سورها را حذف نمی‌کند.
   \end{tam}
وجه تسمیهِ «مدل‌کامل» در تمرین زیر مشخص می‌شود.
  \begin{tam}
  نشان دهید که موارد زیر با هم معادلند:
  \begin{enumerate}
  \item 
  تئوریِ
  $T$
  مدل‌کامل
  است.
  \item 
  برای هر مدلِ
  $\mathfrak{M}\models T$
       تئوریِ
       $\diag(\mathfrak{M})\cup T$
        کامل 
        است.
  \item 
  برای هر دو مدلِ
  $\mathfrak{M},\mathfrak{N}\models T$
  داریم
  \[
  \mathfrak{M}\subseteq \mathfrak{N}\Leftrightarrow \mathfrak{M}\prec \mathfrak{N}.
  \]
  \end{enumerate}
  \end{tam}
\pagebreak
\section{فضای تایپها}
تایپها مصداق نظریه‌ی مدلی ایده‌ی آشنای شناخت «ذات» از روی «صفات» هستند. 
در نظریه‌ی مدل نیز
گاهی میان ذات یک عنصر و مجموعه‌ی صفات آن تمایز قائل نمی‌شویم.  
مجموعه‌ی ویژگی‌های یک عنصر داده شده را در نظریه‌ی مدل، تایپ خواهیم نامید. 
\par 
گیریم
$\mathfrak{M},\mathfrak{N}\models T$،
$\bar{a}\in M$
و
$\bar{b}\in N$.
هدف پاسخ به این سوال است که در چه صورتی  
تئوریِ 
$T$
نمی‌تواند میان دنباله‌های (نه لزوماً متناهی) 
$\bar{a},\bar{b}$
تمایز بگذارد. خواهیم دید که این خواسته زمانی
برآورده می‌شود که
این دو دنباله نسبت به تئوری یادشده همتایپ باشند. 
\par 
بگذارید بحث را با مثالی پی بگیریم.
اگر
$x,y$
دو عنصر متعالی روی میدانِ 
$\mathbb{Q}$
باشند، آنگاه
$\mathbb{Q}(x)\cong \mathbb{Q}(y)$; 
یعنی این دو عنصر از لحاظ جبری روی 
$\mathbb{Q}$
ارزش یکسانی دارند. به زبان نظریه‌ی مدلی، این دو عنصر روی
$\mathbb{Q}$
همتایپند.
در ادامه مطالب بالا را به زبان دقیق نظریه‌ی مدلی درآورده‌‌ایم.
\par 
یادآوری می‌کنیم که منظور از
عبارتِ
\[
\text{
فرمولِ
$\phi(x_1,\ldots,x_n)$
با تئوری
$T$
سازگار است
\quad (*)
}
\]
 این است که
$T\cup \{\exists x_1,\ldots x_n \quad \phi(x_1,\ldots x_n)\}$
مجموعه‌ای سازگار از جمله‌هاست؛ به بیان دیگر، مدل
$\mathfrak{M}$
از
$T$
و عناصرِ
$\alpha_1,\ldots, \alpha_n\in M$
چنان موجودند که
$\mathfrak{M}\models \phi(\alpha_1\ldots,\alpha_n)$.
فرض کنید
$c_1,\ldots, c_n$
ثوابتی جدید باشند و 
$L'=L\cup \{c_1,\ldots,c_n\}$.
آنگاه 
$(*)$
معادل این است که
$L'$
ــ 
تئوریِ
$T\cup \{\phi(c_1,\ldots,c_n)\}$
سازگار باشد؛  یعنی
$L'$
ــ‌
ساختارِ
$\mathfrak{M}'=\langle \mathfrak{M},c_1,\ldots,c_n\rangle$
موجود باشد به طوری که
$\mathfrak{M}'\models T$
و
$\mathfrak{M}'\models \phi(c_1,\ldots,c_n)$.
توجه کنید که هرگاه که
$\mathfrak{M}'=\langle \mathfrak{M},\ldots\rangle$
بسطی
 از
$L$ 
ــ
ساختارِ
$\mathfrak{M}$ 
 به زبانِ
$L'=L\cup \{\ldots\}$
باشد، آنگاه برای هر
$L$ ــ
جمله‌ی
$\phi$
داریم
\[
\mathfrak{M}'\models \phi \Leftrightarrow \mathfrak{M}\models \phi.
\]
در سرتاسرِ ادامه‌ی این جلسه، فرض کرده‌ایم که
$T$
یک تئوری کامل باشد در زبانِ
$L$
که هیچ مدل از آن متناهی نیست.
برای تعریف بعدی، دو مدلِ
$\mathfrak{M},\mathfrak{N}\models T$
و چندتایی‌های
\mbox{$\bar{a}=(a_1,\ldots,a_n)\in M$}
و
  $\bar{b}=(b_1,\ldots,b_n)\in N$
را در نظر بگیرید.
\begin{defn}
دنباله‌‌های
$\bar{a}$
و
$\bar{b}$
را 
\textbf{گالواهم‌ارز }
می‌خوانیم هرگاه
مدلِ
$\mathbb{M}$
و 
نشاندنهای مقدماتیِ
\mbox{$f:\mathfrak{M}\to \mathbb{M}$}
و
$g:\mathfrak{N}\to \mathbb{M}$
چنان موجود باشند که 
$f(\bar{a})=g(\bar{b})$.
در این صورت می‌نویسیم
$\langle \mathfrak{M},\bar{a}\rangle\sim_{Gal} \langle\mathfrak{N},\bar{b}\rangle$
یا
$\bar{a}\sim_{Gal} \bar{b}$.
\end{defn}
توجه کنید که در نماد
$\bar{a}\sim_{Gal} \bar{b}$
مدلها به چشم نمی‌آیند. بعدها خواهیم دید که در این نمادگذاری عمدی در کار است؛‌
مدلها، به پیمانه‌ی نشستنهای مقدماتی دراین تعریف بی‌نقشند.
\begin{mesal}
فرض کنید
$\mathfrak{M}\models T$
و
$f\in \Aut(\mathfrak{M})$.
آنگاه برای هر
$a_1,\ldots a_n\in M$
داریم
\[
a_1\ldots a_n\sim_{Gal} f(a_1)\ldots f(a_n).
\]
\end{mesal}
\begin{tav}
اگر
$\bar{a}\sim_{Gal} \bar{b}$
آنگاه برای هر
$L$ 
ــ‌ 
فرمولِ
$\phi(\bar{x})$
داریم
$\mathfrak{M}\models \phi(\bar{a})$
اگروتنهااگر
\mbox{$\mathfrak{N}\models \phi(\bar{b})$}؛
به عبارت دیگر
\[
\bar{a}\sim_{Gal}\bar{b}\Rightarrow \langle\mathfrak{M},\bar{a}\rangle
\equiv \langle \mathfrak{N},\bar{b}\rangle.
\]
\end{tav}
رابطه‌ی
$\sim_{Gal}$
یک رابطه‌ی هم‌ارزی است. برای اثبات این گفته، به لم زیر نیاز داریم که آن را 
در جلسات تمرین اثبات خواهیم کرد.
\begin{tam}
فرض کنید
تئوری
$T$
کامل باشد. نشان دهید در آن صورت
\begin{enumerate}
\item 
$T$
دارای ویژگیِ ادغام (یا ملغمه‌سازی)
\LTRfootnote{amalgamation property (AP)}
است؛ بدین معنی که هرگاه
$\mathfrak{A},\mathfrak{B},\mathfrak{C}\models T$
و
$f_1:\mathfrak{A}\to \mathfrak{B}$
و
$f_2:\mathfrak{A}\to \mathfrak{C}$
نشاندنهای مقدماتی باشند، آنگاه مدلِ
$\mathfrak{D}\models T$
و نشاندنهای مقدماتیِ
$g_1:\mathfrak{B}\to \mathfrak{D}$
و
$g_2:\mathfrak{C}\to \mathfrak{D}$
چنان موجودند که 
$g_1\circ f_1=g_2\circ f_2$.
\item 
$T$
دارای ويژگیِ 
امکان‌نشاندن‌همزمان 
\LTRfootnote{joint embedding property}
است؛
یعنی برای هر دو مدلِ
$\mathfrak{A},\mathfrak{B}\models T$
مدلی چون
$\mathfrak{C}\models T$
و نشاندنهایی مقدماتی چون
$f:\mathfrak{A}\to \mathfrak{C}$
و
$g:\mathfrak{B}\to \mathfrak{C}$
موجودند.
\end{enumerate}
\end{tam}
\begin{prop}
رابطه‌ی
$\sim_{Gal}$
هم‌ارزی است.
\end{prop}
\begin{proof}
بررسی
انعکاسی و تقارنی بودن رابطه‌ی یادشده چندان دشوار نیست؛ در اینجا تنها به اثبات تعدی آن پرداخته‌ایم. گیریم
$\bar{a}\sim_{Gal}\bar{b}$
و
$\bar{b}\sim_{Gal} \bar{c}$.
پس نگاشتهای مقدماتیِ
\mbox{$f_\mathfrak{M}:\langle \mathfrak{M},\bar{a}\rangle \to \langle \mathfrak{Q}_1,\bar{q}_1\rangle$ }
و
$f:\mathfrak{N}:\langle\mathfrak{N},\bar{b}\rangle\to\langle \mathfrak{Q}_1,\bar{q}_1\rangle$
چنان موجودند که 
\mbox{$f_\mathfrak{M}(\bar{a})=f_\mathfrak{N}(\bar{b})=\bar{q}_1$}.
به همین ترتیب نگاشتهای مقدماتی
$g_\mathfrak{N}:\langle \mathfrak{N},\bar{b}\rangle\to \langle \mathfrak{Q}_2,\bar{q}_2\rangle$
و
$g_\mathfrak{P}:\langle\mathfrak{P},\bar{p}\rangle \to \langle \mathfrak{Q}_2,\bar{q}_2\rangle$
چنان موجودند
که 
$g_\mathfrak{N}(\bar{b})=g_\mathfrak{P}(\bar{p})=\bar{q}_2$.
هدفمان پیدا کردن مدل
$\langle\mathfrak{Q}_3,\bar{q}_3\rangle$
به همراه نگاشتهای مقدماتیِ
$h_\mathfrak{M}:\langle \mathfrak{M},\bar{a}\rangle\to \langle \mathfrak{Q}_3,\bar{q}_3\rangle$
و 
$h_\mathfrak{P}:\langle \mathfrak{P},\bar{p}\rangle\to\langle \mathfrak{Q}_3,\bar{q}_3\rangle$
است به طوری که
$h_\mathfrak{M}(\bar{a})=h_\mathfrak{p}(\bar{p})=\bar{q}_3$.
\par 
بنا به ویژگی ادغام، می‌توان
$L\cup \{c_1,\ldots,c_n\rangle$
ــ‌
ساختارِ 
$\langle \mathfrak{Q}_3,\bar{q}_3\rangle$
و نشاندنهای 
\mbox{$h_{\mathfrak{Q}_1}:\langle \mathfrak{Q}_1,\bar{q}_1\rangle\to
\langle \mathfrak{Q}_3,\bar{q}_3\rangle$}
و
$h_{\mathfrak{Q}_2}:\langle \mathfrak{Q}_2,\bar{q}_2\rangle\to
\langle \mathfrak{Q}_3,\bar{q}_3\rangle$
را چنان یافت که
\mbox{$h_{\mathfrak{Q}_1}\circ f_\mathfrak{N}=h_{\mathfrak{Q}_2}\circ g_\mathfrak{N}$}.
حال
$h_{\mathfrak{Q}_1}\circ f_\mathfrak{M}$
و
$h_{\mathfrak{Q}_2}\circ g_\mathfrak{P}$
نشاندنهای مورد نیاز هستند.
\begin{displaymath}
    \xymatrix{
        \langle \mathfrak{M},\bar{a}\rangle \ar[dr] & &  & \\
                & \langle \mathfrak{Q}_1,\bar{q}_1\rangle \ar[dr] &  & \\
        \langle \mathfrak{N},\bar{b}\rangle  \ar[ur]\ar[dr]&  & \langle \mathfrak{Q}_3,\bar{q}_3\rangle &  \\
          & \langle \mathfrak{Q}_2,\bar{q}_2\rangle \ar[ur]& &  \\
        \langle \mathfrak{P},\bar{p}\rangle \ar[ur] &  &  &  
       }
\end{displaymath}
\end{proof}
گفتیم که
از
$\bar{a}\sim_{Gal}\bar{b}$
نتیجه می‌شود که
$\langle \mathfrak{M},\bar{a}\rangle\equiv\langle \mathfrak{N},\bar{b}\rangle$.
عکس این گفته نیز برقرار است:
\begin{prop}
اگر
$\langle \mathfrak{M},\bar{a}\rangle\equiv\langle \mathfrak{N},\bar{b}\rangle$
آنگاه
$\bar{a}\sim_{Gal}\bar{b}$.
\end{prop}
\begin{proof}
فرض کنید
$\langle \mathfrak{Q},\bar{q}\rangle$
مدلی باشد از تئوریِ
$\Th(\mathfrak{M},\bar{a})=\Th(\mathfrak{N},\bar{b})$.
نشاندنهای مورد نیاز به آسانی پیدا می‌شوند.
\end{proof}
\begin{defn}
گیریم 
$\mathfrak{M}\models T$
و
$a_1,\ldots,a_n\in M$.
تعریف می‌کنیم
\[
\tp^\mathfrak{M}(a_1,\ldots,a_n)=\{\phi(x_1,\ldots,x_n)|\mathfrak{M}\models
\phi(a_1,\ldots,a_n)\}.
\]
\end{defn}
بنابر آنچه گفته شد
\[
\bar{a}\sim_{Gal}\bar{b}\Leftrightarrow \tp^\mathfrak{M}(\bar{a})
=\tp^\mathfrak{N}(\bar{b});
\]
به‌ويژه کلاسهای هم‌ارزی رابطه‌ی 
$\sim_{Gal}$
تشکیل
\textbf{ مجموعه}
 می‌دهند: قرار دهید
 \[
 S_n(T)=\{\tp^\mathfrak{M}(a_1,\ldots,a_n)|
 \mathfrak{M}\models T, a_1,\ldots, a_n\in \mathfrak{M}\},
 \]
 واضح است که
 $|S_n(T)|\leq 2^{\|L\|}$.
\begin{defn}[تایپ جزئی]
مجموعه‌ی
$\Sigma(x_1,\ldots,x_n)$
متشکل از
$L$
ــ 
فرمولهایی با متغیرهای در میانِ
$x_1\ldots,x_n$
را 
یک
\textbf{تایپِ جزئی}
\LTRfootnote{partial type}
می‌خوانیم هرگاه
$T\cup \Sigma(x_1,\ldots,x_n)$
سازگار باشد؛ یعنی
$\mathfrak{M}\models T$
و عناصرِ
$\alpha_1,\ldots,\alpha_n\in M$
چنان موجود باشند که برای هر
$\phi(x_1,\ldots,x_n)\in \Sigma$
داشته باشیم
$\mathfrak{M}\models \phi(\bar{\alpha})$.
\end{defn}
یک تایپ جزئی را می‌توان به عنوان دستگاهی از معادلات دانست که محدودیتهای آن شروط تئوری
$T$
است. سازگاری، ضامن پاسخدار بودن این مجموعه از معادلات است. بنابراین،
$\Sigma(x_1,\ldots,x_n)$
یک تایپ جزئی است اگروتنهااگر مدلِ
$\mathfrak{M}\models T$
و عناصرِ
$\alpha_1,\ldots,\alpha_n\in M$
موجود باشند به طوری که
\[
\Sigma(x_1,\ldots,x_n)\subseteq \tp^\mathfrak{M}(\alpha_1,\ldots,\alpha_n).
\]
در صورتی که در بالا، تساوی رخ دهد، 
$\Sigma$
را یک
\textbf{تایپ کامل،}
\LTRfootnote{complete type}
 یا طور خلاصه یک 
\textbf{ تایپ}
  می‌خوانیم.
\begin{nokte}
متناهی بودن تعداد متغیرها در تعریفهای بالا ضروی نیست. فرض کنید
$\langle x_i:i\in I\rangle$
دنباله‌ای از متغیرها باشد. مجموعه‌ی
$\Sigma(x_i|i\in I)$
متشکل از فرمولهایی چون
$\phi(x_{i1},\ldots,x_{in})$
را تایپ جزئی می‌خوانیم هرگاه با
$T$
سازگار باشد.
بدینسان
نیز می‌توان
مجموعه‌ی
$S_I(T)$
را
متشکل از تایپهای کامل با این متغیرها  تشکیل داد. 
\end{nokte}
\begin{defn}[توپولوژی استون]
در یک زبانِ مشخصِ
$L$
قرار دهید
\[
\Th_L=\{T|\text{ $T$ یک تئوری کامل است }\}
\]
داریم
$\Th_L\subseteq 2^{2^{\text{جمله‌ها}}}$.
برای هر 
$L$
ــ
جمله‌ی
$\phi$
قرار دهید
$[\phi]=\T\in \Th_L|\phi\in T\}$.
مجموعه‌های
$[\phi]$
تشکیل پایه‌ای برای یک توپولوژی روی
$\Th_L$
می‌دهند که این توپولوژی، بنا به قضیه‌ی فشردگی، فشرده است. به طور مشابه، می‌توان روی
 $S_n(T)$
 توپولوژی ایجاد شونده توسط عناصر پایه‌ایِ زیر را در نظر گرفت:
\[ 
 [\phi(x_1,\ldots,x_n]=\{p|
p=\tp^\mathfrak{M}(a_1,\ldots,a_n),
 \phi\in 
p,\mathfrak{M}\models T,a_1,\ldots,a_n\in M\}.
\]
توپولوژی یادشده را \textbf{توپولوژی استون} می‌خوانیم. این توپولوژی فشرده، تماماً ناهمبند و هاسدورف است.
\end{defn}
\pagebreak
\section{جلسه‌ی پنجم، مثالهایی از تایپها}
پیش از آن که به هدف این جلسه، یعنی 
بررسی چند مثال از تایپها بپردازیم، نکته‌ی زیر را درباره‌ی توپولوژی استون متذکر می‌شویم.
\begin{nokte}
فرض کنید
$\langle B,\wedge,\vee, 0,1\rangle$
یک جبر بولی باشد، که روی آن ترتیب 
$a\leq b\Leftrightarrow a\wedge b=a$
تعریف شده است. زیرمجموعه‌ی
$A\subseteq B$
را یک فیلتر، یا یک پالایه می‌خوانیم هرگاه
اصول زیر درباره‌ی آن صادق باشند.
\begin{itemize}
\item 
$a\in A\to \forall b\geq a \quad b\in A$
\item $a,b\in A\to a\wedge b\in A$
\item $0\not\in A$.
\end{itemize}
\end{nokte}
مفهوم
فیلتر،‌ دوگان مفهوم ایده‌آل است (یک جبر بولی را می‌توان حلقه‌ای با مشخصه‌ی صفر در نظر گرفت. ایده‌آل در این بافتار معنا می‌یابد). فیلترِ
$A$
را یک فرافیلتر می‌خوانیم هرگاه برای هر
$a,b\in B$
از
$a\wedge b\in A$
نتیجه شود که
$a\in A$
یا
$b\in A$.
\par 
برای جبر بولیِ
$B$
قرار می‌دهیم
\[
\max({B})=\{A\subseteq B| \text{$A$ یک فرافیلتر روی $B$ است}\}
\]
روی
$\max({B})$
مجموعه‌های زیر تشکیل پایه‌ای برای یک توپولوژی می‌دهند که آن را توپولوژی استون می‌خوانند:
\[
[a]=\{A\in \max(B)|a\in A\}.
\]
این توپولوژی، فشرده و تماماً ناهمبند است. 
\par 
توپولوژی استون در فضایِ
$S_n(T)$
نیز از توپولوژی استون جبری به دست می‌آید  روی جبر
لیندنبام ــ تارسکی  که به صورت زیر تعریف می‌شود:
\[
B_n(T)=\{[\phi(\bar{x})]_\sim| \phi(\bar{x})\in Formul\}
\]
که در آن منظور از
$[\phi(\bar{x})]_\sim$
کلاس فرمولِ
$\phi$
تحت رابطه‌ی هم‌ارزی زیر است:
\[
\phi(\bar{x})\sim \psi(\bar{x})\Leftrightarrow T\models \forall \bar{x}
\quad \phi(\bar{x})\leftrightarrow \psi(\bar{x})
\]
و تعریف کرده‌ایم
\begin{align*}
& [\phi]_\sim \wedge [\psi]_\sim=[\phi\wedge\psi]_\sim\\
& [\phi]_\sim \vee [\psi]_\sim=[\phi\vee\psi]_\sim\\
& 0=[x\not=x]_\sim\\
& 1=[x=x]_\sim 
\end{align*}
\begin{tam}
نشان دهید که ابرفیلترها
در
$B_n(T)$
همان تایپهای کامل هستند.
\end{tam}
\begin{mesal}[تایپها در
$DLO$]
تئوریِ
$DLO$،
یا تئوری مجموعه‌های مرتب خطی بدون ابتدا و انتها، در زبانِ
$L=\{\leq\}$
به صورت زیر اصلبندی می‌شود.
\begin{itemize}
\item 
$\forall x \quad x\leq x$.
\item 
$\forall xy \quad x\leq y \wedge y\leq x\to x=y$
\item 
$\forall xyz \quad x\leq y\wedge y\leq z\to x\leq z$.
\item
$\forall xy \quad x\leq y \vee y\leq z$
\item 
$\forall xy\quad x\leq y\to \exists x  \quad x<z< y$.
\item 
$\forall x\quad \exists y_1y_2 \quad y_1<x<y_2$.
\end{itemize}
به عنوان مثال
$\langle \mathbb{Q},\leq\rangle$
و
$\langle \mathbb{R},\leq\rangle$
دو مدل از 
$DLO$
هستند. 
\par 
با استفاده از سامانه‌های رفت و برگشتی (به تمرینهای سری نخست و دوم مراجعه کنید)
می‌توان نشان داد که تئوریِ
$DLO$
سورها را حذف می‌کند و
$\aleph_0$
ــ 
جازم است؛ این دومی یعنی  هر دو مدل شمارا از
$DLO$
با هم ایزومرفند. در ادامه برآنیم تا تایپها را در
$DLO$
بشناسانیم.
\par 
نخست به بررسی
$S_1(DLO)$
می‌پردازیم. بنا به حذف سور، هر فرمول معادل است با
فصلی متناهی از عطفهای متناهی فرمولهای اتمی. فرمولهای اتمی و نقیض اتمیِ
با تک متغیرِ
$x$
(و بدون پارامتر)
تنها به یکی از صُوَرِ زیرند:
\begin{itemize}
\item $x\leq x$.
\item $x=x$.
\item $\neg (x\leq x)$
\item $\neg (x=x)$.
\end{itemize}
توجه کنید که اگر 
$p_1,p_2$
دو تایپِ متفاوت باشند، از آنجا که تایپ کامل، مجموعه‌ای ماکزیمال از فرمولهاست، باید
فرمولی چون
$\phi(x)$
موجود باشد که ایندو را از هم متمایز کند؛ یعنی
$\phi\in p_1$
و
$\neg \phi\in p_2$.
سه نوع فرمول بالا با هم سازگارند (و فرمول آخر نمی‌تواند در هیچ تایپی باشد چون ناسازگار است)؛ پس
$|S_1(DLO)|=|[x=x]|=|[x\leq x]|=1$.
\par 
حال به
$S_n(DLO)$
می‌پردازیم.
گیریم
$\mathfrak{M}\models DLO$
و
$a_1,\ldots,a_n\in M$.
قرار می‌دهیم
\begin{align*}
&
\diag(x_1,\ldots,x_n)_{a_1,\ldots,a_n}:=\qftp(a_1,\ldots,a_n)=
\{\phi(x_1,\ldots,x_n)|M\models \\
&
\phi(a_1,\ldots,a_n), \text{ $\phi$ اتمی یا نقیض اتمی است }
\}
\end{align*}
بنا به حذف سور،
$\tp(a_1,\ldots,a_n)$
را
$\diag(x_1,\ldots,x_n)_{a_1,\ldots,a_n}$
به طور کامل مشخص می‌کند؛ یعنی
\[
\{\tp(a_1,\ldots,a_n)\}=[\bigwedge_{\phi\in \diag(x_1,\ldots,x_n)_{a_1,\ldots,a_n} }\phi]
\]
بنابراین
برای هر
$n\in \mathbb{N}$
مجموعه‌ی
$S_n(DLO)$
متناهی است.
\par 
در جلسات آینده قضیه‌ی
ریل‌ نارْدِوْسکی
\LTRfootnote{Ryll-Nardewski}
را ثابت خواهیم کرد که بنا به آن، هر تئوری کامل،
$\aleph_0$
ــ
جازم است اگروتنهااگر تعداد
$n$
تایپها در آن متناهی باشد. 
\par 
حال به بررسی تایپهای دارای پارامتر می‌پردازیم. مدلِ
$\langle \mathbb{Q},\leq\rangle \models T$
را در نظر گرفته قرار دهید
$T_\mathbb{Q}=\Th(\langle\mathbb{Q},\leq,r\rangle_{r\in \mathbb{Q}})$.
طبیعتاً زبانِ این تئوری دارای ثوابتِ
$\{c_r\}_{r\in \mathbb{Q}}$
و از این رو هر مدل از تئوریِ یادشده، توسیعی مقدماتی از
$\langle \mathbb{Q},\leq\rangle$
است. 
به آسانی می‌توان تحقیق کرد که
$T_\mathbb{Q}$
سورها را حذف می‌کند (از آنجا که
$DLO$
چنین است). فرمولهای اتمی و نقیض اتمیِ تک‌متغیره در این حالت، به یکی از صُوَر زیرند:
\begin{itemize}
\item $x\leq x$
\item $x=x$.
\item $x<c_r$
\item $x\geq c_r$.
\item $x=c_r$.
\end{itemize}
برای تایپِ
$p(x)$
 در
$S_1(T_\mathbb{Q})$
حالات زیر متصوَر است.
\par 
اگر
$r\in \mathbb{Q}$
موجود باشد، به طوری که 
$``x=c_r"\in p$، 
آنگاه واضح است که 
\[
[x=r]=\{p(x)\}.
\]
\item
اگر
برای هر
$r\in \mathbb{Q}$
داشته باشیم
$``x\not=c_r"\in p$
آنگاه مجموعه‌های زیر را در نظر بگیرید:
\begin{align*}
& U_p=\{s\in \mathbb{Q}|``x<s"\in p\}
\\
& 
L_p=\{s\in \mathbb{Q}|``x>s"\in p\}
\end{align*}
توجه کنید که
$L_p\cap U_p=\emptyset$؛
نیز اگر
$x<s\in p$
و
$x>c_t\in p$
آنگاه
$T_\mathbb{Q}\vdash t<s$.
بنابراین
$L_p<U_p$.
نیز
$L_p\cup U_p=\mathbb{Q}$.
\par 
اگر
$U_p=\emptyset$
آنگاه 
$T_\mathbb{Q}\cup \{x>c_t\}_{t\in \mathbb{Q}}$
سازگار است و 
$p$
را تایپِ 
$+\infty$
می‌خوانیم. این تایپ بیانگر بزرگتر بودنِ
$x$
ازهمه‌ی 
$r\in Q$
است.
\par 
به طور مشابه تایپِ
$p=-\infty$
در حالتی که 
$L_p=\emptyset$
تعریف می‌شود.
\par 
اگر هر دوی
$L_p,U_p$
ناتهی باشند و 
$L_p$
داری ماکزیزمم باشد و 
$\max L_p=r$
آنگاه 
$p$
را
با
$r^+$
نشان می‌دهیم. این تایپ بیانگر نزدیکیِ بودن
$x$
از طرف راست
به عدد گویای 
$r$
است. 
\par 
به طور مشابه تایپ
$r^-$
در صورتی که
$U_p$
دارای عنصر کمینه باشد تعریف می‌شود. 
\par 
در صورتی که نه
$U_p$
مینیموم داشته باشد و 
$L_p$
ماکزیموم، تایپ
$p$
را تایپ اصم می‌خوانیم. تعداد اینگونه تایپها
$2^{\aleph_0}$
است. 
\par 
مطالب بالا را به صورت زیر جمعبندی می‌کنیم:
\[
S_1(\mathbb{Q})=S_1(T_\mathbb{Q})=\{-\infty\}\cup \{+\infty\}\cup
\{x=r\}_{r\in \mathbb{Q}}\cup \{r^-\}_{r\in \mathbb{Q}}\cup \{r^+\}_{r\in \mathbb{Q}}\cup \{r\}_{\text{$r$ اصم}}.
\]
بنابراین
$|S_1(T_\mathbb{Q})|=2^{\aleph_0}$؛
یعنی در این تئوری تعداد تایپهای تک‌متغیره حداکثرِ‌ممکن است. 
\end{mesal}
\begin{mesal}[تایپها در حساب پئانو]
هدفمان بررسی تایپهای تک‌متغیره در 
$T_\mathbb{N}:=\Th(\mathbb{N}, \{n\}_{n\in \mathbb{N}})$
است. 
فرمولِ
$\theta(x,y)$
را در نظر بگیرید که
\[
(\alpha,\beta)\models \theta \Leftrightarrow \alpha|\beta.
\]
برای یک زیرمجموعه‌ی دلخواهِ
$A$
از اعداد اول، مجموعه‌ی زیر از فرمولها را در نظر بگیرید:
\[
\pi_A=\{``p|x": p\in A\}\cup \{``p\not|x":p\not\in A\}.
\]
واضح است که 
$T_\mathbb{N}\cup \pi_A$
سازگار است، پس مجموعه‌ی یادشده یک تایپ جزئی است. توجه کنید که اگر
$A_1\not=A_2$
دو زیرمجموعه از اعداد اول باشند آنگاه
$\pi_{A_1}\cup\pi_{A_2}\cup T_\mathbb{N}$
ناسازگار است. بنابراین تایپهای کاملِ
$\mathbb{P}_{A_1},\mathbb{P}_{A_2}$
که از تکمیل تایپهای جزئی یادشده حاصل می‌آیند، با هم متفاوتند؛ یعنی به اندازه‌ی تعداد زیرمجموعه‌های اعداد اول می‌توان تایپ کامل پیدا کرد. پس
\[
S_1(T_\mathbb{N})=2^{\aleph_0}.
\]
\end{mesal}
در جلسات آینده علاوه بر آوردن مثالهای دیگری از تایپها، به تحلیل تئوریها 
با کمک توپولوژی استون روی فضای تایپهایشان خواهیم پرداخت. 
\begin{defn}[تایپهای ایزوله]
تایپِ
$p(\bar{x})\in S_n(T)$
را یک تایپِ
\textbf{ایزوله}
\LTRfootnote{isolated type}
می‌خوانیم هرگاه به عنوان عنصری از
$S_n(T)$
در توپولوژی استون ایزوله باشد؛ یعنی
$\{p\}$
مجموعه‌ای باز باشد. بنابراین اگر
$p$
ایزوله باشد، آنگاه فرمولِ
$\phi\in p$
چنان موجود است که 
$\{p\}=[\phi]$.
\end{defn}
وقتی تایپِ
$p$
توسط فرمول
$\phi$
ایزوله شود (یعنی هرگاه که 
$\{p\}=[\phi]$)
فرمول یادشده تکلیف تایپ را به طور کامل مشخص می‌کند؛ به بیان دیگر برای هر فرمولِ
$\psi(\bar{x})\in p$
داریم
\[
T\models \forall x\quad (\phi(\bar{x})\to \psi(\bar{x})).
\]
در
$DLO$
همه‌ی تاپیها ایزوله‌اند، زیرا تعداد تایپها متناهی است و از این رو همه‌ی نقاط به لحاظ توپولوژیک ایزوله‌اند. 
اگر
$S_n(T)$
نامتناهی باشد، حتماً دارای یک نقطه‌ی غیرایزوله است (زیرا توپولوژی استون فشرده است و اگر قرار باشد همه‌ی تایپها ایزوله باشند، پوششی نامتناهی از متشکل از 
مجموعه‌های باز تک‌نقطه‌ای برای
$S_n(T)$
یافت می‌شود که دارای هیچ‌ زیرپوشش متناهی‌ای نباشد).

\section{جلسه‌ی ششم، حذف تایپها}
در جلسه‌ی پیش گفتیم که تایپِ
$p(\bar{x})$
ایزوله است هرگاه تنها تایپِ شاملِ
یک فرمولِ
$\phi(\bar{x})\in p$
باشد.
هر تایپِ
ایزوله‌ی
$p(\bar{x})$
در تمام مدلهای
$T$
برآورده 
می‌شود
\LTRfootnote{is realised}:
عناصری چون
$a_1,\ldots,a_n$
که فرمولِ
ایزوله‌کننده‌ی
$\phi$
را برآورده می‌کنند، کُلِ تایپ را نیز برآورده می‌کنند. عکس این گفته تنها در صورتی برقرار است که زبان، شمارا باشد؛ یعنی
اگر
$L$
زبانی شمارا باشد و 
$T$
مدلی کامل در آن و
$p(\bar{x})$
در همه‌ی مدلهای 
$T$
برآورده شود، آنگاه 
$p(\bar{x})$
ایزوله است. باز به بیانی دیگر، اگر
$p(\bar{x})$
تایپی غیرایزوله باشد، آنگاه مدلی
چون
$\mathfrak{M}\models T$
چنان موجود است که 
$p$
در آن برآورده نمی‌شود؛ اصطلاحاً
$p$
در مدلِ
$\mathfrak{M}$
حذف می‌شود
\LTRfootnote{is omitted}.
\par 
گفتیم که تایپِ
$p$
را می‌توان مجموعه‌ای نامتناهی از فرمولها در نظر گرفت. 
پس برآورده شدن آن در
$\mathfrak{M}$
یعنی
\[
\mathfrak{M}\models \exists \bar{x}\quad \bigwedge_{\phi\in p} \phi(\bar{x}),
\]
و حذف شدن آن یعنی
\[
\mathfrak{M}\models \forall\bar{x}\quad \bigvee_{\phi\in p}\neg \phi(\bar{x});
\]
پس حذف شدن 
$p$
در
$\mathfrak{M}$
معادل این است که برای هر
$a_1,\ldots,a_n\in M$
فرمولی چون
$\phi(\bar{x})\in p$
موجود باشد، به طوری که
\[
\mathfrak{M}\models  \neg \phi(\bar{a}).
\]
\begin{thm}[حذف‌تایپها]
گیریم
$L$
زبانی باشد شمارا و 
$T$
یک تئوریِ
کامل در آن. اگر تایپِ
$p(\bar{x})$
غیرایزوله باشد، آنگاه مدل شمارای 
$\mathfrak{M}\models T$
چنان موجود است که 
$p$
در آن حذف شود.
\end{thm}
\begin{nokte}
قضیه‌ی بالا برای زبانهای ناشمارا برقرار نیست؛ مثال نقض آن را در جلسات بعد ارائه خواهیم کرد. 
\end{nokte}
معمولاً برای اثبات قضیه‌ی یادشده از ساختهای هنکینی استفاده می‌شود. اثبات قضیه را بدین روش، 
به عنوان یکی از پروژه‌ها به دانشجویان واگذاشته‌ایم تا در این جا به اثباتی 
توپولوژیک 
\footnote{هر چند در این اثبات نیز از ساختمانهای هنکینی بهره خواهیم جست!}
بپردازیم. 
\begin{nokte}
قضیه‌ی حذف تایپها
\LTRfootnote{omitting type theorem}
برای تعداد شمارا تایپ غیرایزوله به صورت همزمان  و در تعدادی شمارا متغیر نیز برقرار است (و مدل مورداشاره‌ در آن همچنان شماراست).
\end{nokte}
\begin{yad}
فضای توپولوژیکِ هاسدورفِ
$(X,\tau)$
را دارای ویژگیِ
بِئْر
\LTRfootnote{Baire property}
می‌خوانیم هرگاه در آن هر اشتراک شمارا از مجموعه‌های بازِ چگال، چگال باشد.
\end{yad}
معادلاً 
$(X,\tau)$
ويژگی بئر دارد هرگاه در آن هر اجتماع شمارا از مجموعه‌های بسته‌ی
	هیچجاچگال
\LTRfootnote{nowhere-dense}
هیچجاچگال باشد.
\footnote{منظور از مجموعه‌ی هیچجاچگال است آن است که بستارش هیچ نقطه‌ی درونی ندارد.}
نیز 
از آنالیز مقدماتی به خاطر دارید که هر فضای متریکِ کامل دارای ویژگی
بئر است. 
\par 
برای زبان شمارای 
$L$
مجموعه‌ی
\[
X_L:=\Th_L=\{T|\text{$T$ یک تئوری کامل در زبان $L$ است}\}
\]
را
به همراه پایه‌ی توپولوژیکِ
\[
B=\{[\phi]|\text{$\phi$ یک جمله در زبانِ $L$ است}\}
\]
در نظر بگیرید. فضای یادشده، فشرده و دارای پایه‌ای شمارا و از این رو جدایی‌پذیر
\LTRfootnote{separable}
است. 
\footnote{جدائی‌پذیر بودن یعنی دارا بودن یک زیرمجموعه‌ی شمارای چگال.}
هر فضای فشرده‌ی جدایی‌پذیر، لهستانی
\LTRfootnote{Polish space}
است؛ یعنی هومئومرف است با یک فضای متریک کامل جدائی‌پذیر. پس 
$X_L$
به ویژه دارای ویژگی بئر است. 
\par 
زبان شمارایِ
$L$
را با مجموعه‌ی شمارای
$C$
از ثوابت گسترش می‌دهیم تا به زبانِ
\mbox{$L(C)=L\cup C$}
برسیم.  تعریف می‌کنیم
\[ X^{L(C)}=\{T(C)|\text{$T(C)$ یک تئوری کاملِ  شاملِ $T$ در زبانِ $L(C)$ است}\}.\]
\begin{tam}
(به طور مستقیم)
نشان دهید که
$X^{L(C)}$
فشرده است.
\end{tam}
فضای
$X^{L(C)}$
فشرده، جدائی‌پذیر  و دارای ویژگی بئر است. قرار دهید
\begin{align*}
& \Gamma_1=\{T(C)\in X^{L(C)}|
\\
&
\text{$T(C)$ یک تئوری هنکینی در زبانِ
$L(C)$
است که جهانش مجموعه‌ی
$C$
است و در همین مجموعه‌ 
شاهددار است.
}
\end{align*}
یادآوری می‌کنیم که تئوریِ
$T$
یک تئوریِ دارای شواهد در
$C$
است هرگاه برای هر فرمولِ
$\phi(x,c_1,\ldots,c_n)$
ثابتِ
$c_\phi^{c_1,\ldots,c_n}$
موجود باشد به طوری که 
\[
T(C)\models \exists x\phi(x,c_1,\ldots,c_n)\to \phi(c_\phi^{c_1,\ldots,c_n},c_1,\ldots,c_n).
\]
\begin{tam}
روش هنکین را برای به دست آوردن یک تئوری دارای شواهد در یک مجموعه‌ی
$C$
مرور کنید.
\end{tam}
\begin{tam}
نشان دهید (با استفاده‌ی مستقیم از تعاریف که)
$\Gamma_1$
در
$X^{L(C)}$
چگال است.
\end{tam}
فرض کنید
$p(\bar{x})$
تایپی غیرایزوله باشد و قرار دهید
\begin{align*}
& \Gamma_2=\{T(C)\in X^{L(C)}|
\\
&  \text{برای هر
$c_1,\ldots,c_n\in C$ فرمولِ
$\sigma(\bar{x})\in p(\bar{x})$
چنان موجود است که 
$\neg \sigma(c_1,\ldots,c_n)\in T(C)$
}\}.
\end{align*}
\begin{claim}
$\Gamma_2$
چگال است.
\end{claim}
از ادعا نتیجه خواهد شد که 
$\Gamma_1\cap \Gamma_2$
چگال، و بالاخص ناتهی است. هر مدلی از 
یک تئوریِ هنکینیِ
$T(C)\in \Gamma_1\cap \Gamma_2$
تایپِ 
$p$
را حذف می‌کند. 
\begin{proof}[اثبات ادعا]
داریم
\[
\Gamma_2=\bigcap_{c_1,\ldots,c_n\in C} \Gamma_2^{c_1,\ldots,c_n}
\]
که در آن گرفته‌ایم
\[
\Gamma_2^{c_1,\ldots,c_n}=\{T(C)|
\text{$\sigma(\bar{x})\in p$
موجود است به  طوری که
$\neg \sigma(c_1,\ldots,c_n)\in T(C)$
}\}.
\]
ادعا می‌کنیم که هر
$\Gamma_2^{c_1,\ldots,c_n}$
باز و چگال است؛ آنگاه ویژگی بئر ما را به مطلوب می‌رساند.
\par 
\noindent
\textbf{باز بودن.}
داریم
\[
\Gamma_2^{c_1,\ldots,c_n}=\bigcup_{\sigma\in p}
\Gamma_2^{c_1,\ldots,c_n, \sigma}
\]
که در آن
منظور
از
$\Gamma_2^{c_1,\ldots,c_n, \sigma}$
مجموعه‌ی زیر است:
\[
\{T(C)|\neg \sigma(c_1,\ldots,c_n)\in T(C)\}.
\]
مجموعه‌ی بالا یک باز پایه‌ای است. 
\par \noindent
\textbf{چگال بودن.}
برای اثبات چگال بودن،‌ باید نشان دهیم که 
$\Gamma_2^{c_1,\ldots,c_n}$
با هر باز پایه‌ایِ
$[\theta(c'_1,\ldots,c'_n)]$
اشتراک ناتهی دارد.  یعنی برای هر جمله‌ی 
$\theta(c'_1,\ldots,c'_n)$
باید تئوریِ
$T(C)$
و فرمولِ
$\sigma(\bar{x})\in p$
را چنان یافت که
$\theta(c'_1,\ldots,c'_n)\wedge \neg \sigma(c_1,\ldots,c_n)\in T(C)$.
\par 
اگر 
$\theta\in p(\bar{x})$
آنگاه از آن جا که
$p$
غیرایزوله است فرمولِ
$\sigma\in p$
یافت می‌شود که از
$\theta$
نتیجه نشود. پس
$T\cup \{\neg \sigma,\theta\}$
سازگار است. پس تئوریِ هنکینی‌ای شامل
$T\cup \{\neg \sigma,\theta\}$
یافت می‌شود، و این همان مطلوب است.
\par 
اگر
$\neg\theta\in p$.
اگر برای هر
$\sigma\in p$
تئوریِ
$T\cup \{\neg \sigma,\theta\}$
ناسازگار باشد،‌ آنگاه برای هر
$\sigma\in p$
داریم
\[
T\models \theta\to \sigma;
\]
یعنی تایپِ 
$p$
ایزوله است، که این تناقض است. 
\end{proof}
\pagebreak 
\section*{درست نبودن قضیه‌ی حذف تایپها در زبانهای ناشمارا}
\begin{thm}[اِنگْلِر]
گیریم 
$T$
یک تئوری کامل مرتبه‌ی اول باشد در زبانی شمارا. گیریم
$p$
تایپی کامل باشد و 
$\Sigma$
زیرمجموعه‌ای از آن. اگر
$\Sigma$
در همه‌ی مدلهای 
$T$
برآورده شود، آنگاه 
فرمولی در
$p$
آن را ایزوله می‌کند.
\end{thm}
در ادامه نشان داده‌ایم که شرط شمارا بودن زبان برای
قضیه‌ی اِنگْلِر لازم است.
\par 
زبانِ
$L=\{X,Y,(c_i)_{i\in \omega}, (d_i)_{i\in \omega_1}\}$
را در نظر بگیرید که در آن
$X,Y$
دو محمولند و 
$(c_i)_{i\in \omega}$
و
$(d_i)_{i\in \omega_1}$
ثوابت. 
فرض کنیم که اصول تئوریِ
$T$
بیانگر متفاوت بودن دوبه‌دوی همه‌ی ثوابت باشد. تایپ جزئیِ
$\Sigma$
متشکل از فرمولهای زیر را در نظر بگیرید:
\[
x\in X\wedge \{x\not=c_i\}_{i\in \omega}.
\]
گیریم تایپ جزئی یادشده 
دارای تکمیلی چون
$p$
باشد که در آن فرمولی چون
$\phi(x,c,d)$
هست به طوری که
$c\in X, d\in Y$
و
$\phi\vdash \Sigma$.
از آنجا که تایپ جزئی
$\Sigma$
هیچ ثابتی در
$Y$
ندارد، 
در زبانِ
$\{X,Y,(c_i)_{i\in \omega}\}$
داریم
$T\models (\forall y \phi(x,c,y))\to \Sigma$.
اما این تناقض است، زیرا تئوریِ
$T$
در این زبان دارای مدلی است که در آن
بخش
$X$
شماراست. 
\begin{comment}
فرض کنید
$X$
و
$Y$
دو مجموعه‌ی شمارای مجزا از هم باشند و
$Z$
مجموعه‌ای شمارا و نامتناهی از
نگاشتهای یک‌به‌یک میان
$X,Y$
 با این ویژگی که 
 برای هر
 $f\in Z$
 نگاشت متفاوتی چون
 $g\in Z$
 موجود باشد، به طوری که 
 $f$
 و
 $g$
 تنها در تعداد متناهی جایگاه با هم نامساوی باشند (یعنی آندو تقریباً همه‌جا مساوی باشند). قرار دهید
$A=X\cup Y\cup Z$
و رابطه‌ی 
$R$
را روی
$A$
به صورت زیر تعریف کنید:
\[
(x,y,z)\in R\Leftrightarrow z(x)=y.
\]
فرض کنید
$(a_n)_{n\in \omega}$
شمارشی از 
$X$
باشد و ساختارِ
$\mathfrak{A}=\langle A,(a_i)_{i\in \omega}\rangle$
 را در نظر بگیرید. نیز 
$\mathfrak{A}_2$
را توسیعی مقدماتی از
$\mathfrak{A}$ 
بیانگارید که در آن
$Y^{\mathfrak{A}_2}$
دارای اندازه‌ی
$\aleph_1$
است و ساختارِ
$\mathfrak{B}=\langle A_2,(b_i)_{i<\omega_1}\rangle$
را در نظر بگیرید که در آن
$(b_i)$
شمارشی است از
$Y^{\mathfrak{A}_2}$؛
و در نهایت قرار دهید
$T=Th(\mathfrak{B})$.
\par 
اگر
$\mathfrak{M}$
مدلی دلخواه از
$T$
باشد، موارد زیر در آن درستند:
\begin{itemize}
\item 
$X^\mathfrak{M},Y^\mathfrak{M},Z^\mathfrak{M}$
از هم مجزایند و تعبیر ثوابت متفاوت عناصر متفاوتیند.
\item 
عناصرِموجود در
$Z^\mathfrak{M}$
نمایندگان نگاشتهایی یک‌به‌یکند میان عناصرِ
$X,Y$
 (بنا به رابطه‌ی
$R$).
\item
برای هر عدد طبیعیِ
$n$ 
و هر عنصر
$z\in Z^\mathfrak{M}$
که
نماینده‌ی نگاشتی یک‌به‌یک میان عناصر
$X,Y$
است،   عنصری
چون
$z'\in Z^\mathfrak{M}$
موجود است که نماینده‌ی نگاشتی دیگر است که با
$z$
حداکثر در
$n$
جایگاه متفاوت است.  
\item
برخی عناصر در
$X^\mathfrak{M}$
تعبیر هیچ ثابتی نیستند.
\item 
هر جایگشتی چون
$\sigma$
از عناصرِ
 $Y$
 را
که تنها تعداد متناهی عنصر را جابجا کند، می‌توان به اتومرفیسمی از ساختارِ
$\langle A,X,Y,Z,R,(a_n)\rangle$
گستراند که همه‌ی عناصرِ
$X$
را نقطه‌وار حفظ می‌کند.
\end{itemize}
گیریم
$\Sigma$
مجموعه‌ی همه‌ی فرمولهای زیر (برای 
$n\in \omega$)
 باشد:
\[
(x\in X) \wedge (x\not= a_n)
\]
این مجموعه از فرمولها در همه‌ی مدلهای 
$T$
برآورده می‌شود (مطابق مورد چهارم در بالا). 
فرض کنیم
$p$
تکمیل‌شده‌ی
$\Sigma$
باشد که در آن فرمولی چون
$\phi\in p$
هست که 
$\Sigma$
را
ایزوله کند. فرمولِ
$\phi$
در هر مدلی از
$T$،
و به‌ ویژه در
$\mathfrak{B}$،
توسط عنصری نه‌تعبیرهیچ‌ثابتی برآورده می‌شود. بنا به آخرین مورد در بالا،
هر جایگشتی از عناصرِ
$Y$
منجر به فرمولی
معادل با
$\phi$
می‌شود. از آنجا که در
$\mathfrak{B}$
همه‌ی عناصر موجود در
$Y$
با ثوابت نشان داده شده‌اند،
می‌توان فرض کرد که 
 فرمولِ
$\phi$
هیچ ثابتی از نوع 
$Y$
ندارد (آنها را می‌توان با سور زدن کنار زد). بنابراین
فرمول یادشده در 
$\mathfrak{A}_2$،
و از این رو در هر
زیرسیستم مقدماتی از آن چون
$\mathfrak{A}$،
توسط عناصر غیرثابت برآورده می‌شود؛ و این به طور واضح
تناقض است. 
\end{comment}
\pagebreak
\section{جلسه‌ی هفتم}
\begin{thm}
\label{andazeyefazayetype}
در زبان شمارای 
$L$
اگر
$|S_n(T)|>\aleph_0$
آنگاه
$|S_n(T)|=2^{\aleph_0}$.
\end{thm}
در صورت پذیرش فرضیه‌ی پیوستار، قضیه‌ی بالا بدیهی است. بنا به فرضیه‌ی پیوستار اگر
$\aleph_0<\kappa\leq  2^{\aleph_0}$
آنگاه 
$\kappa=2^{\aleph_0}$؛
به بیان دیگر
$\aleph_1=2^{\aleph_0}$.
فرضیه‌ی پیوستار از اصول
ZFC
برای نظریه‌ی مجموعه‌ها مستقل است، یعنی خود و نقیضش با
ZFC
سازگارند. با وجود این، برخی زیرمجموعه‌های اعداد حقیقی در همان
ZFC
فرضیه‌ی پیوستار را برآورده می‌کنند. برای مثال اگر
$X\subseteq \mathbb{R}$
مجموعه‌ای بسته باشد آنگاه
$X$
یا شماراست یا
$|X|=2^{\aleph_0}$.
نیز اگر
$X\subseteq \mathbb{R}$
بورل باشد، آنگاه 
$X$
فرضیه‌ی پیوستار را برآورده می‌کند. بررسی اینگونه خوشرفتاریِ زیرمجموعه‌های اعداد حقیقی
رسالت شاخه‌ای از ریاضیات است به نام 
نظریه‌ی توصیفی مجموعه‌ها.
\LTRfootnote{descriptive set theory}
\begin{thm}[کانتور ــ‌ بندیکسون]
هر زیرمجموعه‌ی بسته‌ی
$A$
از فضای متریک تام 
\LTRfootnote{complete}
و جدائی‌پذیرِ
$(X,d)$
را می‌توان به صورت
اجتماع مجزایی چون
$U_1\cup U_2$
نوشت که در آن
$U_1$
مجموعه‌ای شمارا و باز (در
$A$
با توپولوژی زیرفضایی)‌ است و 
$U_2$
یک مجموعه‌ی بسته‌ی 
کامل.
\LTRfootnote{perfect}
\end{thm}
مجموعه‌ی بسته‌ی
$C$
را کامل می‌خوانیم هرگاه همه‌ی نقاط آن حدی باشند؛  به بیان دیگر هرگاه هیچ نقطه‌ی ایزوله‌ای نداشته باشد. (ثابت کنید که) هر زیرمجموعه‌ی بسته‌ی کامل از
$X$
دارای اندازه‌ی
$2^{\aleph_0}$
است. بنابر قضیه‌ی کانتور ــ‌ بندیکسون، 
\LTRfootnote{Cantor - Bendixon}
هر زیرمجموعه‌ی بسته‌ (تحت شرایط آن قضیه)‌ یا
شماراست یا دارای اندازه‌ی 
$2^{\aleph_0}$.
\begin{proof}[اثبات قضیه‌ی
\ref{andazeyefazayetype}
]
می‌دانیم که
$S_n(T)$
فشرده و جدائی‌پذیر، و از این رو لهستانی است. پس بنا به قضیه‌ی کانتور ــ‌ بندیکسون،
$S_n(T)$
یا شمارا و  یا دارای اندازه‌ی
$2^{\aleph_0}$
است.
\end{proof}
\subsection{تئوری رابطه‌های تک‌موضعیِ‌مستقل}
تئوریِ
$T$
را در زبانِ
$L=\{p_1(x),p_2(x),\ldots,\}$
مشتمِل بر اصول‌موضوعه‌ی زیر، برای عناصرِ دوبه‌دومتفاوتِ
$i_1,\ldots,i_n, j_1,\ldots,j_k\in \mathbb{N}$
در نظر بگیرید:
\[
\theta_{i_1,\ldots,i_n,j_1,\ldots,j_k}:\quad \exists x\quad \left( \bigwedge_{i=i_1,\ldots,i_n} p_i(x) \wedge \bigwedge_{j=j_1,\ldots, j_n} \neg p_j(x)\right)
\]
نخست توجه کنید که
$T$
سازگار است: ساختارِ
$\langle \mathbb{N}, p_0^\mathbb{N},p_1^\mathbb{N},\ldots\rangle$
که در آن گرفته‌ایم
\begin{align*}
& p_0^\mathbb{N}=\text{مضارب ۲}
\\
& 
\vdots \\
& 
p_k^\mathbb{N}=\text{مضاربِ $k$ اُمین عدد اول}
\end{align*}
مدلی برای
$T$
است. 
\begin{tam}\hfill 
\begin{itemize}
\item 
نشان دهید که
$T$
دارای حذف سور است.
\item 
نشان دهید که 
\[
T\models \exists^{\infty } x\quad  p_n(x).
\]
\end{itemize}
\end{tam}
هر تایپِ کامل در
$S_1(T)$
دقیقاً تعیین می‌کند که 
$x$
در کدام
$p_n$
ها
واقع و در کدام ناواقع است. بنابراین برای هر
$I\subseteq \mathbb{N}$
مجموعه‌ی
$p_I(x)$
تعریف‌شده در زیر یک تایپ کامل مشخص می‌کند:
\[
p_I(x)=\{p_i(x)|x\in I\} \cup \{\neg p_j(x)|j\not\in I\}.
\]
پس
$|S_1(T)|=2^{\aleph_0}$؛
و در نتیجه تئوری یادشده 
$\aleph_0$ ــ
جازم نیست (بنا به قضیه‌ای که در جلسات بعد ثابت خواهیم کرد).
هر
$p_I$
تایپی غیرایزوله است و در جلسه‌ی بعد نشان خواهیم داد که در نتیجه‌ی
این، تئوری
$T$
هیچ مدل اولی ندارد. 

\section{جلسه‌ی هشتم}
هدف از بحثهای این جلسه، ارائه‌ی دسته‌بندی‌ای است برای تئوریها بر حسب
برآورده شدن یا نشدن تایپها در آنها. در این نوع دسته‌بندی، مدلهایی را که حداقل
تعدادِ تایپ در آنها محقَق شود، کوچک، و آنهایی را که 
حداکثر تعداد تایپها را برآورده کنند، بزرگ خواهیم دانست. 
\begin{defn}
گیریم
$T$
کامل باشد و 
$\mathfrak{M}\models T$.
\begin{enumerate}
\item 
مدلِ
$\mathfrak{M}$
را 
اتمیک
می‌خوانیم هرگاه برای هر
$a_1,\ldots,a_n\in M$
تایپِ
$\tp^\mathfrak{M}(a_1,\ldots,a_n)$
ایزوله باشد. 
در یک مدل اتمیک، تعداد تایپهایی که برآورده می‌شوند، حداقلِ ممکن است. 
\item
مدلِ
 $\mathfrak{M}$
را اول 
\LTRfootnote{prime}
می‌خوانیم هرگاه به صورت مقدماتی 
(یعنی به وسیله‌ی نگاشتی مقدماتی)
 در همه‌ی مدلهای
$T$
بنشیند.
\end{enumerate}
\end{defn}
در ادامه‌ی این بحث، زبان را شمارا در نظر گرفته‌ایم.
\begin{prop}
هر مدلِ اول، اتمیک است.
\end{prop}
\begin{proof}
مدلِ
اولِ
$\mathfrak{M}$
را در نظر بگیرید. اگر
$\mathfrak{M}$
اتمیک نباشد، 
$a_1,\ldots,a_n\in M$
چنان یافت می‌شوند که 
$\tp^\mathfrak{M}(a_1,\ldots,a_n)$
ایزوله نباشد. از آنجا که زبان شماراست، بنا به قضیه‌ی حذف تایپها،
تایپِ یادشده در
مدلی شمارا مانند
$\mathfrak{N}\models T$
حذف می‌شود. از طرفی
$\mathfrak{M}$
در این مدل، مثلاً توسط نگاشتِ
مقدماتیِ
$f$،
به طور مقدماتی نشسته است؛ یعنی
\mbox{$\tp^\mathfrak{M}(a_1,\ldots,a_n)=\tp^\mathfrak{N}(f(a)_1,\ldots,f(a_n))$}،
و این حذف شدن تایپ را ناقض است. 
\end{proof}
\begin{prop}
موارد زیر با هم معادلند:
\begin{enumerate}
\item 
$\mathfrak{M}$
مدلی اول است.
\item 
$\mathfrak{M}$
مدلی اتمیک و شماراست.
\end{enumerate}
\end{prop}
اثباتِ 
$1\to 2$
را به دانشجو وامی‌گذاریم. برای اثباتِ
$2\to 1$
فرض کنید 
$M=(a_i)_{i\in \omega}$
مدلی اتمیک  و شمارا باشد و 
$\mathfrak{N}\models T$
مدلی دلخواه. از آنجا که
$\tp^\mathfrak{M}(a_0)$
ایزوله است، در هر مدلی و از جمله در
$\mathfrak{N}$
برآورده می‌شود. از این رو عنصرِ
$b_0\in \mathfrak{N}$
چنان موجود است که
$b_0\equiv a_0$؛
که منظور از علامت یادشده، عبارت زیر است:
\[
\langle \mathfrak{M},a_0\rangle\equiv \langle \mathfrak{N},b_0\rangle.
\]
حال اگر
$b_0,\ldots,b_{n-1}\in N$
چنان یافت شده باشند که
\[b_0,\ldots,b_{n-1}\equiv a_0,\ldots,a_{n-1}\quad (*)\] 
آنگاه، نشان می‌دهیم که عنصرِ
$b_n\in N$
چنان موجود است که
\[
a_0,\ldots,a_n\equiv b_0,\ldots,b_n.
\]
فرض کنیم تایپِ
$\tp^\mathfrak{M}(a_0,\ldots,a_n)$
توسط
فرمولِ
$\phi(x_0,\ldots,x_n)$
ایزوله شده باشد. از
\mbox{$\mathfrak{M}\models \phi(a_0,\ldots,a_n)$}
نتیجه می‌شود که
\[
\mathfrak{M}\models \exists x\quad \phi(a_0,\ldots,a_{n-1},x).\]
بنا به 
$(*)$
داریم
\[
\mathfrak{N}\models \exists x\quad \phi(b_0,\ldots,b_{n-1},x)\]
\begin{tam}
نشان دهید که اگر
$\mathfrak{N}\models \phi(b_0,\ldots,b_n)$
آنگاه 
\[
\langle \mathfrak{N}, b_0,\ldots,b_n\rangle\equiv \langle \mathfrak{M},a_0,\ldots,a_n\rangle\]
\end{tam}
\begin{tam}
نشان دهید که نگاشت
$f:M\to N$
که هر
$a_i$
را به
$b_i$
می‌برد، مقدماتی است. 
\end{tam}
دیدیم که طبق تعریف، هر مدل اول به طور مقدماتی در سایر مدلها می‌نشیند. بنابراین اگر
$\mathfrak{M},\mathfrak{N}$
دو مدل اول باشند، هر یک به طور مقدماتی در دیگری می‌نشیند؛ ولی از این ایزومرف بودنِ
$\mathfrak{M}$
و
$\mathfrak{N}$
نتیجه نمی‌شود.
\begin{tam}
مثالی از دو ساختارِ
$\mathfrak{
M},\mathfrak{N}$
بزنید که ایزومرف نیستند ولی به طور مقدماتی در یکدیگر می‌نشینند.
\end{tam}
با این همه، در زبان شمارا، این مطلوب برقرار است.
\begin{tam}
نشان دهید که اگر
$\mathfrak{M},\mathfrak{N}$
دو مدل اول برای
$T$
باشند، آنگاه
$\mathfrak{M}\cong \mathfrak{N}$.
\end{tam}
در ادامه به بررسی چند مثال پرداخته‌ایم. 
\begin{mesal}
قبلاً دیده‌ایم که در
$T=DLO$
مجموعه‌ی
$S_n(T)$
متناهی است، و در نتیجه همه‌ی تایپها در این تئوری ایزوله هستند. از این رو همه‌ی مدلهای
$DLO$
اتمیک هستند؛ به ویژه
$\langle\mathbb{Q},\leq\rangle$
یگانه مدلِ اول برای آن است. 
\end{mesal}
بعداً ثابت خواهیم کرد که تمام مدلهای یک تئوریِ
 $\aleph_0$ 
 ــ
جازم اتمیک هستند و یگانه مدلِ شمارای یک‌چنین تئوری‌ای، همواره اول است.
\begin{mesal}
در تئوریِ
روابط تک‌موضعی مستقل دیدیم که
$S_1(T)$
شامل هیچ تایپ ایزوله‌ای نیست. از این رو هیچیک از مدلهای تئوریِ یادشده اتمیک نیستند، و به ویژه این تئوری دارای مدل اول نیست.
\end{mesal}
\begin{mesal}
در زبانِ
$L=\{E\}$
تئوریِ
$T$
را به گونه‌ای در نظر بگیرید که بیانگر این  باشد که
$E$
رابطه‌ای هم‌ارزی است با نامتناهی کلاس و هر کلاس از این رابطه نیز نامتناهی است.
\begin{tam}
نشان دهید تئوری یادشده کامل، دارای حذف سور، و 
$\aleph_0$
ــ‌ جازم است (مورد آخر یعنی هر دو مدل شمارا از این تئوری با هم ایزومرفند).
\end{tam}
حال زبانِ
$L=\{E_1,E_2,\ldots\}$
و تئوریِ
$T$
در آن را در نظر بگیرید که بگوید که
هر
$E_i$
یک رابطه‌ی هم‌ارزی است و همه‌ی کلاسهای آن نامتناهی است و به‌علاوه 
هر
$E_i$
توسط
$E_{i+1}$
تظریف می‌شود؛ یعنی
$E_{i+1}\subseteq E_i$.
\begin{tam}
نشان دهید
$T$
کامل و دارای حذف سور است ولی
$\aleph_0$  ــ‌
جازم نیست.
\end{tam}
\begin{tam}
نشان دهید که
تایپِ جزئیِ
$\Sigma(x,y)$
متشکل از فرمولهای
$\{E_i(x,y)\}_{i\in \omega}$
و فرمولِ
$x\not=y$
غیرایزوله است.
\end{tam}
\begin{tam}
نشان دهید تئوری یادشده دارای مدل اول است و در این مدل اول،
\[
x=y\leftrightarrow \bigwedge_{i\in \omega} E_i(x,y).\]
\end{tam}
\end{mesal}
دقت کنید که رابطه‌ی
\[
E_\infty:=\bigwedge E_i(x,y)
\]
خود نیز یک رابطه‌ی هم‌ارزی است که بنا به تمرین بالا، تعبیر آن در مدل اول، تساوی است. (در مدلِ اشباع این تئوری، رابطه‌ی
یادشده دارای نامتناهی کلاس است).
\par 
تا اینجا با مدل اول آشنا شده‌ایم و دانسته‌ایم که هر مدل اتمیکِ شمارا اول است. در زیر محکی برای
وجودِ چنین مدلی ارائه کرده‌ایم.
\begin{thm}[وجود مدل اول]
موارد زیر با هم معادلند.
\begin{enumerate}
\item 
$T$
دارای مدل اول است.
\item 
$T$
دارای مدل اتمیک است.
\item 
برای هر
$n\in \mathbb{N}$
مجموعه‌ی متشکل از تایپهای کاملِ
ایزوله، در
$S_n(T)$
چگال است. 
\end{enumerate}
\end{thm}
توجه کنید که وجود مدل اول، متضمن کم بودن تعداد تایپها نیست؛ کمااینکه در
$\Th(\mathbb{N})$
دیدیم که 
$S_1(T)=2^{\aleph_0}$
و داریم
\begin{tam}
نشان دهید که 
$\langle \mathbb{N},+,\cdot,0,1\rangle$
مدلی اول برای تئوری یادشده است. 
\end{tam}
\pagebreak
\section{جلسه‌ی نهم}
تمرین زیر در جلسه‌ی آموختال به طور کامل مورد بحث قرار خواهد گرفت، ولی در اینجا ایده‌ای برای حل آن ارائه کرده‌ایم. 
\begin{tam}
آیا تئوریِ ساختارِ
$\langle \mathbb{R},\mathbb{Q},\leq \rangle$
دارای مدل اول است؟
\end{tam}
برای پاسخ به سوال بالا، یک اصلبندی کامل برای ساختار یادشده بنویسید (کامل بودن اصلبندی مورد نظر را می‌توانید با به‌کارگیری یک سامانه‌ی رفت‌وبرگشتی تحقیق کنید).
ادعا می‌کنیم که  
ساختارِ
$\langle \mathbb{Q}, P,\leq\rangle$
که در آن
$P$
به صورت زیر تعریف شده،  یک مدل اول برای ساختار یادشده است.
\[
P=\{\frac{p}{q}\in \mathbb{Q}|\text{$q$ توسط یک عدد اولِ
$r_{2k}$ عاد می‌شود}\}
\]
در بالا فرض کرده‌ایم که
$\{r_n\}$
شمارشی از همه‌ی اعداد اول باشد. نیز در نمایشِ
$\frac{p}{q}$
فرض کرده‌ایم که
$(p,q)=1$.
با به کارگیری یک سامانه‌ی رفت و برگشتی، نشان دهید که تئوری یادشده 
$\aleph_0$
ــ
جازم است و از این رو، مدلی که در بالا معرفی کرده‌ایم تنها مدل شمارای آن، و از این رو اول است.
\qed
\par 
حال به ادامه‌ی درس بازمی‌گردیم. 
فرض کنید 
$T$
یک تئوری کامل و فاقد مدل متناهی در زبانِ شمارای 
$L$
باشد. موارد زیر با هم معادلند:
\begin{enumerate}
\item 
$T$
دارای مدل اول است.
\item 
$T$
دارای مدلی اتمی است.
\item 
برای هر
$n\in \mathbb{N}$
مجموعه‌ی
$n$
ــ
تایپهای ایزوله در فضای
$S_n(T)$
چگال است. 
\end{enumerate}
\begin{proof}
$2\to 3$.
فرض کنید
$M\models T$
اتمی باشد و 
$[\phi(x_1,\ldots,x_n)]$
بازی پایه‌ای در
$S_n(T)$.
معلوم است که 
$T\cup \{\phi(\bar{x})\}$
سازگار است و  از این رو
$\bar{a}\in M$
چنان موجود است که
$M\models \phi(\bar{a})$.
پس
$\phi(\bar{x})\in \tp(\bar{a})$؛
بنا به اتمیک بودن مدلِ
$M$
تایپِ
$\tp(\bar{a})$
ایزوله است. 
\par 
$3\to 1$.
بنا بر آنچه در جلسه‌ی قبل ثابت کردیم، کافی است مدلی شمارا و اتمیک بیابیم. در این کار از رهیافتی توپولوژیک، اما بر پایه‌ی ساختمانهای هنکین بهره  پی خواهیم گرفت.    درجلسات قبل، قضیه حذف تایپ را به روش مشابهی ثابت کرده بودیم.
\par 
قرار دهید
\[
X=\{T(C)|\text{$T(C)$ یک تئوری کامل در  زبانِ
$L\cup C$ است که
$T$
را دربردارد}
\}
\]
با  توپولوژی استون،
$X$
فشرده و دارای ویژگیِ
بئر است. (بررسی کنید که)
مجموعه‌ی
$\Gamma_1$
تعریف شده در زیر، در
$X$
چگال است. 
\[
\Gamma_1=\{T(C)| \text{$T(C)$ هنکینی است.}\}
\]
نیز با کمک ویژگی بئر نشان می‌دهیم که مجموعه‌ی زیر چگال است.
\[
\Gamma_2=\{T(C)|\text{برای هر 
$\bar{c}\in C$
فرمولی کامل چون
$\sigma(\bar{c})$
در
$T(C)$
واقع است
}\}
\]
(در ادامه به طور دقیق تعریف خواهیم کرد که) منظور از فرمولِ کامل، فرمولی است که 
تایپی ایزوله کند. برای اثبات چگال بودنِ
$\Gamma_2$
آن را به صورت زیر در نظر می‌گیریم
\[
\Gamma_2=\bigcap_{c_1\ldots c_n\in C }\Gamma_2^{c_1\ldots c_n}
\]
 که در آن
 \[
 \Gamma_2^{c_1\ldots c_n}=\{T(C)| \text{$T(C)$ شامل فرمول کاملی چون
$\sigma(c_1,\ldots,c_n)$
است. 
 }
\} 
 \]
به بیان دیگر
\[
 \Gamma_2^{c_1\ldots c_n}=\bigcup_{\text{$\sigma$ فرمولی کامل}} \{T(C)|\sigma(c_1,\ldots,c_n)\in T(C)\}
\]
یا
$\Gamma_2^{c_1\ldots c_n}=\bigcup_{\text{$\sigma$ فرمولی کامل}} [\sigma(c_1,\ldots,c_n)]$.
پس
$\Gamma_2$
اشتراکی شمارا از مجموعه‌های باز و چگال، و از این رو چگال است. هر مدلِ واقع در
$\Gamma_1\cap \Gamma_2$
شمارا و اتمیک است. 
\end{proof}
\begin{tam}
$3\to 1$
را مستقیماً با روش هنکین (و نه با استفاده از روشهای توپولوژیک) ثابت کنید.
\end{tam}
\begin{defn}
فرمولِ
$\theta(x_1,\ldots,x_n)$
را نسبت به تئوریِ
$T$
\textbf{کامل}
\LTRfootnote{complete}
 می‌خوانیم هرگاه 
\begin{enumerate}
\item 
$T\models \exists\bar{x}\quad \theta(\bar{x})$.
\item
برای هر فرمولِ
$\phi(\bar{x})$
اگر
$T\models \exists \bar{x}\quad (\theta(\bar{x})\wedge \phi(\bar{x}))$
آنگاه 
$T\models \forall\bar{x}\quad (\theta(\bar{x})\to \phi(\bar{x}))$.
\end{enumerate}
\end{defn}
به عبارت دیگر، فرمول
$\theta(\bar{x})$
کامل خوانده می‌شود، هرگاه تایپ کاملی ایزوله کند. روی چنین فرمولی، نمی‌توان توسط هیچ فرمول دیگری انشعاب زد؛ یعنی، برای هر فرمولِ 
$\phi(\bar{x})$
اگر
$T\cup \{\theta\wedge \phi\}$
سازگار باشد، آنگاه 
$T\cup \{\theta\wedge\neg\phi\}$
لزوماً ناسازگار است. عموماً از تکنیک ساخت درخت، در اثبات قضایای مربوط به تعداد تایپها استفاده می‌شود. چنین رویکردی را در جلسه‌های آموختال بیشتر خواهیم کاوید، و در این جا تنها به مثالی از این نوع کاربرد بسنده کرده‌ایم. 
\begin{prop}
\label{deraxt1}
اگر تئوریِ
$T$
مدل اول نداشته باشد، آنگاه 
$n\in \mathbb{N}$
چنان موجود است که 
\mbox{$|S_n(T)|\geq 2^{\aleph_0}$}.
\end{prop}
به عبارت دیگر، اگر برای هر
$n\in \mathbb{N}$
داشته باشیم
$|S_n(T)|=\aleph_0$
آنگاه 
$T$
دارای مدل اول است.
\begin{proof}
اگر
$T$
دارای مدل اول نباشد،
$n\in \mathbb{N}$
موجود است به طوری که 
$n$ ــ
تایپهای ایزوله در
$S_n(T)$
چگال نیستند. پس بازی چون
$[\phi(\bar{x})]$
موجود است، فاقد هیچ تایپ ایزوله‌ای. پس فرمولی چون
$\theta(\bar{x})$
چنان موجود است،  که هر دو مجموعه‌ی
$T\cup \phi\cup \{\theta\}$
و
$T\cup\phi\cup \{\neg\theta\}$
سازگار باشند.
\[
\xymatrix{
& \theta & & \neg\theta\\
&& \phi \ar[ul]\ar[ur]
}
\]
حال توجه کنید که هر دو فرمولِ
$\phi\wedge\theta$
و 
$\phi\wedge\neg \theta$
ناکاملند. علت این است که اگر برای مثال،
$\phi\wedge \theta$
تایپی ایزوله کند، این تایپ در
$[\phi]$
واقع می‌شود، که این با فرض در تناقض است. بنابراین فرمولی چون
$\xi$
موجود است به طوری که 
$T\cup \phi\cup \{\theta\}\cup \{\xi\}$
و
$T\cup\phi\cup \{\theta\}\cup \{\neg \xi\}$
هر دو سازگارند؛‌ نیز فرمول
$\chi$
چنان موجود است که 
$T\cup\phi\cup \{\neg \theta\}\cup \{ \chi\}$
و
$T\cup\phi\cup \{\neg \theta\}\cup \{\neg \chi\}$
سازگارند. پس
 بر هر دو گره‌ی درخت بالا می‌توان انشعاب زد:
\[
\xymatrix{
\xi & & \neg \xi& & \chi & & \neg \chi\\
 & \theta\ar[ul]\ar[ur] & & & & \neg\theta\ar[ul]\ar[ur]& \\
 & & & \phi\ar[ull]\ar[urr]& & & \\
}
\]
فراروند بالا را می‌توان ادامه داد و به یک درخت نامتناهی رسید. از آنجا که تعداد گره‌های هر درختِ اینچنین
$\aleph_0$
و
تعداد شاخه‌های آن
$2^{\aleph_0}$
است و هر شاخه از درخت یک تایپ کامل مشخص می‌کند، تعداد تایپها، حداقل
$2^{\aleph_0}$
است.
\end{proof}
\pagebreak
\section*{تعداد تایپها و درخت}
گفتیم که اگر
$(s)_{s\in 2^{<\omega}}$
یک درخت باشد، آنگاه تعداد گره‌های آن
$\aleph_0$
و تعداد شاخه‌های آن
$2^{\aleph_0}$
است. بنابراین اگر در تئوریِ
$T$
درختی از فرمولها مانند 
\[
\xymatrix{
\xi & & \neg \xi& & \chi & & \neg \chi\\
 & \theta\ar[ul]\ar[ur] & & & & \neg\theta\ar[ul]\ar[ur]& \\
 & & & \phi\ar[ull]\ar[urr]& & & \\
}
\]
موجود باشد که تا نامتناهی ادامه یابد،‌ آنگاه تعداد تایپها در این تئوری بیشتر از یا مساوی با 
$2^{\aleph_0}$
است. نیز تأکید کردیم که برای ساخت درخت، همواره باید علتی برای شاخه زدن وجود داشته باشد. برای مثال در گزاره‌ی
\ref{deraxt1}
علت این که روی
$\phi$
می‌شد انشعاب زد،‌ این بود که 
$[\phi]$
تایپ ایزوله‌ای در برنداشت. همچنین (تحقیق کنید) که اگر
$\phi$
تایپی ایزوله کند، روی آن نمی‌توان منشعب شد. در درختی که در اثبات گزاره‌ی زیر آمده است، با نوع دیگری از دلایل امکان انشعاب مواجهیم.
\begin{prop}
\label{deraxt2}
اگر
$|S_n(T)|>\aleph_0$
آنگاه
$|S_n(T)|\geq 2^{\aleph_0}$.
\end{prop}
گزاره‌ی بالا را پیشتر در کلاس درس با روشی توپولوژیک ثابت کرده بودیم. در زیر 
(و در کلاس آموختال) دو
 اثبات مستقیم (با یک ایده‌ی واحد) برای آن آورده‌ایم.
 \begin{proof}[اثبات اول]
اگر تعداد تایپها ناشمارا باشد، فرمولی چون
$\phi$
موجود است که در ناشمارا تایپ واقع شود؛ آن را در ریشه‌ می‌گذاریم. 
\begin{claim}
اگر
$[\phi]$
ناشمارا باشد، آنگاه دو تایپِ متفاوتِ
$p_1,p_2$
شاملِ
$\phi$
موجودند، به طوری که برای هر
$\psi\in p_i$
مجموعه‌ی
$[\psi]$
ناشمارا باشد. 
\end{claim} 
\begin{proof}[اثبات ادعا]
فرمولِ
$\psi$
را کوچک بخوانید هرگاه
$[\psi]$
شمارا باشد. تعداد تاپیهایی که حداقل یک فرمولِ‌کوچک را شاملند، شماراست؛ زیرا زبان شماراست و اندازه‌ی مجموعه‌ی زیر کرانی برای این تعداد است:
\[
\bigcup_{\text{$\psi$ کوچک}}[\psi]
\]
پس  در
$[\phi]$
تایپهایی یافت می‌شوند که هیچ فرمول کوچکی ندارند. 
\end{proof}
 فرض کنیم که
$p_1,p_2$
دو تایپ متمایز باشند، هر دو تنها متشکل از فرمولهای غیرکوچک و شاملِ
$\phi$.
فرض کنیم
$\psi\in p_1$
و
$\neg \psi\in p_2$
دو سر شاخه‌ی یادشده باشند. روی آنها نیز می‌توان دوشاخه شد؛ زیرا آنها نیز فرمولهایی غیرکوچکند. 
 \end{proof}

\begin{proof}[اثبات دوم]
فرض کنید فرمولِ
$\phi$
در ناشمارا تایپ واقع شده باشد. تحقیق کنید که مجموعه‌ی زیر یک تایپ کامل است:
\[
p_1=\{\psi| \text{ناشماراست }[\phi\wedge\psi]\}
\]
بنابراین اگر
$\psi\not\in p_1$
آنگاه
$[\phi\wedge\neg\psi]$
ناشماراست. پس می‌توان روی
$\phi$
توسط
$\psi,\neg\psi$
منشعب شد. همین فراروند را برای فرمولهای
$\phi\wedge\psi$
و 
$\phi\wedge\neg\psi$
و الی‌آخر اجرا کنید. 

\end{proof}
\pagebreak
\section{جلسه‌ی دهم، آکَندِگی}
در جلسه‌ی پیش درباره‌ی 
مدلهای اول و اتمیک، چونان
مدلهای حداقلی بحث کردیم. در این جلسه، به مدلهای اشباع، بمثابه مدلهای حداکثری خواهیم پرداخت.
\par 
پیش از شروع بحث اصلی، کمی به فلسفه‌ی آکندگی می‌پردازیم. 
قبلاً درباره‌ی شناخت ذات از روی صفات و تناظر فلسفی آن با مفهوم تایپها گفتگو کرده‌ بودیم. گفتیم
گاهی ذات یک موجود در دسترس نیست، ولی می‌توان آن را از روی مجموعه‌ی صفاتش شناخت.  بزرگترین مجموعه‌ی ممکن از این صفاتِ ثبوتیه  و نقیضِ‌ صفات سلبیه‌ی یک موجود را می‌توان با خودِ آن موجود یکی در نظر گرفت. در جهانهای اشباع، برای هر مجموعه‌ی سازگار از صفات، موجودی هست که آن صفات وصف اویند.
در این جهان هر چه پیش آمدنش ممکن باشد، رخ می‌دهد و هر مجموعه‌ از صفاتی که با هم در تناقض نباشند، در حقیقت مجموعه‌ی صفات یک ذات بخصوص است. 
\par 
طبق تعریف، یک تایپ کامل در تئوریِ
$T$
 مجموعه‌ای از فرمولهای سازگار با این تئوری است. این مجموعه از فرمولها لزوماً در هر مدلی برآورده نمی‌شود. در واقع اگر
 $p(x)\in S^M(A)$
 تایپی کامل باشد در مدل 
 $M$
 و روی مجموعه‌ی پارامترِ
 $A$
 آنگاه مدلی چون
 $N\succ M$
 و در آن عنصری چون
 $a$
 موجودند به طوری که
 $p(x)=\tp^N(a/A)$.
 اگر مدلِ
 $M$
 به‌اندازه‌اشباع باشد، عنصرِ
 $a$
 را می‌توان در خودش جُست. به زبان دقیق بازگردیم:
 \par 
فرض کنیم
$\mathfrak{M}$
مدلی باشد از
$T$
و 
$A$
زیرمجموعه‌ای از آن. زبانِ
$L_A$
توسیعی از زبانِ
$L$
است که با افزودن یک ثابتِ
$c_a$
برای هر
$a\in A$
حاصل می‌شود. بدیهی است که 
$\mathfrak{M}$
را می‌توان با تعبیرِ طبیعی عناصرِ
$A$
به عنوان یک
$L_A$
ــ‌
ساختارِ
$\langle \mathfrak{M},a\rangle_{a\in A}$
 در نظر گرفت. قرار می‌دهیم
\mbox{$T_A:=\Th\langle \mathfrak{M},a\rangle_{a\in A}$}
 و تعریف می‌کنیم
\[S_n^\mathfrak{M}(A)=S_n(T_A).\]
هر
$p(\bar{x})\in S_n(A)$
را یک تایپ کامل با پارامتر در
$A$
می‌خوانیم. واضح است که اگر
$p(\bar{x})$
یک تایپ کامل با پارامتر در
$A$
باشد، آنگاه
$T_A\cup p(\bar{x})$
سازگار و از این رو برآورده‌شدنی در یک توسیعِ مقدماتیِ
$\mathfrak{N}$
از
$\mathfrak{M}$
است. به بیان دیگر، برای یکچنین تایپِ
$p(\bar{x})$
مجموعه‌ی
$\diag_{el}(\mathfrak{M})\cup p$
سازگار است. 
\begin{defn}[مدل آکنده]
مدلِ
$\mathfrak{M}$
را
$\omega$ ــ
آکنده یا
$\omega$ ــ
اشباع 
\LTRfootnote{$\omega$-saturated}
می‌خوانیم هرگاه برای هر زیرمجموعه‌ی متناهیِ
$A\subseteq M$
و هر
$n\in \mathbb{N}$،
هر تایپِ کاملِ
$p(\bar{x})\in S_n^\mathfrak{M}$
در
$M$
برآورده شود.
\end{defn}
\begin{mesal}
مدلِ اولِ 
$DLO$
یعنی
$\langle \mathbb{Q},\leq\rangle$
مدلی
$\omega$ ــ
آکنده است. فرض کنید
$\alpha_1,\ldots,\alpha_n\in \mathbb{Q}$
و
تئوریِ
$T=\Th(\langle \mathbb{Q},\leq,\alpha_1,\ldots,\alpha_n\rangle)$
و تایپِ
$p(\bar{x})\in S_n(T)$
 در نظر بگیرید. با استفاده از سامانه‌ای رفت و برگشتی، (تحقیق کنید که) می‌توان نشان داد که
 $T$
 یک تئوریِ
 $\aleph_0$
 ــ
 جازم است.
تایپِ
$p(\bar{x})$
در مدلی شمارا باید برآورده شود. این مدل شمارا ایزومرف با مدل اولِ تئوریِ
$T$
است. یعنی یگانه 
(به پیمانه‌ی ایزومرفیسم)‌
مدل  شمارای این تئوری، اشباع 
 نیز هست. 
\end{mesal}
\begin{tam}
نشان دهید که یگانه مدل شمارای یک تئوریِ
$\aleph_0$ ــ
جازم، (هم اول است و هم) 
$\omega$ ــ
اشباع است. اثبات قسمت داخل پرانتز آسان نیست و بعداً بدان خواهیم پرداخت).
\end{tam}
\begin{mesal}
دیدیم که ساختارِ
$\langle \mathbb{N},+,\cdot,<,0,1\rangle$
مدلی اول برای تئوریِ
حساب پئانو است. تایپ جزئی زیر را در نظر بگیرید
\[
\pi(x)=\{x>1,x>s(1),x>ss(1)),\ldots\}
\]
و فرض کنید
$p(x)$
کامل‌شده‌ی آن باشد. واضح است که 
$p(x)$
در
$\mathbb{N}$
برآورده نمی‌شود؛ پس
$\langle \mathbb{N},+,\cdot,<,0,1\rangle$
مدلی 
$\omega$ ــ
آکنده نیست.  نیز قبلاً ثابت کرده‌ایم که 
\mbox{$|S_1(\Th(\langle \mathbb{N},+,\cdot,<,0,1\rangle)|=2^{\aleph_0}$؛}
بنابراین تئوری یادشده هیچ مدل 
شمارای آکنده‌ای  ندارد (تعداد تایپهای متفاوت ناشماراست و از این رو نمی‌توان آنها را با  شمارا عنصر برآورده کرد).
\end{mesal}
\begin{defn}
مدل
$\mathfrak{M}$
را شمارای‌اشباع
\LTRfootnote{countably saturated}
 میخوانیم هرگاه هم شمارا و هم اشباع باشد. 
\end{defn}
مطابقِ مثال بالا، شرط لازم برای این که یک تئوری مدلی شمارای‌اشباع داشته باشد این است که برای هر
$n\in \mathbb{N}$
داشته باشیم
$|S_n(T)|\leq \aleph_0$.
(در جلسات بعد خواهیم دید که)
عکس این گفته نیز برقرار است. یعنی هر تئوری 
$T$
که در آن
$S_n(T)\leq \aleph_0$
یک مدل شمارای‌اشباع دارد. پس هر تئوری‌ای که یک مدل شمارای‌اشباع داشته باشد، 
دارای مدلی اول است (قبلاً ثابت کردیم که اگر یک تئوری
مدل اول نداشته باشد، تعداد تایپهای آن ناشماراست). پس
\begin{prop}
اگر
تئوریِ
$T$
دارای مدل شمارای‌اشباع باشد 
دارای مدل اول است.
\end{prop}
هرگاه ویژگی‌ای بر حسب تایپها بیان شده باشد، 
رسم معمول در نظریه‌ی‌مدلْ تحقیق آن است که
آیا
برقرای این ویژگی 
برای تایپهای تک متغیره، برقراری آن را در حالت کلی  نتیجه می‌دهد. برای مثال گفتیم که
اگر برای
هر
$n\in \mathbb{N}$
مجموعه‌ی
تایپهای ایزوله‌ی
$n$
ــ
متغیره‌، در
$S_n(T)$
چگال باشد، آنگاه تئوری مورد نظر دارای مدل اول است. آیا 
چگال بودن تایپهای ایزوله‌ی تک‌متغیره‌ در
$S_n(T)$
برای برقراری  این حکم کافی است؟ پاسخ این سوال منفی است. از طرفی گفتیم که مدلِ
$\mathfrak{M}$
اشباع است هرگاه برای هر
$n\in \mathbb{N}$
هر تایپِ
$p\in S_n^\mathfrak{M}(A)$
در آن برآورده شود. در زیر نشان داده‌ایم
که آکندگی از برآورده‌شدنِ تنها تایپهای تک‌متغیره نیز نتیجه می‌شود.
(مشابه این گفته برای ساده بودن یا وابسته بودن تئوریها نیز برقرار است، که پرداختن بدانها جزو چارچوب این درس نیست).
\begin{prop}
برای
$\mathfrak{M}\models T$
موارد زیر با هم معادلند.
\begin{enumerate}
\item
$\mathfrak{M}$
مدلی 
$\omega$ 
ــ 
اشباع است.
\item
برای هر مجموعه‌ی متناهی مانند
$A\subseteq M$
هر
$n$ ــ
تایپِ جزئیِ
$\pi(\bar{x})$
در
$\Th(\langle\mathfrak{M},a\rangle_{a\in  A})$
در
$M$
برآورده می‌شود.
\item
برای هر مجموعه‌ی متناهیِ
$A\subseteq M$
هر تایپِ کاملِ
$p(x)\in S_1^\mathfrak{M}(A)$
در
$M$
برآورده می‌شود.
\item 
حکم شماره‌ی ۳ برای تایپهای جزئی تک‌متغیره.
\end{enumerate}
\end{prop}
\begin{proof}
$3\to 1$.
فرض کنید 
بدانیم که
هر
$p(\bar{x})\in S_n^{\mathfrak{M}}(A)$
در
$M$
برآورده می‌شود و تایپِ
$p'(x_1,\ldots,x_{n+1})\in S_{n+1}^{\mathfrak{M}}$
داده شده باشد. تعریف کنید
\[
\exists y p'=\{\exists y \phi(x_1,\ldots,x_{n}, y)| \phi(x_1,\ldots,x_{n+1})\in p\}.
\]
(بررسی کنید که)
$\exists y p'$
یک
$n$ ــ
تایپ جزئی است، پس در
$M$
توسط عناصری چون
$\alpha_1,\ldots,\alpha_n$
برآورده می‌شود. روی مجموعه‌ی
$A\cup \{\alpha_1,\ldots,\alpha_n\}$
ادعا می‌کنیم که مجموعه‌ی زیر از فرمولهای دارای یک متغیر واحد تایپی جزئی است.
\[
p'(\bar{\alpha},x)=\{\theta(\alpha_1,\ldots,\alpha_n,x))|\theta(x_1,\ldots,x_n,x_{n+1})\in p'\}.
\]
در این صورت
این تایپ جزئی نیز بنا به فرض استقراء در
$M$
توسط عنصری چون
$\alpha_{n+1}$
برآورده خواهد شد و به آسانی می‌توان دید که 
$\alpha_1,\ldots,\alpha_{n+1}$
تایپِ
$p'$
را برآورده می‌کنند. 
\par 
برای اثبات ادعا، فرمولهای
$\theta_i(\alpha_1,\ldots,\alpha_n,x)$
را در مجموعه‌ی بالا در نظر بگیرید
 (\mbox{$i=1,\ldots,k$}).
برای هر
$i$
داریم
\[\exists y\quad \theta_i(x_1,\ldots,x_n,y)\in \exists y p'\]
و
از آنجا که 
$p'$
تایپی کامل است،
\[
\bigwedge_i \theta_i (x_1,\ldots,x_n,x_{n+1})\in p'.
\]
بنابراین
\[
\exists y\quad \bigwedge_i \theta_i(x_1,\ldots,x_n,y)\in \exists y p'.
\]
فرمول بالا بنا به فرض استقراء در
$M$
برآورده می‌شود.
\end{proof}
برای هر تئوریِ کاملِ
$T$
مدلی 
$\omega$ 
ــ
اشباع لزوماً موجود است:
\begin{prop}
\label{vojudeakande}
اگر
$\mathfrak{M}\models T$
آنگاه توسیعی مقدماتی مانند
$\mathfrak{N}\succ \mathfrak{M}$
موجود است چنانکه 
$\mathfrak{N}$
مدلی است
$\omega$ ــ
اشباع و 
$|N|\leq |M|^{\aleph_0}$.
\end{prop}
با ترکیب گزاره‌ی بالا با لم لونهایم اسکولم، می‌توان مدلِ
$\mathfrak{N}$
را با اندازه‌ی دقیقاً‌برابر با
$|M|^{\aleph_0}$
به دست آورد.
\begin{proof}[اثبات کامل را در جلسه‌ی بعد خواهیم دید؛ در این جا به یک راهنمایی برای اثبات بسنده می‌کنیم.]
\hfill \newline
\textbf{قدم اول:}
توسیع مقدماتیِ
$\mathfrak{N}\succ \mathfrak{M}$
را چنان بیابید که
اولاً
$|N|\leq |M|^{\aleph_0}$
و ثانیاً برای هر زیرمجموعه‌ی متناهیِ
$A\subseteq M$،
هر تایپِ
$p\in S_1^\mathfrak{M}(A)$
در
$N$
برآورده شود. 
\newline
\textbf{قدم دوم.}
قدم اول را به مدلِ
$\mathfrak{N}$
اعمال کنید.
\newline
\textbf{قدم سوم.}
بررسی کنید که
اگر
$\mathfrak{N}_1\prec \mathfrak{N}_2\prec \ldots$
زنجیر حاصل‌شده بدین طریق باشد، آنگاه
$\bigcup \mathfrak{N}_i$
مدل مطلوب است. 
\end{proof}
\pagebreak
\section{جلسه‌ی یازدهم}
پیش از ورود به بحث، در زیر برای یادآوری تعاریفی معادل برای مفهوم تایپ آورده‌ایم.
فرض کنید
$\mathfrak{M}\models T$
و
$\bar{a}\in M$
و
$A\subseteq M$.
\begin{enumerate}
\item 
می‌نویسیم
$p(\bar{x})\in S^\mathfrak{M}_n(A)$،
و می‌گوییم که
$p(\bar{x})$
یک تایپ کامل در
$\mathfrak{M}$
روی
$A$
است، هرگاه 
$p(\bar{x})$
یک مجموعه‌ی سازگار بیشینال از فرمولها باشد با متغیرِ
$\bar{x}$
که با
$\Th(\langle \mathfrak{M},c\rangle_{c\in A}$
سازگار است. طبق این تعریف،‌ اگر
$\mathfrak{M}\prec \mathfrak{N}$
آنگاه 
\[
p(\bar{x})\in S^\mathfrak{M}_n(A)\Leftrightarrow p(\bar{x})\in S^\mathfrak{N}_n(A)
\]
می‌گوییم دو عنصرِ
$a,b\in M$
روی
$A$
همتایپند، و می‌نویسیم
$a\equiv_{A}b$
هرگاه
تایپی چون
$p(x)\in S^\mathfrak{M}_n(A)$
چنان موجود باشد که هر دوی
$a,b$
آن را برآورده کنند؛ معادلاً هرگاه
توسیع
$\mathfrak{N}\succ \mathfrak{M}$
و تایپی چون
$p(x)\in S^\mathfrak{N}_n(A)$
موجود باشد چنان که
$a,b$
هر دو 
$p(x)$
را برآورند. در این صورت،‌ نیز می‌نویسیم
$p(x)=\tp^\mathfrak{M}(a/A)=\tp^\mathfrak{M}(b/A)$.
\par 
فرض کنیم
$\mathfrak{N}\supseteq A$
نه لزوماً
توسیعی مقدماتی از
$\mathfrak{M}$
باشد و 
$b\in  N$.
در این صورت می‌نویسیم
$a\equiv_A b$
هرگاه هر دوی 
$a$
و
$b$
یک تایپِ
$p(x)\in S^\mathfrak{M}_n(A)$
را برآورده کنند. 
\item 
می‌نویسیم
$p(\bar{x})\in S^\mathfrak{M}_n(A)$،
و می‌گوییم که
$p(\bar{x})$
یک تایپ کامل در
$\mathfrak{M}$
روی
$A$
است، هرگاه توسیعِ مقدماتیِ
$\mathfrak{N}\succ \mathfrak{M}$
و عنصر
$\bar{b}\in N$
چنان موجود باشند که
$p(\bar{x})=\tp^\mathfrak{N}(\bar{b}/A)$
که در این جا، بنا به تعریف
\[
\tp^\mathfrak{N}(\bar{b}/A)=\{\phi(\bar{x})|\mathfrak{N}\models \phi(\bar{b})\}.
\]
\item 
برای هر دو مدلِ
$\mathfrak{M},\mathfrak{N}$
که 
$A\subseteq M,N$
و هر
$\bar{a}\in M$
و
$\bar{b}\in N$
تعریف می‌کنیم
$\bar{a}\equiv_A \bar{b}$
هرگاه مدلِ
$\mathfrak{K}$
و نگاشتهای مقدماتیِ
$f:\mathfrak{M}\to \mathfrak{K}$
و
$g:\mathfrak{N}\to \mathfrak{K}$
چنان موجود باشند که 
$g(\bar{a})=g(\bar{b})$.
در این صورت می‌نویسیم
$\tp^\mathfrak{M}(\bar{a}/A)=\tp^\mathfrak{N}(\bar{b}/A)$.
\item 
برای دو چندتاییِ
$\bar{a},\bar{b}\in M$
می‌نویسیم 
$\bar{a}\equiv_A\bar{b}$
هرگاه یک توسیع مقدماتیِ
$\mathfrak{N}\succ \mathfrak{M}$
موجود باشد به همراه یک اتومرفیسمِ
$\sigma:\mathfrak{N}\to \mathfrak{N}$
چنان که
$\sigma(\bar{a})=\bar{b}$.
\end{enumerate}
همان گونه که از تعاریف بالا برمی‌آید، 
تعریف تایپ
بسته به توسیعهای مقدماتی یک مدل است. 
اگر یک مدل سِتُرگ (فعلاً 
نه به معنای اصطلاحیش و تنها به معنی
بسیار بزرگ)، مثلاً به نام
$\mathbb{M}$
داشتیم که همه‌ی مدلها به طور مقدماتی در آن می‌نشستند به راحتی
می‌شد بگوییم 
هر تایپِ
$p(\bar{x})\in S^\mathfrak{M}_n(A)$
در واقع برابر است با
$\tp^\mathbb{M}(\bar{a}/A)$
برای یک
$\bar{a}\in \mathbb{M}$.
در جلسات بعد بدین نکته خواهیم پرداخت. 
\par 
در پایان جلسه‌ی پیش گزاره‌ی زیر را بیان و اثباتش را به این جلسه موکول کرده‌ بودیم.
\begin{prop}
\label{eshba'sizekuchak}
اگر
$\mathfrak{M}\models T$
آنگاه 
$T$
مدلی
$\omega$
ــ
اشباع مانند
$\mathfrak{N}\succ \mathfrak{M}$
دارد چنان که
$|N|\leq |M|^{\aleph_0}$.
\end{prop}
\begin{proof}
فرض کنید
$\langle p_\gamma(x)\rangle_{\gamma\in \lambda}$
شمارشی از همه‌ی تایپهای روی زیرمجموعه‌های متناهیِ
$M$
باشد. تعداد این چنین تایپها، حداکثر
$|M|\times 2^{\aleph_0}$
است. (تحقیق کنید که)
$\diag_{el}(\mathfrak{M})\cup \bigcup p_\gamma(c_\gamma)$
به طور متناهی ارضاء پذیر است ($(c_\gamma)$ را مجموعه‌ای از ثوابت در نظر گرفته‌ایم).
بنا بر فشردگی، مجموعه‌ی یادشده دارای مدلی از اندازه‌ی حداکثر (و بنا به لونهایم اسکولم از اندازه‌ی دقیقاً برابر با)
$|M|^{\aleph_0}$
است. این مدل را
$\mathfrak{N}_0$
می‌نامیم و روند بالا را بدان اعمال می‌کنیم تا به مدلِ
$\mathfrak{N}_1$
برسیم. گیریم
$(\mathfrak{N}_i)$
زنجیری باشد که بدین رهگذر حاصل شده است و قرار می‌دهیم
$\mathfrak{N}_\omega=\bigcup \mathfrak{N}_i$.
اگر
$\beta_1,\ldots\beta_n\in N_\omega$
و
$p(x)\in S^{\mathfrak{N}_\omega}(\beta_1,\ldots,\beta_m)$
باشد، آنگاه مدلی چون
$\mathfrak{N}_k$
برای یک
$k<\omega$
همه‌ی
$\beta_i$
ها را دربردارد. بنابراین
$p(x)$
در
$\mathfrak{N}_{k+1}$
(و از این رو در
$\mathfrak{N}_\omega$)
برآورده می‌شود. 
\end{proof}
\begin{nokte}
\hfill
\begin{enumerate}
\item
اگر
$\mathfrak{N}\models T$
مدلی
$\omega$
ــ
اشباع باشد و 
$\alpha_1,\ldots,\alpha_n\in N$
آنگاه
$\langle \mathfrak{N},\alpha_1,\ldots,\alpha_n\rangle$
مدلی 
$\omega$
ــ 
اشباع برای
$\Th(\langle \mathfrak{N},\alpha_1,\ldots,\alpha_n\rangle)$
است. 
\item 
اگر برای هر
$n\in \mathbb{N}$
مجموعه‌ی
$S_n(\Th(\mathfrak{N}))$
شمارا باشد،‌
آنگاه 
برای هر
$n\in \mathbb{N}$
مجموعه‌ی
$S_n(\Th(\langle \mathfrak{N},\alpha_1,\ldots,\alpha_k\rangle)$
نیز شماراست.
\end{enumerate}
\end{nokte}
\begin{proof}[اثبات شماره‌ی ۲]
نگاشتِ
\begin{align*}
&
S_n(\Th(\langle \mathfrak{N},\alpha_1,\ldots,\alpha_n\rangle)\to
S_{n+k}(T)\\
& p(x_1,\ldots,x_n,\alpha_1,\ldots,\alpha_k)\mapsto p(x_1,\ldots,x_n,x_{n+1},\ldots,x_{n+k})
\end{align*}
نگاشتی یک به یک است. شمارا بودنِ
فضای دامنه‌ی آن از شمارا بودن فضای بُردِ آن نتیجه می‌شود.
\end{proof}
بنا به تمرین زیر، ممکن است که در یک تئوریِ
$T$
مجموعه‌ی
$S_1(T)$
شمارا باشد ولی
$S_2(T)$
ناشمارا:
\begin{tam}
\hfill
\begin{enumerate}
\item 
نشان دهید که در
$\langle \mathbb{R},+,0\rangle$
تعداد تایپهای با یک متغیر، دو تاست و تعداد تایپهای با دو متغیر برابر با
$\aleph_0$.
\item 
نشان دهید که در
$\langle \mathbb{R},+,0,<\rangle$
تعداد تایپهای با یک متغیر، سه تاست و تعداد تایپهای با دو متغیر برابر با
$2^{\aleph_0}$.
\end{enumerate}
\end{tam}
در جلسه‌ی پیش همچنین اثبات قضیه‌ی زیر را وعده کرده‌ بودیم.
\begin{thm}
تئوریِ
$T$
دارای یک مدل شمارای‌اشباع است اگروتنهااگر 
$S_n(T)$
برای هر
$n\in \mathbb{N}$
شمارا باشد. 
\end{thm}
\begin{proof}
اگر
$T$
دارای مدل شمارای‌اشباعِ
$\mathfrak{M}$ 
باشد، برای هر
$n\in \mathbb{N}$
هر تایپِ در
$S_n(T)$
در
$M$
برآورده می‌شود. بنابراین 
$|S_n(T)|\leq \aleph_0$.
\par 
اگر هر
$S_n(T)$
شمارا باشد، بنا بر مورد دوم
در نکته‌ی بالا، برای هر
$\alpha_1,\ldots,\alpha_n\in M$
مجموعه‌ی
$S_1(\Th(\langle \mathfrak{M},\bar{\alpha}\rangle))$
نیز شماراست. پس مدل شمارای 
$\mathfrak{N}_1$
چنان موجود است که هر تایپِ متعلق به
$S_1(\Th(\langle \mathfrak{M},\bar{\alpha}\rangle))$
در آن برآورده شود. نیز مدلی شمارا چون
$\mathfrak{N}_2$
چنان موجود است که همه‌ی تایپهای متعلق به 
$S_1(\Th(\langle \mathfrak{N}_1,\bar{\alpha}\rangle))$
برای هر
$\bar{\alpha}\in N$
در آن برآورده می‌شوند. اجتماع زنجیرِ
$\mathfrak{N}_i$
هایی که از این رهگذر حاصل می‌شود،‌ مدل مطلوب است.
\end{proof}
\begin{tam}
نشان دهید که اگر
$T$
یک مدل اول داشته باشد که
$\omega$
ــ
اشباع باشد، آنگاه $T$
یک تئوریِ
$\aleph_0$
ــ
جازم است. 
\end{tam}
\begin{prop}
اگر
$\mathfrak{M},\mathfrak{N}$
دو مدل شمارای‌اشباع باشند آنگاه
$\mathfrak{M}\cong \mathfrak{N}$.
\end{prop}
\begin{proof}
گیریم
$M=(a_i)_{i\in \omega}$
و
$N=(b_i)_{i\in \omega}$.
از آنجا که
$\mathfrak{N}$
اشباع است، عنصری چون
$b$
را  چنان شامل است که
\[
b\equiv a_0
\]
به همین ترتیب از آنجا که
$\mathfrak{M}$
اشباع است، عنصری چون
$a$
را چنان شامل است که
\[
b_0b\equiv aa_0
\]
برای اثبات این گفته، فرض کنید
$\phi(x,b)$
فرمولی باشد که توسط
$b_0$
برآورده می‌شود. پس داریم
$\mathfrak{M}\models \exists x\quad \phi(x,b)$.
از آنجا که
$b\equiv a_0$
داریم
$\mathfrak{N}\models \exists x\quad \phi(x,a_0)$.
از این رو، اگر 
$p(x,b)=\tp(b_0/b)$
آنگاه هر بخش متناهی از 
$p(x,a_0)$
در
$N$
برآورده می‌شود و از آنجا که
$\mathfrak{N}$
شماراست، این تایپ در آن به‌کلی برآورده می‌شود. 
\par 
پس نگاشتِ
مقدماتیِ
$f_0$
را با ضابطه‌ی
$f_0(a_0)=b$
و
$f_0(a)=b_0$.
تعریف می‌کنیم (این نگاشت،
$a_0$
را در دامنه و 
$b_0$
را در بُرد دارد). فرض کنیم نگاشتِ مقدماتیِ
$f_n$
ساخته شده است که 
$a_{i\leq n}$
را در دامنه و
$b_{i\leq n}$
را در بُرد دارد. 
قرار می‌دهیم
$p(x,\bar{c})=\tp(a_{n+1}/\dom f_n)$
و بنا به مقدماتی بودنِ
نگاشتِ
$f_n$
عنصری چون
$b'\in N$
را
چنان می‌یابیم که
$b'\models p(x,f(\bar{c}))$.
به همین ترتیب 
قرار می‌دهیم
$q(x,b'f(\bar{c}))=\tp(b_{n+1}/b'f(\bar{c}))$
 و 
عنصری چون
$a'\in N$
می‌یابیم که
$a'\models q(x,a_{n+1},\bar{c})$.
نگاشت مقدماتیِ
$f_{n+1}$
را توسیعی از
$f_n$
می‌گیریم که
$a_{n+1}$
را به
$b'$
می‌برد و 
$a'$
را
به
$b_{n+1}$.
نگاشتِ
$f=\bigcup f_i$
نگاشتی یک و یک و پوشا و ضامنِ
ایزومرف بودنِ
$\mathfrak{M},\mathfrak{N}$
است. 
\end{proof}
\begin{nokte}
مدلهای اشباع را گاهی مدلهای
\textbf{ فشرده }
 نیز می‌خوانند، از آن جهت که هرگاه
$\pi(x)$
مجموعه‌ای از فرمولها باشد که به طور متناهی در یک مدلِ اشباعِ
$\mathfrak{M}$
برآورده می‌شود، آنگاه $M$
عنصری دارد که این تایپ را برآورده کند. 
\end{nokte}
\begin{prop}
فرض کنیم 
$\{M_i\}_{i\in I}$
خانواده‌ای از مدلهای تئوریِ
$T$
باشد و 
$F$
فرافیلتری روی
$I$.
آنگاه
$\prod_{F}M_i$
مدلی است
$\omega$
ــ‌
 اشباع. 
\end{prop}
گزاره‌ی بالا را در جلسه‌ی بعد ثابت خواهیم کرد. پیش از آن بدین نکته توجه می‌دهیم که مدلی که در گزاره‌ی بالا بدان اشاره شده است، در حقیقت،
$\omega_1$
ــ 
اشباع است؛ بدین معنی که هر تایپ روی یک زیرمجموعه‌ی شمارا  از آن در آن برآورده می‌شود.
\pagebreak
\section{جلسه‌ی دوازدهم}
\begin{defn}
فرض کنید
$\kappa\geq \aleph_0$
یک کاردینال نامتناهیِ دلخواه باشد. مدلِ
$\mathfrak{M}$
را 
\textbf{$\kappa$ 
ــ
اشباع }
می‌خوانیم هرگاه
برای هر زیرمجموعه‌ی
$A\subseteq M$
با
$|A|<\kappa$،
هر 
$n$ 
تایپِ
متعلق به
$S_n^\mathfrak{M}(A)$
در خودِ
$M$
برآورده شود.
\end{defn}
بنابراین مدلِ
$\mathfrak{M}$
را
$\aleph_1$ 
ــ
اشباع می‌خوانیم هرگاه همه‌ی تایپهای روی زیرمجموعه‌های شمارای 
$M$
در آن محقق شوند. نیز واضح است که هرگاه
$\mathfrak{M}$
مدلی
$\kappa$ 
ــ 
اشباع باشد، آنگاه 
$\lambda$
ــ
اشباع نیز برای هر
$\lambda<\kappa$
هست. منظور از مدلِ 
$\aleph_0$
اشباع نیز، دقیقاً همان است که پیشتر به نام
$\omega$
ــ
اشباع معرفی کرده‌ بودیم. 
\begin{defn}
مدلِ
$\mathfrak{M}$
را 
اشباع می‌خوانیم هرگاه
$|M|$
ــ‌
اشباع باشد.
\end{defn}
در تعریفِ
آکندگی، این که مجموعه‌ی پارامتر اندازه‌ی اکیداً
کمتر از
$\kappa$
داشته باشد ضروری است؛‌ برای مثال مجموعه‌ی زیر یک تایپ جزئی روی مجموعه‌ی پارامترِ
$M$
است که برآورده شدنش در 
$M$
میسر نیست:
\[
p(x)=\{x\not=m|m\in M\}.
\]
گزاره‌ای 
مشابه گزاره‌ی زیر  در بحثِ
$\omega$
ــ
آکندگی اثبات کرده بودیم و از این رو در زیر به بیان گزاره بسنده می‌کنیم.
\begin{prop}
برای
$\mathfrak{M}\models T$
موارد زیر با هم معادلند.
\begin{enumerate}
\item
$\mathfrak{M}$
مدلی 
$\kappa$ 
ــ 
اشباع است.
\item
برای هر مجموعه‌ی 
$A\subseteq M$
با اندازه‌ی اکیداً کمتر از
$\kappa$
هر
$n$ ــ
تایپِ جزئیِ
$\pi(\bar{x})$
در
$\Th(\langle\mathfrak{M},a\rangle_{a\in  A})$
در
$M$
برآورده می‌شود.
\item
برای هر مجموعه‌ی 
$A\subseteq M$
با اندازه‌ی اکیداً کمتر از
$\kappa$
هر تایپِ کاملِ
$p(x)\in S_1^\mathfrak{M}(A)$
در
$M$
برآورده می‌شود.
\item 
حکم شماره‌ی ۳ برای تایپهای جزئی تک‌متغیره.
\end{enumerate}
\end{prop}
\begin{tam}[وجود مدل اشباع]
برای هر
$\mathfrak{M}\models T$
مدلی 
$\kappa$
ــ
اشباع چون
$\mathfrak{N}\succ \mathfrak{M}$
چنان موجود است که 
$|N|\leq |M|^{\kappa}$.
\end{tam}
تمرین ــ‌ قضیه‌ی بالا مشابه گزاره‌ی
\ref{eshba'sizekuchak}
اثبات می‌شود با این تفاوت که
در اینجا به اجتماع زنجیری از مدلها با طول
$\kappa^+$
نیاز است. بررسی کنید که اگر
$\kappa$
کاردینالی منتظم
\LTRfootnote{regular}
 باشد، می‌توان با زنجیری از طول
$\kappa$
نیز به مطلوب رسید.
\begin{tam}
اگر
$\mathfrak{M},\mathfrak{N}$
دو مدل اشباع برای 
$T$
باشند و 
$|M|=|N|$
آنگاه 
$\mathfrak{M}\cong \mathfrak{N}$.
\end{tam}
\begin{mesal}
فرض کنید 
$\mathfrak{M}\models T$
و
$F$
یک فرافیلترِ غیراصلی روی
$\mathbb{N}$
باشد. آنگاه
$\prod_F \mathfrak{M}$
مدلی 
$\aleph_1$
ــ
اشباع است. 
\end{mesal}
\begin{mesal}[حکمی کلّی‌تر]
اگر
$F$
یک فرافیلترِ غیراصلی روی
$\mathbb{N}$
باشد و
$\{\mathfrak{M}^h\}_{h\in \mathbb{N}}$
خانواده‌ای از مدلها،‌ آنگاه
$\prod_F M^h$
مدلی
$\aleph_1$
ــ‌
اشباع است.
\end{mesal}
پیش از آنکه حکم مثال بالا را اثبات کنیم، یادآوری می‌کنیم که اگر
$\{\mathfrak{M}_i\}_{i\in I}$
خانواده‌ای باشد از مدلهای یک تئوریِ
$T$، 
آنگاه عناصرِ
$\prod_F \mathfrak{M}_i$
دنباله‌های
$(a_i)_{i\in I}$
هستند به هنگِ رابطه‌ی هم‌ارزی زیر:
\[
(a_i)\sim (b_i)\Leftrightarrow \{i|\mathfrak{M}\models a_i=b_i\}\in F.
\]
کلاسهای هم‌ارزی اینچنین را با نمادی چون
$[(a_i)]$
نمایش می‌دهیم.  بنا به قضیه‌ی واش
\LTRfootnote{Łoś's theorem}
 (که آن را در کلاس آموختال ثابت خواهیم کرد) داریم
 \[
 \prod_F \mathfrak{M}_i\models \phi([(a_i)_{i\in I}],\ldots,[(b_i)_{i\in I}])\Leftrightarrow \{i\in I|\mathfrak{M}\models \phi(a_i,\ldots,b_i)\}\in F.
 \]
 \begin{proof}[اثبات حکم مثال بالا]
 فرض کنید
 $\Sigma(x)$
 تایپی جزئی باشد با مجموعه‌ی پارامترِ شمارای
\mbox{$A=(a_i)_{i\in \omega}$}.
 فرض می‌کنیم
 $a_i=[(a_i^h)_{h\in \omega}]$.
 از آنجا که 
 $\Sigma$
 با
 $\Th(\prod_F \mathfrak{M}^h)$
 سازگار است، برای هر
 $i\in \omega$
 داریم
 $\prod_F \mathfrak{M}^h\models \exists x \quad \phi_0(x,a_0)\wedge \ldots\wedge \phi_i(x,a_i)$؛ 
 یعنی
مجموعه‌های
$D_i$،
تعریف‌شده در زیر، همه‌ اعضایی از
$F$
هستند.
 \[
 D_i:=\{h\in \omega| \mathfrak{M}^h\models \exists x\quad  \phi_0(x,a_0^h)\wedge \ldots \wedge \phi_i(x,a_i^h)\}
 \]
 برای این که نشان دهیم
 $\Sigma$
 در 
 $\prod_F \mathfrak{M}^h$
 برآورده می‌شود، کافی است عنصر
 $x=[(x^h)_{h\in \omega}]$
 را چنان بیابیم که برای هر
 $i\in \omega$
داشته باشیم
$\prod_F \mathfrak{M}^h\models \phi_i(x,a_i)\wedge \ldots\wedge\phi_0(x,a_0)$؛
به بیان بهتر چنان، که:
\[\{h\in \omega|\mathfrak{M}^h\models \phi_i(x^h,a_i^h)\wedge \phi_{i-1}(x^h,a_{i-1}^h)\wedge\ldots \wedge\phi_0(x_0,a_0^h)\}\in F.\]
طبق تعریف داریم
\[
D_0=
 \{h\in \omega| \mathfrak{M}^h\models \exists x\quad  \phi_0(x,a_0^h)\}
\] 
فرض کنیم
$x_0=(x_0^h)_{h\in \omega}$
 به گونه‌ای باشد که 
\[\forall h\in D_0\quad \mathfrak{M}^h\models \phi(x_0^h,a_0^h).\]
نیز طبق تعریف داریم:
\[
D_1=
 \{h\in \omega| \mathfrak{M}^h\models \exists x\quad  \phi_0(x,a_0^h)\wedge \phi_1(x,a_1^h)\}
 \]
  فرض کنیم
  $x_1=(x_1^h)_{h\in \omega}$
  از رهگذر زیر حاصل شده باشد:
  \begin{itemize}
  \item 
  $(x_1^h)_{h\leq min D_0}=(x_0^h)_{h\leq min D_0}$.
  \item 
  برای 
  $h>\min D_0$
  اگر
  $h\in D_1$
  آنگاه
  $x_1^h$
  را یکی از عناصری می‌گیریم که شاهد
 \mbox{$\mathfrak{M}^h\models \exists x\quad \phi_0(x,a_0^h)\wedge \phi_1(x,a_1^h)$}
  هستند. اگر
  $h\in D_0-D_1$
  آنگاه
  قرار می‌دهیم
  $x_1^h=x_0^h$.
  به سایرِ
  $h>\min D_0$
  دست نمی‌زنیم.
  \end{itemize}
تا اینجا داریم
$\{h\in \omega |\mathfrak{M}^h\models \phi_0(x_1^h,a_0^h)\}=D_0\in F$.
نیز 
\[\{h\in \omega |\mathfrak{M}^h\models \phi_1(x_1^h,a_1^h)\wedge \phi_0(x_1^h,a_0^h)\}\quad (*)\]
از دو حال خارج نیست. یا برابر با
$D_1$
است که در این صورت عضوی است از
$F$؛
و یا برابر است با
$D_1-\min D_0$.
به دو نکته توجه کنیم. از آنجا که فیلتر مورد نظر غیر اصلی است، برداشتن یک عضو از یکی از 
عناصر آن موجب خارج شدن از فیلتر نمی‌شود. زیرا
اولاً هر فیلتر غیراصلی شامل فیلتر فرشه است؛ پس
$\{i|i\geq \min D_0+1\}\in F$.
ثانیاً
$D_0\cap \{i|i\geq \min D_0+1\}\in F$
همان 
$(*)$
است.
\par 
بدین ترتیب برای تعریفِ
$(x_2^h)$
قرار می‌دهیم:
 \begin{itemize}
  \item 
  $(x_2^h)_{h\leq \min(D_1-\{\min D_0\})}=(x_1^h)_{h\leq \min (D_1-\{\min D_0\})}$.
  \item 
  برای 
  $h> \min(D_1-\{\min D_0\})$
  اگر
  $h\in D_1$
  آنگاه
  $x_2^h$
  را یکی از عناصری می‌گیریم که ضامنِ
  $\mathfrak{M}^h\models \exists x\quad \phi_0(x,a_0^h)\wedge \phi_1(x,a_1^h)\wedge\phi_2(x,a_2^h)$
  هستند. 
  \end{itemize}
  بدینسان اگر
  $(x_\omega^h)_{h\in \omega}$
  از ادامه‌ی همین روند به صورت استقرائی حاصل شود، در شرط مطلوب ما صدق می‌کند. 
  \end{proof}
  پیشتر درباره‌ی سامانه‌های رفت‌وبرگشتی و رابطه‌ی آنها با حذف سور و 
  هم‌ارز بودنِ مقدماتی صحبت کرده‌ بودیم (بخش بحثهای جانبی در تارنمای درس).
  گفته بودیم که سامانه‌های رفت و برگشتی، گاه
  از نگاشتهای مقدماتی جزئی تشکیل می‌شوند و گاه از ایزومرفیسمهای جزئی. گاه میان زیرمجموعه‌های یک مدل در نظر گرفته می‌شوند، گاه میان زیرمجموعه‌های دو مدل مختلف. گاه تحت اجتماعگیری بسته‌اند و گاه خیر،‌ و آنجا که تحت اجتماعگیری بسته باشند، ایزومرفیسم یا اتومرفیسم به دست می‌دهند. در زیر با مدلهای همگن آشنا می‌شویم که در‌ آنها، بنا به تعریف، هم‌ارزیهای کوچک در سامانه‌های رفت‌وبرگشتی واقعند. 
\begin{defn}[همگن]
مدل
$\mathfrak{M}$
را 
\textbf{$\omega$
ــ
همگن
\LTRfootnote{homogeneous}}
می‌خوانیم یا
$\aleph_0$
ــ
همگن می‌خوانیم هرگاه برای هر
$a_1,\ldots, a_n\in M$
و 
$b_1,\ldots,b_n\in M$
اگر
\[
\langle \mathfrak{M},a_1,\ldots,a_n\rangle \equiv \langle \mathfrak{M},b_1,\ldots,b_n\rangle
\]
آنگاه برای هر
$c\in M$
عنصر
$d\in M$
چنان یافت شود که
\[
\langle\mathfrak{M},a_1,\ldots,a_n,c\rangle \equiv \langle \mathfrak{M},b_1,\ldots,b_n, d\rangle.
\]
\end{defn}
به بیان دیگر مدل
$\mathfrak{M}$
وقتی
$\omega$
ــ
همگن است که مجموعه‌ی
$I$
در زیر یک سامانه‌ی رفت‌وبرگشتی از نگاشتهای مقدماتیِ جزئی باشد:
\[
\{(\bar{a},\bar{b}):|\bar{a}|=|\bar{b}|,  \tp^\mathfrak{M}(\bar{a})= \tp^\mathfrak{M}(\bar{b})\}.
\]
یا به بیان دیگر، در یک مدلِ
$\omega$
ــ
همگن، هر نگاشت مقدماتیِ جزئیِ
$f:\bar{a}\to \bar{b}$
در یک سامانه‌ی رفت‌وبرگشتی از نگاشتهای جزئیِ مقدماتی واقع است.
\begin{defn}
مدل
$\mathfrak{M}$
را
$\kappa$
ــ
همگن می‌خوانیم هرگاه برای هر
$\lambda<\kappa$
و هر
دو دنباله‌ی
$(a_i)_{i<\lambda}$
و
$(b_i)_{i<\lambda}$
از عناصرِ
$M$،
اگر
\[
\langle \mathfrak{M},(a_i)_{i<\lambda}\rangle \equiv \langle \mathfrak{M},(b_i)_{i<\lambda}\rangle
\]
آن گاه برای هر
$c\in M$
عنصر
$d\in M$
چنان موجود باشد که 
\[
\langle \mathfrak{M},(a_i)_{i<\lambda},c\rangle \equiv \langle \mathfrak{M},(b_i)_{i<\lambda},d\rangle.
\]
به بیان دیگر هر نگاشت مقدماتیِ
جزئیِ
$f:(a_i)_{i<\lambda}\to (b_i)_{i<\lambda}$
در سامانه‌ای رفت‌وبرگشتی از نگاشتهای مقدماتی جزئی میان زیرمجموعه‌های
$M$
واقع شود.
\end{defn}
\begin{defn}
مدلِ
$\mathfrak{M}$
را همگن می‌خوانیم هرگاه
$|M|$
ــ
همگن باشد.
\end{defn}
\begin{defn}
مدلِ
$\mathfrak{M}$
را
\textbf{ قویّاً 
$\kappa$
ــ
همگن }
می‌خوانیم هرگاه برای هر
$\lambda<\kappa$
چنانچه برای دو دنباله‌ی
$(a_i)_{i<\lambda}$
و
$(b_i)_{i<\lambda}$
از عناصرِ
$M$
داشته باشیم
\[
\langle \mathfrak{M},(a_i)_{i<\lambda}\rangle \equiv 
\langle \mathfrak{M},(b_i)_{i<\lambda}\rangle 
\]
آنگاه اتومرفیسمِ
$f\in \Aut(\mathfrak{M})$
چنان موجود باشد که
\[
\forall i<\lambda \quad f(a_i)=b_i.
\]
به بیان دیگر هرگاه هر نگاشتِ 
مقدماتی جزئی میان زیرمجموعه‌های کوچکتر از
$\kappa$
قابل گسترش به یک اتومرفیسم باشد. مدل
$\mathfrak{M}$
را قویّاً همگن می‌خوانیم هرگاه
$|M|$
ــ
قویاً‌همگن باشد.
\end{defn}
\begin{prop}
اگر
$\mathfrak{M}$
همگن باشد، قویاًهمگن است.
\end{prop}
قضیه‌ی بالا را در جلسه‌ی بعد ثابت خواهیم کرد. توجه کنید که در بالا ادعا\textbf{ نکرده‌ایم} که اگر
مدلی،
$\kappa$ ــ
همگن باشد، آنگاه
$\kappa$ 
ــ 
قویاً همگن است. 
\newpage
\section{جلسه‌ی سیزدهم}
\begin{yad}
مدلِ
$\mathfrak{M}$
را
$\kappa$
ــ
همگن خواندیم هرگاه برای هر
$\bar{a}$
و
$\bar{b}$
با
$|\bar{a}|, |\bar{b}|<\kappa$
اگر
\mbox{$\tp^\mathfrak{M}(\bar{a})=\tp^\mathfrak{M}(\bar{b})$}
آنگاه برای هر
$c\in M$
عنصر
$d\in M$
موجود باشد، به طوری که
\[\tp^\mathfrak{M}(\bar{a},c)=\tp^\mathfrak{M}(\bar{b},d).\]
به بیان دیگر، در یک مدلِ
$\kappa$
ــ
همگن هر نگاشت مقدماتی میان دو زیرمجموعه‌ای از اندازه‌ی کمتر از
$\kappa$
در سامانه‌ای رفت‌وبرگشتی  از نگاشتهای مقدماتی جزئی واقع می‌شود.
\end{yad}
در پایان جلسه‌ی قبل گفتیم (و در زیر ثابت خواهیم کرد)‌ که اگرچنانچه هر نگاشت مقدماتی
در سامانه‌ای رفت‌وبرگشتی از نگاشتهای مقدماتی واقع شود، آنگاه هر نگاشت مقدماتی را می‌توان به یک اتومرفیسم گستراند:
\begin{prop}
اگر
$\mathfrak{M}$
همگن باشد، قویاً همگن است.
\end{prop}
\begin{proof}
شمارشِ
$M=(m_i)_{i< |M|}$
را برای 
$M$
در نظر گرفته فرض کنید
$\tp^\mathfrak{M}(\bar{a})=\tp^\mathfrak{M}(\bar{b})$
و
$|\bar{a}|=|\bar{b}|<|M|$.
دنباله‌ی
$\langle f_i|i<|M|\rangle$
را
از نگاشتهای مقدماتی جزئی  به طریق زیر می‌سازیم. 
\par 
قرار می‌دهیم
$f_0=\{(a_t,b_t):{t<|\bar{a}|}\}$.
اگر
برای یک
$i<|M|$
دنباله‌ی
$(f_\lambda)_{\lambda<i}$
ساخته شده باشد، آنگاه
\begin{itemize}
\item 
 اگر
 $i$
 حدی باشد،‌قرار می‌دهیم
 $f_i=\bigcup_{\lambda<i} f_\lambda$.
 \item 
 اگر
 $i=\lambda+1$
 آنگاه
 به طریق زیر، از قرارگرفتنِ
 $m_\lambda$
 را در دامنه و 
 بردِ
 $f_i$
 اطمینان حاصل می‌کنیم:
 \par 
اگر
 $m_\lambda$
 هم در دامنه و هم در برُدِ
 $f_\lambda$
 باشد،
 آنگاه قرار می‌دهیم 
 $f_{i}=f_\lambda$.
 اگر
 $m_\lambda$
 در دامنه‌ی
 $f_\lambda$
 نباشد، آنگاه هم‌ارزی زیر را در نظر می‌گیریم:
 \[
 \dom(f_\lambda)\equiv \range (f_\lambda).
 \]
 از آنجا که
 $\mathfrak{M}$
 همگن است، می‌توان عنصرِ
 $d$
 را چنان یافت که
 \[
 \dom(f_\lambda)m_\lambda\equiv \range (f_\lambda)d.
 \]
 قرار می‌دهیم
 $f'_\lambda=f_\lambda\cup \{(m_\lambda,d)\}$.
 اگر
 $m_\lambda\in \range (f'_\lambda)$
 آنگاه قرار می‌دهیم
 $f_{i}=f'_\lambda$
 و در غیر این صورت، از هم‌ارزیِ
 \[
 \dom(f_\lambda)m_\lambda\equiv \range (f_\lambda) d
 \]
 و همگنیِ
 $\mathfrak{M}$
 استفاده کرده عنصرِ
 $d'$
 را چنان می‌یابیم که 
 \[
  \dom(f_\lambda)m_\lambda d'\equiv \range (f_\lambda) dm_\lambda
 \]
 و قرار می‌دهیم
 $f_i=f'_\lambda\cup \{(d',m_\lambda)\}$.
\end{itemize}
\begin{tam}
نشان دهید که 
$f=\bigcup_{\lambda<|M|} f_\lambda$
اتومرفیسمی از 
$\mathfrak{M}$
است.
\end{tam}
\end{proof}
پیشتر ثابت کرده‌ بودیم که هر مدلِ
$\mathfrak{M}$
را می‌توان در یک مدل 
$\omega$ 
ــ
اشباع نشاند که اندازه‌ی آن بزرگتر از
$|M|^{\aleph_0}$
نباشد. در زیر خواهیم دید که همگنی در همان اندازه‌ی
$M$
دست‌یافتنی است.
\begin{prop}
هر مدلِ
$\mathfrak{M}$
در مدلی 
$\omega$
ــ
همگن چون
$\mathfrak{M}\preceq \mathfrak{N}$
می‌نشیند که
$|N|=|M|$.
\end{prop}
\begin{proof}
نخست ادعا می‌کنیم که
مدل
$\mathfrak{N}$
چنان موجود است که
$|N|=|M|$
و برای هر دو دنباله‌ی متناهیِ
$\bar{a},\bar{b}$
از اعضای 
$M$
اگرچنانچه
$\tp^\mathfrak{M}(\bar{a})=\tp^\mathfrak{M}(\bar{b})$
آنگاه برای هر
$c\in M$
عنصرِ
$d\in N$
چنان موجودند که
\[
\tp^\mathfrak{M}(\bar{a}c)=\tp^\mathfrak{N}(\bar{b}d).
\]
برای اثبات ادعا‌، مجموعه‌ی
$I$
را به صورت زیر در نظر بگیرید:
\[
I=\{\langle \bar{a},\bar{b},c\rangle: |\bar{a}|=|\bar{b}|, \bar{a},\bar{b},c\in M,\tp^\mathfrak{M}(\bar{a})=\tp^\mathfrak{M}(\bar{b})\}.
\]
دقت کنید
که
$|I|=|M|$
و شمارش 
$\{\langle\bar{a}_i,\bar{b}_i,c_i\rangle\}$
را برای 
$I$
در نظر بگیرید. قرار دهید
$\Sigma_i(\bar{x},y_i)=\tp^\mathfrak{M}(\bar{a}_i,c_i)$
و توجه کنید که
$\bigcup \Sigma_i(\bar{b}_i,y_i)\cup \diag_{el}(\mathfrak{M})$
سازگار است (چرا؟ ـــ توجه کنید که هر بخش متناهی آن، بنا به 
همتایپیِ
$\bar{a}_i$
ها با
$\bar{b}_i$
ها در خودِ
$M$
برآورده می‌شود). بنابراین
گسترشی مقدماتی از
$\mathfrak{M}$
چون
$\mathfrak{N}$
و عناصرِ
$d_i$
در آن چنان موجودند که 
\[
\langle a_i,c_i\rangle \equiv \langle b_i,d_i\rangle \quad \forall i<|M|.
\]
برای راحت شدن ادامه‌ی بحث،‌
مدلی 
 را که با شروع از مدلِ
$\mathfrak{M}$
و اعمال روند بالا حاصل می‌شود، با 
$H(\mathfrak{M})$
نشان می‌دهیم.
\begin{yad}
می‌شد برای اثبات ادعای بالا، مدلِ
$\mathfrak{N}$
را از اجتماع زنجیری از مدلهای
$\mathfrak{N}'_i$
به دست آورد که در
هر
$N'_i$
عنصری چون
$d_i$
همتایپ با
$c_i$
موجود است.  
\end{yad}
\begin{tam}
ادعای بالا را با استفاده‌ی مستقیم از تعریفِ گالواتایپها
ثابت کنید.
\end{tam}
ادامه‌ی اثبات. زنجیرِ
$\langle \mathfrak{N}_i\rangle_{i\in \omega}$
را از مدلها، با استقرا و با قرار دادنِ
\mbox{$\mathfrak{N}_0=\mathfrak{N}$}
و 
\mbox{$\mathfrak{N}_{i+1}=H(\mathfrak{N}_i)$}
می‌سازیم. (بررسی کنید که)
\mbox{$\mathfrak{N}_\omega=\bigcup \mathfrak{N}_i$}
مدل مطلوب است.
\end{proof}
\begin{tam}
هر مدلِ
اتمیک،
$\omega$
ــ
همگن است؛ پس بویژه هر مدلِ اول، همگن است.
\end{tam}
\begin{tam}
هر مدلِ
$\kappa$
ــ
اشباع،
$\kappa$
ــ
همگن است؛ به ویژه هر مدل اشباع، همگن است.
\end{tam}
\begin{defn}[مدل جهانی]
مدل
$\mathfrak{M}$
را
\textbf{$\kappa$
ــ
جهانی }
می‌خوانیم هرگاه برای هر
$\mathfrak{N}\models T$
اگر
$|N|<\kappa$
آنگاه بتوان نگاشتی مقدماتی مانند
$f:\mathfrak{N}\to \mathfrak{M}$
یافت.
\end{defn}
\begin{tam}
اگر
$\mathfrak{M}$
مدلی
$\kappa$ ــ
اشباع باشد، آنگاه 
$\kappa^+$
ــ 
جهانی است.
\end{tam}
قضیه‌ی زیر در جلسه‌ی بعد ثابت خواهیم کرد.
\begin{thm}
مدلِ
$\mathfrak{M}$
مدلی
$\kappa$ ــ
اشباع است اگروتنهااگر
$\kappa$
ــ
همگن و 
$\kappa^+$
جهانی باشد.
\end{thm}
در حالت کلی اگر یک
خانواده‌ی
$(\mathcal{K},\leq)$
از مدلها
داشته باشیم که در آن
$\leq$
نوعی نشانش اعضای این خانواده باشد دارای ویژگیهای مطلوبی چون ادغام و  همنشانی،
آنگاه با درنظرگرفتن
گالواتایپها،   مشابه بالا 
اشباع بودن با همگن و جهانی بودن معادل است. در این باره در یکی از پروژه‌های درس صحبت خواهد شد. 
\chapter{قضیه‌ی مُرلی}
\section{جلسه‌ی چهاردهم،‌ لم رمزی}
در فصل قبل 
و در جلسات آموختال،
با یادگیریِ پیشنیازهای نظریه‌ی مدلی، ورزیدگی لازم  را برای 
 ورود به بحث اصلی کسب کرده‌ایم. 
از این نقطه‌ی درس به بعد با یادگیریِ 
مقدمات پیشرفته‌تری به سمت بیان و
اثبات قضیه‌ی مُرلی پیش خواهیم رفت.  نخست، به طور خلاصه به تبیین  ابزار ترکیباتی مورد نظر خود، یعنی قضیه‌ی رمزی می‌پردازیم.
\par 
عموماً در نظریه‌ی مدل، از قضیه‌ی رمزی
\LTRfootnote{Ramsey}
 و یا از تعمیمی از آن، به نام قضیه‌ی اردوش ــ‌ رادو
\LTRfootnote{Erd\"{o}s - Rado}
 برای یافتن دنباله‌های بازنشناختنی استفاده می‌شود. دنباله‌های بازنشناختنی، که در جلسات بعد مفصلاً  بدانها خواهیم پرداخت، دنباله‌هایی هستند که هر تعداد از اعضایشان، بسته به ترتیب 
 قرارگیریشان در دنباله،
از منظرِ تئوری مورد نظر، هم‌ارزش هستند. برای  یافتن این چنین دنباله‌ای، 
 عناصر یک دنباله‌ی دلخواه را،‌ بسته‌ به هم‌ارزش بودنشان نسبت به فرمولها، «رنگ‌آمیزی» می‌کنیم و با استفاده از لم‌ رمزی، زیرمجموعه‌ای «تک‌رنگ» از این دنباله استخراج می‌کنیم. 
\subsection{لم رمزی}
برای
مجموعه‌ی دلخواهِ
$X$ 
و عدد طبیعیِ
$k$
تعریف می‌کنیم:
\[
[X]^k=\{Y\subseteq X: |Y|=k\}.
\]
برای هر عددِ طبیعیِ
،$n\geq 1$ 
 هر تابعِ
\[
f:[X]^k\to n=\{0,1,\ldots,n-1\}
\]
را یک 
\textbf{رنگ‌آمیزی }
از زیرمجموعه‌های 
$k$
عضویِ مجموعه‌ی
$X$
توسط 
$n$
رنگ می‌خوانیم. 
مجموعه‌ی
$Y\subseteq X$
را برای رنگ‌آمیزیِ
$f$
همگن (یا تکرنگ)‌ می‌خوانیم هرگاه همه‌ی زیرمجموعه‌های
$k$
عضوی آن، تحتِ
$f$
رنگ یکسان داشته باشند؛ به بیان دیگر، هرگاه
$f|_{[Y]^n}$
ثابت باشد. پس اگر 
$Y$
مجموعه‌ای تکرنگ برایِ
رنگ‌آمیزیِ
$f$
باشد، عددِ
$i<n$
چنان موجود است که برای هر
$Z\subseteq Y$
با
$|Z|=n$
داریم
$f(Z)=i$.
\begin{thm}[رمزی]
فرض کنید مجموعه‌ی نامتناهیِ
$X$
و اعدادِ
$k,n\geq 1$
داده شده باشند.
برای 
هر رنگ‌آمیزیِ
$f:[X]^k\to n$
یک مجموعه‌ی تکرنگِ نامتناهیِ
$Y\subseteq X$
موجود است. 
\end{thm}
قضیه‌ی بالا، صورت نامتناهی لمِ رمزی است. عموماً در ترکیبیات، نخست صورت متناهی این لم را ثابت می‌کنند و از آن صورت نامتناهیش
را نتیجه‌ می‌گیرند، اما طرزفکرِ نظریه‌ی مدلی،  ما را بر آن می‌دارد که همواره برای یافتن رابطه میان متناهی و نامتناهی از قضیه‌ی فشردگی استفاده کنیم. 
\begin{yad}
حکم قضیه‌ی بالا در نمادگذاری زیر خلاصه می‌شود:
\[
\aleph_0\to (\aleph_0)^n_k
\]
\end{yad}
پیش از اثبات قضیه به بیان چند مصداق آشنا از آن می‌پردازیم.
\begin{mesal}
در حالتِ
$k=1$
قضیه‌ی رمزی همان اصل خانه‌ی کبوتری است: اگر نامتناهی عنصر در متناهی جایگاه قرار گرفته باشند،‌حداقل در یک جایگاه، نامتناهی عنصر جای گرفته است. 
\end{mesal}
\begin{mesal}
	فرض کنید
	$(V,R)$
	گرافی نامتناهی باشد. روی
	$[V]^2$
	یک رنگ‌آمیزی به صورت زیر تعریف کنید:
	\[
	f(\{x,y\})=
	\begin{cases}
1 & V\models R(x,y)	\\
0 & V\models \neg R(x,y).
	\end{cases}
	\]
	بنا به قضیه‌ی رمزی، یا زیرگرافی نامتناهی از گرافِ
	$V$
موجود است که همه‌ی رأسهای آن  دوبه‌دو به هم متصلند (زیرگراف کامل)، و یا زیرگرافی نامتناهی موجود است که هیچ یالی میان رأسهای آن 
وجود ندارد.
	\end{mesal}
\begin{mesal}
	با استفاده از قضیه‌ی رمزی ثابت کنید که هرگاه
	$(X,\leq)$
	یک مجموعه‌ی مرتبِ خطی نامتناهی باشد، آنگاه 
$X$
یا
شامل یک دنباله‌ی صعودی نامتناهی و یا شامل یک دنباله‌ی نزولی نامتناهی است.
	\end{mesal}
در ادامه، قضیه‌ی رمزی را با استقراء روی
$k$
ثابت کرده‌ایم.
\begin{proof}[اثبات قضیه‌ی رمزی]
	اگر
	$k=1$
	آنگاه قضیه‌ی رمزی همان اصل لانه‌ی کبوتری است. فرض کنیم حکم قضیه‌ی برای $k$ برقرار باشد؛ یعنی برای هر
	$n$
	داشته باشیم:
	\[
	\aleph_0\to (\aleph_0)^n_k.
	\]
	می‌خواهیم درستی آن را برای
	$k+1$
	تحقیق کنیم. 
فرض کنیم که
$X$
مجموعه‌ای نامتناهی باشد و
$f$
یک رنگ‌آمیزی از زیرمجموعه‌های
$k+1$
عضوی آن با 
$n$
رنگ. بدون کاسته شدن از کلیت، مجموعه‌ی
$X$
را شمارا و دارای ترتیبِ
$x_1<x_2<\ldots $
 فرض می‌کنیم. 
 در ادامه دو دنباله‌ی
 $(y_i)_{i\in \omega}$
 از عناصرِ
 $X$
 و
  $(n_i)_{i\in \omega}$
  از اعضای
 $\{0,1,\ldots,n\}$
 خواهیم ساخت و مجموعه‌ی مورد نظر را از میان عناصر دنباله‌ی اول استخراج می‌کنیم. قرار می‌دهیم
 $y_1=x_1$.
 \par 
 رنگ‌آمیزی
 $f_1:[X-\{x_1\}]^k\to n$
را با ضابطه‌ی
$f_1(Z)=f(Z\cup \{x_1\})$
در نظر می‌گیریم.
بنا به فرض استقراء، 
این رنگ‌آمیزی دارای یک مجموعه‌ی همگن به نامِ
$Y_1$
است. قرار می‌دهیم
$y_2=\min Y_1$
و
$n_2$
را رنگ مشترکِ همه‌ی زیرمجموعه‌های
$k$
عضویِ
$Y_1$
فرض می‌کنیم. به همین ترتیب روی
زیرمجموعه‌های
$k$
عضویِ
$Y_1-\{y_2\}$
رنگ آمیزیِ
$f_2(Z)=f(Z\cup \{y_2\})$
را در نظر گرفته فرض می‌کنیم که 
$Y_2$
مجموعه‌ی همگنِ آن و 
$n_3$
رنگ مشترک زیرمجموعه‌های
$k$
عضویِ
$Y_2$
 باشند. نیز قرار می‌دهیم
 $y_3=\min Y_2$.
 فراروند بالا را ادامه داده به عناصرِ
 \[
 y_1<y_2<\ldots
 \]
 می‌رسیم.
 نیز داریم
 \[
 Y_1\subseteq Y_2\subseteq \ldots.
 \]
 قرار دهید
 $Y'=\{y_1,\ldots\}$.
 هر زیرمجموعه‌ی
 $k+1$
 عضوی از
 $Y'$
 اگر شاملِ
 $y_1$
 باشد، به رنگِ
 $n_2$
 است و اگر شاملِ
 $y_i$
 باشد به رنگ 
 $n_{i+1}$.
از آنجا که تعداد رنگها متناهی است، یکی از رنگها بی‌نهایت بار تکرار می‌شود  و
مجموعه‌ی
$y_i$
های متناظر این رنگ، همان مجموعه‌ی مطلوب ماست.
	\end{proof}
\begin{thm}[رمزی متناهی]
برای  اعداد طبیعیِ
دلخواهِ
$n,m,k$
عددی طبیعی چون
$r(n,m,k)$
موجود است،‌ به طوری 
که
\[
r(n,m,k)\to (m)^k_n;
\]
یعنی 
بدان گونه که اگر 
$X$
مجموعه‌ای با اندازه‌ی حداقل
$r(n,m,k)$
باشد و 
$f$
یک رنگ‌آمیزی از زیرمجموعه‌های
$k$
عضوی آن با استفاده از
$n$
رنگ، آنگاه زیرمجموعه‌ای
از 
$X$
با اندازه‌ی حداقل
$m$
یافت می‌شود که همه‌ی زیرمجموعه‌های 
$k$
عضوی آن همرنگ هستند. 
\end{thm}
\begin{proof}
به برهان خلف، گیریم اعدادِ
$m,n,k$
چنان موجود باشند که مجموعه‌ای با هیچ‌اندازه‌ی متناهی یافت نشود که اگر زیرمجوعه‌های
$k$
عضوی آن را 
$n$
رنگ کنیم زیرمجموعه‌ای
همگن و 
$m$
عضوی پیدا شود. زبانِ
$L=\{R_1(v_1,\ldots,v_k),\ldots,R_n(v_,\ldots,v_k)\}$
و تئوری
$T$
را در آن با اصول زیر در نظر بگیرید:
\begin{align*}
&\forall x_1,\ldots,x_k\quad  R_i(x_1,\ldots,x_k)\to \bigwedge_{l\not=k\in \{1,\ldots,k\}} x_l\not=x_t\quad i\in \{1,\ldots,n\}
\\
&
\forall x_1,\ldots,x_k\quad  
R_i(x_1,\ldots,x_k)\to R_i(x_{\sigma(1)},\ldots,x_{\sigma(k)})\quad \sigma\in \textbf{جایگشت}(k),
i\in \{1,\ldots,n\}
\\
&
\forall x_1,\ldots,x_k \quad \left(\bigwedge x_l\not=x_t\to \bigvee_{i=1}^n R_i(x_1,\ldots,x_k)\right)\\
&
\forall x_1,\ldots,x_k  \quad \neg \left(R_i(x_1,\ldots,x_k)\wedge R_j(x_1,\ldots,x_k)\right)\quad  i\neq j.
\end{align*}
بنابراین اگر
$M\models T$
آنگاه هر 
$k$
عنصر از آن  توسط رابطه‌های
$R_1,\ldots,R_n$
در کلاسِ یک رنگ قرار می‌گیرند (عناصر هم رنگ را در یک کلاس قرار داده‌ایم).  جمله‌‌ی زیر را در نظر بگیرید:
\[
\psi_m:= \neg \left[\exists Y\quad |Y|\geq m\quad \wedge \quad 
\bigvee_i
\left(\forall y_1,\ldots,y_k\in Y \quad R_i(y_1,\ldots,y_k)\right)\right]. 
\]
جمله‌ی
$\psi_m$
می‌گوید که هیچ مجموعه‌ی همگنی از اندازه‌ی حداقل
$m$
وجود ندارد. 
تئوریِ
$T\cup \{\psi_m\}$
بنا به فرض،
دارای مدلهای متناهیِ
به‌اندازه‌ی‌دلخواه‌بزرگ، و بنا به فشردگی، دارای 
مدلی نامتناهی است.
وجود مدل نامتناهی برای این تئوری،
معادل وجود یک رنگ‌آمیزی از زیرمجموعه‌های
$k$
عضوی یک مجموعه‌ی نامتناهی است که هیچ زیرمجموعه‌ی همگن نامتناهی‌ای برایش یافت نشود،‌ و این لمِ رمزی نامتناهی را نقض می‌کند.
\end{proof}
قضیه‌ی رمزیِ متناهی از حساب پئانو اثبات شدنی است؛ 
لیکن 
صورتی پیشرفته‌تر از آن
\footnote{در این صورت روی مجموعه‌ی همگنِ
$Y$
قیدِ
$|Y|>\min Y$
گذاشته می‌شود.
}
 هست که
بنا به قضیه‌ای از
پاریس و هرینگتون
\LTRfootnote{Paris, Harrington}
با آنکه در
اعداد طبیعی درست است،‌
از حساب
پئانو قابل اثبات نیست. این صورتْ در واقع اولین مثال عینی برای ناتمامیت بوده است. 
\newpage 
\section{جلسه‌ی پانزدهم،‌ دنباله‌های بازنشناختنی}
در جلسه‌ی قبل ثابت کردیم که:
\begin{yad}[رمزی]
فرض کنیم 
$X$
یک مجموعه‌ی متناهی باشد و 
$[X]^n$
گردایه‌ی همه‌ی زیرمجموعه‌های
$n$
عضویِ آن. نیز فرض کنیم که  یک
رابطه‌ای هم‌ارزی روی
$[X]^n$
داریم که تعداد کلاسهای آن
متناهی
است. آنگاه زیرمجموعه‌ای نامتناهی چون
$Y\subseteq X$
چنان موجود است که همه‌ی اعضای
$[Y]^n$
در یک کلاس واقع باشند.
\end{yad}
تئوریِ
$T$
را کامل و زبانْ را شمارا انگاشته‌ایم. 
\begin{defn}
گیریم
$\mathfrak{M}\models T$
و
$A\subseteq M$
و فرض می‌کنیم که
$(I,\leq)$
یک مجموعه‌ی مرتب خطی باشد. دنباله‌ی
$(a_i)_{i\in I}$
از اعضای
$M$
را
نسبت به
مجموعه‌ی
$\Delta\subseteq L_A$
بازنشناختنی 
می‌خوانیم 
(یا آن را 
$\Delta$
بازنشناختنی می‌خوانیم)
هرگاه
برای هر دو دنباله‌ی
$i_1<i_2<\ldots<i_n\in I$
و
$j_1<j_2<\ldots<j_n\in I$
و هر فرمولِ
$\phi(x_1,\ldots,x_n)\in \Delta$
داشته باشیم
\[
\mathfrak{M}\models \phi(a_{i_1},\ldots,a_{i_n})\leftrightarrow \phi(a_{j_1},\ldots,a_{j_n}).
\]
اگر
$\Delta=L_A$
آنگاه دنباله‌ی یادشده را بازنشناختنی 
(روی
$A$)
می‌خوانیم. 
\footnote{ 
چنین دنباله‌ای را می‌توان «تمییزناپذیر» یا «تشخیص‌ناپذیر» هم خواند. ترجیح من همان واژه‌ی 
بازنشناختنی است. درواقع  یک دنباله‌ی این چنینْ
از اعضایی تشکیل شده است که توسط تئوری از هم بازشناخته نمی‌شوند. 

}
\end{defn}
به بیان دیگر، دنباله‌ی
$(a_i)_{i\in I}$
وقتی 
روی
$A$
بازنشناختنی است که در آن برای هر
$i_1<\ldots<i_n$
و
$j_1<\ldots<j_n$
داشته باشیم
\[
\tp^\mathfrak{M}(a_{i_1}\ldots a_{i_n}/A)=\tp^\mathfrak{M}(a_{j_1}\ldots a_{j_n}/A).
\]
باز به بیان دیگر، دنباله‌ی یادشده وقتی
روی
$A$
 بازنشناختنی است که هر تایپِ
$\tp^\mathfrak{M}(a_{i_1}\ldots a_{i_n}/A)$
تنها به تایپِ بدون سورِ
$\qftp_{DLO}(i_1,\ldots,i_n)$
بستگی داشته باشد. پس مثلاً اگر
$(a_i)_{i\in \omega}$
روی
$A$
بازنشناختنی باشد، داریم
\begin{align*}
& a_0 \equiv_A a_1 \equiv_A  a_2 \ldots \\
& a_0 a_1 \equiv_A a_0 a_2 \equiv_A a_0 a_3 \ldots \equiv_A a_1 a_2 \equiv_A a_1 a_3\ldots a_{240}a_{897}\ldots \\
& a_0 a_1 a_2 \equiv_A a_0 a_1 a_3\equiv_A a_1 a_2 a_3\ldots \equiv a_{10}a_{50}a_{62}\equiv_A \ldots
\end{align*}
\begin{mesal}[میدانهای بسته‌ی جبری]
فرض کنیم
$F\models ACF_0$
و 
$k\subseteq F$
زیرمیدانی از
آن باشد. هر دنباله‌ی
$(a_i)_{i\in \omega}$
از عناصرِ
$F$
که روی
$k$
متعالیند، روی
$k$
بازنشناختنی است. 
\par 
برای اثبات گفته‌ی فوق، توجه کنید که برای هر
$i_1<i_2<\ldots<i_n$
بنا به متعالی بودنِ عناصر دنباله داریم
$k(a_1, \ldots,a_n)\cong k(a_{i1},\ldots,a_{i_n})\cong k(X_1,\ldots,X_n)$.
بنا به حذف سور، این ایزومرفیسم در سامانه‌ای رفت‌وبرگشتی واقع است که همتایپیِ
$a_1\ldots a_n$
و
$a_{i_1}\ldots a_{in}$
را روی
$k$
ضامن می‌شود. 
\par 
به زبان ساده‌تر، توجه کنید که بنا به حذف سور، دو دنباله‌ی
$a_0\ldots a_n$
و
$a_{i0}\ldots a_{in}$
وقتی و تنها وقتی روی 
$A$
همتایپند که برای هر چندجمله‌ایِ
$p(x_0,\ldots,x_n)\in k[X_0,\ldots,X_n]$
داشته باشیم
\[
p(a_0,\ldots,a_n)=0 \Leftrightarrow p(a_{i0},\ldots,a_{in})=0;
\]
یعنی وقتی همه‌ی چندجمله‌ای‌ها درباره‌ی آنها هم‌نظر باشند. برقراری عبارت بالا 
برای دنباله‌ی ما
واضح است؛ زیرا بنا به متعالی بودن عناصرِ دنباله، برای هر
$i_0,\ldots,i_n$
و هر چندجمله‌ایِ چنان،
داریم
\[
p(a_{i0},\ldots,a_{in})\neq 0.
\]
\end{mesal}
\begin{mesal}[میدانهای بسته‌ی حقیقی]
گیریم
$\mathbb{R}^*$
توسیع (نااستانداردِ) مقدماتی‌ای از
$\langle\mathbb{R},+,\cdot,0,1,\leq\rangle$
باشد. بنا به حذف سور داریم
$\tp(a/\mathbb{Q})=\tp(b/\mathbb{Q})$
اگروتنهااگر
$a,b$
در نامساوی‌های یکسانی به شکلِ
$f(x)>0$
صدق کنند که
$f(x)\in\mathbb{Q}[X]$.
از طرفی، با دستکاریهای جبری می‌توان نشان داد که 
$f(x)>0$
معادل است با عبارتی چون
\mbox{
$x\in (a_1,b_1)\cup \ldots \cup (a_n,b_n)\cup \{c_1,\ldots,c_m\}$}
که در آن نقاط پایانی بازه‌ها را می‌توان به صورت تعریف‌پذیر
(حتی تنها به صورت چند جمله‌ای) بر حسب ضرایبِ
$f$
محاسبه کرد. این ویژگی، 
\textbf{ترتیب‌کمینگی}
\LTRfootnote{o-minimality}
 نام دارد. بنا به ترتیب‌کمینگی، هر زیرمجموعه‌ی
تعریف‌پذیرِ یک‌بعدی از
$\mathbb{R}^*$
اجتماعی است متناهی از باز‌ها و نقطه‌ها. پس یک دنباله‌ی
$(a_i)_{i\in \omega}$
از عناصرِ
$\mathbb{R}^*$
روی
$\mathbb{Q}$
بازنشناختنی است هرگاه عناصرِ آن همه در یک شکاف از
$\mathbb{Q}$
واقع باشند و دنباله‌ی یادشده اکیداً صعودی و یا اکیداً نزولی باشد.
\par 
توجه کنید که در خودِ
$\mathbb{R}$
نمی‌توان دنباله‌ای یازنشناختنی حتی روی مجموعه‌ی
تهی یافت. اگر قرار باشد که
$(a_n)$
بازنشناختنی باشد، برای هر
$\alpha\in \mathbb{Q}$
از آنجا که
$\alpha$
را می‌توان به صورت
$\frac{a}{b}$
در تئوری نوشت، باید داشته باشیم
$a_i<\alpha\Leftrightarrow a_j<\alpha$.
این در حالی است که برای هر دو جمله‌ی
$a_i<a_j$
در دنباله‌ی مورد نظر، عنصری چون
$\alpha\in \mathbb{Q}$
یافت می‌شود که 
$a_i<\alpha<a_j$.
\end{mesal}
\begin{mesal}[ترتیبهای خطی چگال]
از آنجا که
$DLO$
سورها را حذف می‌کند، هر دنباله‌ی صعودی در هر مدل آن، روی
تهی بازنشناختنی است. روی یک مجموعه‌ی داده‌شده‌ی
$A$
یک دنباله تنها در صورتی بازنشناختی است که اکیداً صعودی یا اکیداً نزولی باشد و اعضایش همه در شکاف یکسانی واقع شده‌ باشند. 
\end{mesal}
\begin{mesal}
اعضای یک دنباله‌ی بازنشناختنی نمی‌توانند جبری باشند. در واقع اگر
جمله‌ی اول جبری باشد، تایپِ آن تنها توسط تعداد متناهی عنصر برآورده می‌شود. از طرفی همه‌ی بقیه‌ی دنباله نیز تایپ آن را برمی‌آورند. پس دنباله‌ی یادشده باید متناهی باشد (دنباله‌های بازنشناختنی را نامتناهی فرض کرده‌ایم).
\end{mesal}
\begin{mesal}
اگر
دنباله‌ی
$(a_i)$
روی
$A$
بازنشناختنی باشد هر تصویر آن تحت اتومرفیسمی چون
\mbox{$f\in \Aut(M/A)$}
نیز روی
$A$
بازنشناختنی است.
\end{mesal}
\begin{mesal}
اگر
$K$
یک فضای برداری باشد و 
$\mathrm{a}=(a_i)_{i\in \omega}$
پایه‌ای نامتناهی از آن، آنگاه 
$\mathrm{a}$
دنباله‌ای بازنشناختنی است. 
\end{mesal}

در ادامه، به مسئله‌ی وجود دنباله‌های بازنشناختی می‌پردازیم. در زیر نشان داده‌ایم که 
$\Delta$ ــ
بازنشناختنی بودن،‌ برای یک مجموعه‌ی متناهیِ
$\Delta$
از فرمولها، به آسانی حاصل‌شدنی است.  
\begin{prop}
\label{indis1}
گیریم 
$\Delta$
مجموعه‌ای متناهی باشد از فرمولها و 
$(I,\leq)$
مجموعه‌ای مرتبِ خطی و 
$X=(a_i)_{i\in I}$
دنباله‌ای دلخواه در یک مدلِ
$M$.
آنگاه
$X$
دارای زیردنباله‌ای
$\Delta$
ــ
بازنشناختنی 
(مانند
$(b_j)_{j\in \omega}$)
است. 
\end{prop}
\begin{proof}
گیریم
$\Delta=\{\phi_1,\ldots,\phi_k\}$
که در آن متغیرهای آزادْ
$x_1,\ldots,x_n$
هستند،
و قرار می‌دهیم
\[
A=\{\psi(\bar{x})| \psi=\bigwedge_{i=1}^k \theta_i, \theta_i\in \{\phi_i,\neg\phi_i\}\}.
\]
مجموعه‌ی
$A$
متناهی است و برای هر
$\bar{\alpha}\in M$
فرمول یکتایی چون
$\psi\in A$
موجود است به طوری که
$\models \psi(\bar{\alpha})$.
روی
$[X]^n=\{\{a_{j1}<\ldots<a_{jn}\}|j_1<\ldots<j_n\in I\}$
رنگ‌آمیزیِ
$f(\{a_{j1}<\ldots,a_{jn}\})=\{\psi(\bar{x})|\models \psi(\{a_{j1}<\ldots<a_{jn}\})\}$
را در نظر می‌گیریم. بنا به قضیه‌ی رمزی،
$X$
زیرمجموعه‌ای متناهی چون
$Y$
دارد که همه‌ی زیرمجموعه‌های
$n$
عضوی آن همرنگند. دنباله‌ی
$Y$
همان دنباله‌ی مورد نظر است. 
\end{proof}
\begin{proof}[بیان دیگری برای اثبات]
روی
$[X]^n$
رابطه‌ی هم‌ارزی زیر را تعریف می‌کنیم:
\[
\{a_1<\ldots<a_n\}\cong \{b_1<\ldots<b_n\}\Leftrightarrow \tp_\Delta(\bar{a})=\tp_\Delta(\bar{b}).
\]
تعدادِ کلاسهای رابطه‌ی بالا (تعداد رنگها) متناهی است. زیرمجموعه‌ای از
$X$
که از اعمال لم رمزی حاصل می‌شود، دنباله‌ی مورد نظر است. 
\end{proof}
بنا به گزاره‌ی بالا، می‌توان از اندرون یک دنباله‌ی شمارا، یک دنباله‌ی بازنشناختنی نسبت به تعداد متناهی فرمول بیرون کشید. بنا به قضیه‌ی اردوش ــ رادو (که صورتی کلی‌تر است از رمزی و در این درس بدان نخواهیم پرداخت)
برای هر مجموعه‌ی
(نه لزوماً متناهی)
$\Delta$
از فرمولها 
اگر اندازه‌ی دنباله‌ای که با آن شروع می‌کنیم به قدر کافی
نسبت به اندازه‌ی
$\Delta$
 بزرگ باشد، می‌توان از دلِ آن دنباله‌ای با اندازه‌ی شمارا و در عین حال
$\Delta$ ــ
بازنشناختنی بیرون کشید.
\par 
روش معمول
دیگر (غیر از روش استفاده از لم اردوش ــ رادو)
 برای یافتن دنباله‌های بازنشناختنی،‌ آمیختن لم رمزی و لم فشردگی است. در زیر 
صورتی ساده از اعمال این روش را ارائه کرده‌ایم. در جلسات آینده صورتی کارگشاتر از قضیه‌ی زیر را بررسی خواهیم کرد که در آن ویژگی‌های دنباله‌ی بازنشناختنیِ موردنظر را (به صورت موضعی حول هر فرمول)
تحت کنترل بیشتری درخواهیم آورد. 
\footnote{جزوه‌ی «سادگی به زبان ساده»‌ تألیف نویسنده‌ی دوم را 
برای دانستن  تفاوت دنباله‌های حاصل از لم رمزی و لم اردوش رادو مطالعه بفرمایید. 
}
\begin{thm}
\label{indis2}
فرض کنیم
$I$
مجموعه‌ای باشد مرتب خطی. در آن صورت مدلِ
$\mathfrak{M}\models T$
و در آن دنباله‌ای بازنشناختنی چون
$(a_i)_{i\in I}$
موجودند. 
\end{thm}
\begin{proof}
نخست بسط زبانیِ
$L'=L\cup \{c_i\}_{i\in I}$
را از
$L$
توسط ثوابت جدید در نظر بگیرید. تئوریِ
زیر را در نظر بگیرید:
\[
T'=T\cup \{\text{$(c_i)_{i\in I}$ دنباله‌ای بازنشناختنی است}\}
\]
به بیان دقیقتر،
$T'$
از اجتماعِ
$T$
با مجموعه‌های زیر از جملات حاصل شده است:
\begin{align*}
& \{\phi(c_i)\leftrightarrow \phi(c_j) \}_{i,j\in I, \phi(x)\in L}
\\
& \vdots
\\
& \{\phi(c_{i1},\ldots, c_{in})\leftrightarrow \phi(c_{j1},\ldots,c_{jn})\}_{\phi(x_1,\ldots,x_n)\in L, i1<\ldots<i_n, j1<\ldots<jn\in I}
\end{align*}
کافی است نشان دهیم که
$T'$
دارای مدل است، و برای آن کافی است مدل داشتن هر بخش متناهی از
$T'$
را ثابت کنیم. هر بخش متناهی از 
$T'$
را می‌توان به مجموعه‌ای از جملات گستراند که بیانگرِ 
$\Delta$
بازنشناختنی بودن یک دنباله‌ی متناهی، برای یک مجموعه‌ی متناهیِ
$\Delta$
از فرمولها هستند. گزاره‌ی
\ref{indis1}
مدل مورد نظر را فراهم می‌آورد.
\end{proof}
\newpage 
\section{جلسه‌ی شانزدهم، توابع اسکولمی و اِسْکولِمیزه کردن}
فرض کنید که
$\mathfrak{M}$
یک 
$L$
ساختار باشد و 
$A$
زیرمجموعه‌ای از آن. می‌دانیم که 
$\langle A\rangle$،
زیرساختار تولیدشده توسطِ 
$A$،
مجموعه‌ی متشکل از همه‌ی
$t^\mathfrak{M}(a_1,\ldots,a_n)$
هاست. این مجموعه،‌ لزوماً یک زیرساخت مقدماتی نیست؛ برای مثال
زیرساخت تولید شده توسطِ
$1$
در ساختارِ
$\langle \mathbb{Q},+,-,\cdot\rangle$
برابر است با
$\langle \mathbb{Z},+,-,\cdot\rangle$.
با این همه، در مثال یادشده، اگر در زبان توابعِ
$f_{\frac{1}{n}}(x)=\frac{x}{n}$
را می‌داشتیم، آنگاه زیرساخت تولیدشده توسط عنصر
$1$
برابر می‌شد با خودِ
$\mathbb{Q},+,-,\cdot\rangle$، 
که مسلماً زیرساختی مقدماتی از ساختار یادشده است.
\par 
بنا به لمِ تارسکی، اگر
$\mathfrak{M}\subseteq \mathfrak{N}$
آنگاه
$\mathfrak{M}\prec \mathfrak{N}$
اگروتنهااگر برای هر فرمولِ بدونِ سور
$\phi(x,\bar{a})\in L_M$،
اگر
$\mathfrak{N}\models \exists x \quad \phi(x,\bar{a})$
آنگاه
\[
\mathfrak{N}\models \exists x\in M\quad \phi(x,\bar{a}).
\]
اگر ترمهایی  مانند
$t$
در زبان داشتیم، چنانکه از
\[
\mathfrak{N}\models \exists x \quad \phi(x,\bar{a})
\]
نتیجه می‌شد
\[
\mathfrak{N}\models \phi(t(\bar{a}),\bar{a}),
\]
آنگاه دو ساختار مورد نظر، لوازم لم تارسکی را می‌داشتند. 
\begin{defn}[ویژگی اسکولم]
گوئیم در
تئوری
$T$
توابع اسکولم تعبیه شده‌اند 
\LTRfootnote{$T$ has built-in Skolem functions.}،
هرگاه برای هر فرمولِ
$\phi(x,\bar{y})$
ترمِ
$t_\phi(\bar{y})$
چنان موجود باشد که 
\[
T\models \forall \bar{y}\quad \left( \exists x\quad \phi(x,\bar{y})\to \phi(t_\phi(\bar{y}),\bar{y})\right).
\]
\end{defn}
توجه کنید که اگر
$|\bar{y}|=0$
آنگاه ترم مورد نظر باید یک ثابت باشد؛ یعنی
\[
T\models \exists x \phi(x)\to \phi(c_\phi).
\]
\begin{prop}
اگر 
$T$
یک تئوری سازگار 
در زبانِ
$L$
باشد، 
زبانِ
$L'$
شامل
$L$
و تئوری
$T'$
در آن شاملِ
$T$
چنان موجودند که 
$T'$
دارای توابع اسکولمیِ	 تعبیه‌شده است.
\end{prop}
\begin{proof}
قرار دهید
$L_0=L$
و
$T_0=T$
و فرض کنید
$L_1$
زبانی باشد که در آن برای هر
$L_0$
فرمولِ
بدون سورِ
$\phi(x,\bar{y})$
یک نماد تابعیِ
$f_\phi$
داریم. نیز تئوریِ
$T_1$
را اجتماعِ
$T_0$
بگیرید با همه‌ی جمله‌های
\[
\forall \bar{y}\left(
\exists x \phi(x,\bar{y})\to \phi(f_\phi(\bar{y}),\bar{y})\right)
\]
که در آن
$\exists x \phi(x,\bar{y})\in T_0$.
تئوریِ
$T_1$
دارای مدل است؛ برای تعبیر تابعِ
$f$
کافی است 
هر
$f_\phi(\bar{y})$
را عنصری بگیریم که ضامنِ
$\exists x\quad \phi(x,\bar{y})$
باشد. بدین ترتیب، 
زبانِ
$L_{n+1}$
را زبانی می‌گیریم که از اجتماعِ
$L_n$
با نمادهای تابعیِ
$f_\phi$
برای هر
$\phi(x,\bar{y})\in L_n$
حاصل شده است و فرض می‌کنیم تئوریِ
$T_{n+1}$
از اجتماعِ
$T_n$
با جملات زیر حاصل شده باشد:
\[
\forall\bar{y} \left(\exists x \phi(x,\bar{y})\to \phi(f_\phi(\bar{y},\bar{y})\right).
\]
برای هر
$\phi(x,\bar{y})\in L_n$.
تئوریِ
$T_\omega=\bigcup_{i<\omega} T_i$
در زبانِ
$L_\omega=\bigcup_{i<\omega}L_i$
 تئوریِ مورد نظر ماست. 
\end{proof}
\begin{defn}
تئوریِ
$T_\omega$
را
اسکولمیزشِ ( یا اسکولمیزه‌شده‌ی)
\LTRfootnote{Skolemization}
$T$
می‌خوانیم و آن را با
$T_{skolem}$
نشان می‌دهیم.
\end{defn}
\begin{tam}
\hfill
\begin{enumerate}
\item 
نشان دهید که 
$T_{skolem}$
سورها را حذف می‌کند.
\item
نشان دهید که به هنگِ
$T_{skolem}$
همه‌ی جمله‌ها دارای معادل عمومیند (معادلی تنها دارای سور عمومی). به طور خاص، این تئوری دارای 
 اصل‌بندی عمومی است.
\item 
با استفاده اسکولمیزه‌سازی، و بدینسان تقلیل منطق مرتبه‌ی اول به منطق گزاره‌ها، اثباتی توپولوژیک برای قضیه‌ی فشردگی ارائه کنید.
\end{enumerate}
\end{tam}
فرض کنیم که تئوریِ
$T$
دارای توابع اسکولمی باشد. دیدیم که برای هر مجموعه‌ی مرتبِ خطیِ
$\langle I,\leq\rangle$
می‌توان دنباله‌ای بازنشناختنی چون
$(a_i)_{i\in \omega}$
در مدلی از
$T$
یافت. مدل تولیدشده‌ توسطِ
$a_i$
ها را با 
$S_{EM}(a_i|i\in I)$
نشان می‌دهیم و آن را پوش اسکولمی ِ
(یا غلاف اسکولمیِ)
\LTRfootnote{Skolem hull}
این دنباله می‌خوانیم. (با توجه به نقش توابع اسکولمی نشان دهید که) داریم
\[
S_{EM}(a_i|i\in I)\prec \mathfrak{M}
\]
و به ويژه
\[
S_{EM}(a_i|i\in I)\models T.
\]
\begin{tam}
فرض کنید
$f:I\to I$
یک اتومرفیسم ترتیبی باشد. نشان دهید که نگاشتِ
$\hat{f}:S_{EM}(a_i|i\in I)\to S_{EM}(a_i|i\in I)$
با ضابطه‌ی
\mbox{
$\hat{f}(t(a_{i_1},\ldots,a_{i_n})=
\hat{f}(t(a_{f(i_1)},\ldots,a_{f(i_n}))$}
یک اتومرفیسم است. 
\end{tam}
\newpage 
\section{جلسه‌ی هفدهم}
یکی از 
سودمندیهای 
مدلهای اهغن‌فُیشت موستفسکیِ
تولیدشده توسط دنباله‌های بازنشناختنی، 
که در اثباتهای بعدی بسیار به کارمان خواهد آمد،
این است
که در آنها تایپهای زیادی برآورده نمی‌شوند.
\begin{lem}
فرض کنید که 
$I$
مجموعه‌ای خوشترتیب باشد،
$(a_i)_{i\in I}$
دنباله‌ای بازنشناختنی، و
$\mathfrak{M}$
مدلِ اسکولمی تولیدشده توسط دنباله‌ی یادشده. برای هر مجموعه‌ی شمارای
$A\subseteq M$،
حداکثر تعدادِ شمارا تایپِ‌روی
$A$،
در
$M$
محقق می‌شوند.
\end{lem}
\begin{proof}
نخست لم بالا را در حالت خاصِ
$A=(a_i)_{i\in I_0}\subseteq (a_i)_{i\in I}$
اثبات می‌کنیم. نیز نخست ادعا می‌کنیم که در این حالت
\[
|\{\tp^\mathfrak{M}(a_i/A)|i\in I\}|\leq \aleph_0.
\]
توجه کنید که  اگر
$i,j\in I_0$
 وضعیت ترتیبیِ یکسانی نسبت به
$I_0$
داشته باشند، 
آنگاه
$a_i,a_j$
تایپ یکسانی روی
$A$
دارند؛ به بیان دیگر
اگر
$\qftp_{DLO}(i/I_0)=\qftp_{DLO}(j/I_0)$
آنگاه
$\tp^\mathfrak{M}(a_i/A)=\tp^\mathfrak{M}(a_j/A)$.
علت این امر ساده است؛ برای هر
$i_0,\ldots,i_k\in I_0$
اگر وضع ترتیبیِ
$i,j$ 
نسبت به این دنباله یکسان باشد، آنگاه 
وضعیت ترتیبی دو دنباله‌ی
$i,i_0,\ldots,i_k$
و
$j,i_0,\ldots,i_k$،
نیز یکسان است؛ که این 
 بنا به بازنشناختنی بودنِ دنباله‌ی
$(a_i)_{i\in I}$
نتیجه می‌دهد که
\[
\mathfrak{
M}\models \phi(a_i,a_{i_0},\ldots,a_{i_k})
\leftrightarrow 
\phi(a_j,a_{i_0},\ldots,a_{i_k})
\]
حالات مخلتفی که 
$i$
از لحاظ ترتیبی نسبت به
$I_0$
می‌تواند داشته باشد، به صورت زیر است.
\begin{itemize}
\item 
در
$I_0$
باشد (شمارا حالت).
\item 
از تمام عناصرِ
$I_0$
بزرگتر باشد.
(یک حالت)
\item 
از تمام عناصرِ
$I_0$
کوچکتر باشد (یک حالت).
\item
از برخی از عناصرِ
$I_0$
کوچکتر باشد و از برخی دیگر بزرگتر (ادعا: شماراحالت)
\end{itemize}
حال به محاسبه‌ی تعداد حالات در مورد آخر می‌پردازیم. طبیعتاً تعداد آن حالات برابر است با تعداد شکافها در مجموعه‌ی
$I_0$.
شهود ما عموماً ما را بدین تصور وامی‌دارد که تعداد شکافها در یک  مجموعه‌ی مرتبِ‌ شمارا، ناشماراست. این شهود در اینجا کار نمی‌کند و
فرض خوشترتیب بودن مجموعه‌ی
$I$
در اینجا به کار می‌آید.
\par 
قرار دهید
$I^i=\{k\in I_0|k>i\}$.
بنا به خوشترتیبی،
$\min I^i$
موجود است. برای هر
$i,j\in I$
داریم
$I^i=I^j$
اگروتنهااگر
$\min I^i=\min I^j$.
پس تعداد شکافهای اینچنین، بنا به شمارا بودنِ
$I_0$
شماراست.
\par 
تا اینجا ثابت کرده‌ایم که تعداد تایپهای
تک متغیره‌ی به صورت
$\tp^\mathfrak{M}(a_i/A)$
حداکثر شماراست. همین گفته درباره‌ی تایپهای
$n$
متغیره نیز برقرار است. تعداد این تایپها نیز برابر با تعداد
حالات ترتیبیِ مجموعه‌های
$i_0,\ldots,i_n$
از اعضای
$I$
است نسبت به
$I_0$.
به بیان دیگر اگر
$i_0,\ldots,i_n$
و
$j_0,\ldots,j_n$
وضعیت ترتیبی یکسانی نسبت به
$I_0$
داشته باشند آنگاه
\[\tp^\mathfrak{M}(a_{i_0},\ldots,a_{i_n})=\tp^\mathfrak{M}(a_{j_0},\ldots,a_{j_n}).\]
تعداد حالات ترتیبی متصوَر برای
$i_0,\ldots,i_n$
نسبت به
$I_0$
نیز شماراست؛ زیرا تعداد حالات ترتیبی آنها نسبت به هم متناهی است، و تعداد حالات ترتیبی هر یک نسبت به 
$I_0$
شماراست.
پس تا اینجا ثابت کرده‌ایم که تعداد تایپهای به شکلِ
$\tp^\mathfrak{M}(a_{i_0},\ldots,a_{i_n})/A)$
شماراست.
\par 
می‌خواستیم تعداد تایپهای برآورده شونده در
$M$
را بیابیم. می‌دانیم که عناصر 
$M$
به شکل زیر هستند:
\[
M=\{t^\mathfrak{M}(a_{i_0},\ldots,a_{i_n})|i_0,\ldots,i_n\in I, \text{ $t$ ترمی اسکولمی }\}
\]
دوباره معلوم است که اگر
$\qftp_{DLO}(i_0,\ldots,i_n/I_0)=\qftp_{DLO}(j_0,\ldots,j_n/I_0)$
آنگاه 
\[
\tp^\mathfrak{M}(t^\mathfrak{M}(a_{i_0},\ldots,a_{i_n})/A)=
\tp^\mathfrak{M}(t^\mathfrak{M}(a_{j_0},\ldots,a_{j_n})/A)
\]
پس تعداد تایپها روی
$A$
در حالتی که
$A\subseteq (a_i)_{i\in I}$
شماراست. 
\par 
همان بحث بالا برای اثبات این که تعداد تایپهای روی هر
$A\subseteq M$
شماراست کار می‌کند. 
فرض کنیم
\[
A=\{t^\mathfrak{M}(a_{i_0},\ldots,a_{i_k})|i_0,\ldots,i_k\in I_0, t\in T\}
\]
که در آن
$T$
مجموعه‌ای است از ترمها و
$I_0\subseteq I$.
تعداد تایپهای 
برآورده شونده در
$M$
روی
$A$
کمتر یا مساوی تعداد تایپها روی مجموعه‌ی
$A'=\{a_{i}|i\in I_0\}$
است؛ و آنْ شماراست. 
\end{proof}
\section*{بحث جدید، قضیه‌ی مُرلی}
همچنان تئوریِ
$T$
را شمارا، کامل و فاقد مدل متناهی فرض کرده‌ایم. فرض کنید
$\kappa>\aleph_0$
یک کاردینال نامتناهی باشد.
\begin{defn}
تئوریِ
$T$
را 
\textbf{$\kappa$
جازم 
}
\LTRfootnote{$\kappa$-categorical}
می‌خوانیم هرگاه هر دو مدلِ آن از اندازه‌ی
$\kappa$
با هم ایزومرف باشند. در حالتی که
$\kappa=\aleph_0$
تئوری دارای شرط یادشده را 
$\aleph_0$
جازم می‌خوانیم.
تئوریِ
$T$
را به
\textbf{ جازم‌درکاردینالی‌نامتناهی}
 می‌خوانیم هرگاه 
در یک کاردینالِ
ناشمارای
$\kappa$
جازم باشد. 
\end{defn}
همانگونه که بارها گفته‌ایم، هدف نهایی این درس اثبات قضیه‌ی زیر است:
\begin{prop}[مُرلی]
اگر
$T$
جازم‌در‌کاردینالی‌ناشمارا باشد،
آنگاه در تمامِ 
کاردینالهای
$\kappa\geq \aleph_0$،
جازم است.
\end{prop}
\begin{mesal}
فرض کنید
$V$
یک فضای برداری باشد روی یک میدان شمارا.
اگر
$V$
دارای یک پایه‌ی متناهی یا شمارا باشد، آنگاه
$V$
شماراست.
از این رو، دو فضای برداری شمارا لزوماً با هم ایزومرف نیستند (شاید یکی دارای بعد 
$n$
و دیگری دارای بعد
$m\not=n$
باشد).
 برای این که اندازه‌ی
$V$
ناشمارا شود، نیازمند پایه‌ای ناشمارا هستیم. در واقع
\[
|V|=\aleph_0\times \dim(V)=\max\{\aleph_0,\dim(V)\}.
\]
از طرفی، هر دو فضای برداریِ دارای بُعدِ مساوی با هم ایزومرفند. پس هر دو فضای برداریِ از اندازه‌ی
$\kappa>\aleph_0$
از آنجا  که همبُعدند با هم ایزومرفند.
\end{mesal}

\end{document}

